\chapter{Wymagania projektowe}
\section{Sformułowanie problemu}
\section{Pozycjonowanie produktu}
\section{Opis udziałowców i użytkowników}
\subsection{Podsumowanie udziałowców}
\subsection{Podsumowanie użytkowników}

\begin{minipage}{\textwidth}
    \begin{table}[H]
        \centering\caption{Użytkownicy (opr.wł)\label{tabela:uzytkownicy}}
        \begin{tabular}{|p{.2\textwidth}|p{.25\textwidth}|p{.45\textwidth}|}

            \hline
            Nazwa & Opis & Odpowiedzialności\\

            \hline
            Administrator &
            Osoba zarządzająca działaniem aplikacji &
            \begin{itemize}
                \item Przydzielanie i odbieranie użytkownikom uprawnień
            \end{itemize} \\
            \hline
            Dietetyk &
            Specjalista w dziedzinie dietetyki &
            \begin{itemize}
                \item Używa założonego konta
                \item Wprowadza, edytuje i usuwa produkty, przepisy i jadłospisy
            \end{itemize} \\
            \hline
            Pacjent &
            Klient dietetyka &
            \begin{itemize}
                \item Otrzymuje (mailowo) ułożony jadłospis
            \end{itemize} \\
            \hline
        \end{tabular}
    \end{table}
\end{minipage}

\section{Słownik}
\begin{itemize}
    \item Administrator – użytkownik posiadający uprawnienia do zarządzania uprawnieniami użytkowników
    \item Dietetyk – specjalista w dziedzinie dietetyki
    \item Jadłospis – plan posiłków zdefiniowany na określoną liczbę dni z uwzględnieniem określonych wymagań
    \item Karta pacjenta – karta przedstawiająca przebieg współpracy dietetyka z pacjentem
    \item Miara domowa – definicja pospolitej miary, takiej jak np. łyżeczka w gramach
    \item Pacjent – klient dietetyka
    \item Pomiary ciała – pomiary ciała pacjenta przeprowadzane przez dietetyka
    \item Posiłek – posiłek jest przydzielany do jadłospisu; zawiera produkty i przepisy
    \item Produkt – produkt spożywczy, dla którego specyfikowane są wartości odżywcze i miary domowe
    \item Przepis – opis składników i kroków przygotowania dania
    \item Sekcja przepisu – semantyczny podział przepisu, np. sernik może mieć sekcje związane z przygotowaniem ciasta, nadzienia i polewy
    \item Wartość odżywcza – ilość elementu takiego jak np. węglowodanów albo białka w 100g produktu
    \item Wizyta – konkretna wizyta pacjenta
    \item Wywiad żywieniowy – wywiad przeprowadzany z pacjentem uwzględniający jego nawyki żywieniowe, nietolerancje, choroby, przyjmowane leki, itp.
\end{itemize}
\thispagestyle{normal}
