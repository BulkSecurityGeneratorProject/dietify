\chapter{Stan wiedzy i techniki w zakresie tematyki pracy}
\section{Przegląd istniejących rozwiązań konkurencyjnych}
\todo{opisać rozwiązania konkurencyjne}
\begin{itemize}
    \item Dietico
    \item TiqDiet
    \item Kcalmar PRO
    \item Dietetyk Pro
\end{itemize}

\begin{minipage}{\textwidth}
    \begin{table}[H]
        \centering\caption{Rozwiązania konkurencyjne - cechy funkcjonalne (opr.wł)\label{tabela:rozwiazania-konkurencyjne-funkcjonalne}}
        \begin{tabular}{|P{.22\textwidth}|P{.09\textwidth}|P{.09\textwidth}|P{.09\textwidth}|P{.09\textwidth}|P{.09\textwidth}|P{.09\textwidth}|}
            \hline
                                                & Tiqdiet   & Kcalmar Pro   & Dietetyk Pro & Aliant         & Dietico   & Vitme   \\ \hline
            Tworzenie jadłospisów               & TAK       & TAK           & TAK          & TAK            & TAK       & TAK     \\ \hline
            Gotowe szablony diet                & TAK       & TAK           & TAK          & TAK            & NIE       & NIE     \\ \hline
            Zapis diety do pliku                & TAK       & TAK           & TAK          & TAK            & TAK       & TAK     \\ \hline
            Wysyłanie diet do pacjenta          & TAK       & TAK           & TAK          & TAK            & NIE       & TAK     \\ \hline
            Komunikacja z pacjentem             & TAK       & TAK           & TAK          & NIE            & NIE       & TAK     \\ \hline
            Karta pacjenta                      & TAK       & TAK           & TAK          & TAK            & TAK       & TAK     \\ \hline
            Wywiad żywieniowy                   & TAK       & TAK           & TAK          & NIE            & TAK       & TAK     \\ \hline
            Lista zakupów                       & TAK       & TAK           & TAK          & TAK            & TAK       & NIE     \\ \hline
            Dodawanie własnych produktów        & TAK       & TAK           & TAK          & TAK            & TAK       & TAK     \\ \hline
            Dodawanie własnych przepisów        & TAK       & TAK           & TAK          & TAK            & TAK       & TAK     \\ \hline
        \end{tabular}
    \end{table}
\end{minipage}

\begin{minipage}{\textwidth}
    \begin{table}[H]
        \centering\caption{Rozwiązania konkurencyjne - cechy niefunkcjonalne (opr.wł)\label{tabela:rozwiazania-konkurencyjne-niefunkcjonalne}}
        \begin{tabular}{|P{.22\textwidth}|P{.09\textwidth}|P{.09\textwidth}|P{.09\textwidth}|P{.09\textwidth}|P{.09\textwidth}|P{.09\textwidth}|}
            \hline
                                                & Tiqdiet   & Kcalmar Pro   & Dietetyk Pro & Aliant         & Dietico   & Vitme   \\ \hline
            Liczba produktów w bazie            & 1000      & 1400          & 6000         & 3500           & 900       & 5000    \\ \hline
            Liczba gotowych przepisów           & 200       & 800           & 2800         & 1700           & 1900      & 400     \\ \hline
            Praca offline                       & NIE       & NIE           & NIE          & TAK            & NIE       & NIE     \\ \hline
            Praca online                        & TAK       & TAK           & TAK          & NIE            & TAK       & TAK     \\ \hline
            Aplikacja mobilna dla dietetyka     & TAK       & TAK           & TAK          & NIE            & NIE       & NIE     \\ \hline
            Aplikacja mobilna dla pacjenta      & TAK       & TAK           & NIE          & NIE            & NIE       & TAK     \\ \hline
            Dostęp dla pacjenta przez przeglądarkę internetową       & TAK       & TAK           & TAK          & NIE            & NIE       & TAK     \\ \hline
            Darmowy okres testowy               & 14dni     & 14dni         & 7dni         & bezter- minowo & 14dni     & 14dni   \\ \hline
            Cena w abonamencie rocznym          & 199       & 1188          & 246          & 699            & 546       & 219     \\ \hline
        \end{tabular}
    \end{table}
\end{minipage}

\section{Przegląd przydatnych technologii i technik}
\todo{Tutaj opisać architekturę aplikacji webowych. Porównać monolit, soa i mikroserwisy}

\thispagestyle{normal}
