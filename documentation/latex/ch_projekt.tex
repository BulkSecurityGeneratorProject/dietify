% !TeX spellcheck = pl_PL
\chapter{Projekt}\label{ch:project}

\section{Przypadki użycia}\label{sec:usecase}
\todo{scenariusze przypadków użycia}

\image{0.55}{../uml/use_case_diagrams/users.png}{Diagram przypadków użycia - użytkownicy}{use-case-diagram:users}
\image{0.55}{../uml/use_case_diagrams/gateway.png}{Diagram przypadków użycia - brama aplikacji}{use-case-diagram:gateway}
\image{0.55}{../uml/use_case_diagrams/products.png}{Diagram przypadków użycia - produkty}{use-case-diagram:products}
\image{0.55}{../uml/use_case_diagrams/recipes.png}{Diagram przypadków użycia - przepisy}{use-case-diagram:recipes}
\image{0.55}{../uml/use_case_diagrams/mealplans.png}{Diagram przypadków użycia - jadłospisy}{use-case-diagram:mealplans}
\image{0.55}{../uml/use_case_diagrams/appointments.png}{Diagram przypadków użycia - wizyty}{use-case-diagram:appointments}

\section{Prototyp interfejsu}\label{sec:mockups}
\noindent
\todo{opisać mockupy}
\image{0.55}{../mockup/0home.png}{Prototyp interfejsu - strona startowa}{mockup:0home}
\image{0.55}{../mockup/0home_1sign-up.png}{Prototyp interfejsu - rejestrowanie do systemu}{mockup:0home_1sign-up}
\image{0.55}{../mockup/0home_2login.png}{Prototyp interfejsu - logowanie do systemu}{mockup:0home_2login}
\image{0.55}{../mockup/1products.png}{Prototyp interfejsu - lista produktów}{mockup:1products}
\image{0.55}{../mockup/1products_1new.png}{Prototyp interfejsu - dodawanie nowego lub edycja istniejącego produktu}{mockup:1products_1new}
\image{0.55}{../mockup/1products_2details.png}{Prototyp interfejsu - szczegóły produktu}{mockup:1products_2details}
\image{0.55}{../mockup/2recipes.png}{Prototyp interfejsu - lista przepisów}{mockup:2recipes}
\image{0.55}{../mockup/2recipes_1new.png}{Prototyp interfejsu - dodawanie nowego lub edycja istniejącego przepisu}{mockup:2recipes_1new}
\image{0.55}{../mockup/2recipes_1new_1add-product.png}{Prototyp interfejsu - dodawanie produktu do przepisu}{mockup:2recipes_1new_1add-product}
\image{0.55}{../mockup/2recipes_2details.png}{Prototyp interfejsu - szczegóły przepisu}{mockup:2recipes_2details}
\image{0.55}{../mockup/3mealplans.png}{Prototyp interfejsu - lista jadłospisów}{mockup:3mealplans}
\image{0.55}{../mockup/3mealplans_1new_1settings.png}{Prototyp interfejsu - dodawanie nowego lub edycja istniejącego jadłospisu - zakładka ustawień}{mockup:3mealplans_1new_1settings}
\image{0.55}{../mockup/3mealplans_1new_2calendar.png}{Prototyp interfejsu - dodawanie nowego lub edycja istniejącego jadłospisu - zakładka kalendarza}{mockup:3mealplans_1new_2calendar}
\image{0.55}{../mockup/3mealplans_1new_2calendar_1meal.png}{Prototyp interfejsu - zarządzanie posiłkiem w~edytowanym jadłospisie}{mockup:3mealplans_1new_2calendar_1meal}
\image{0.55}{../mockup/3mealplans_1new_2calendar_2add-product.png}{Prototyp interfejsu - dodawanie produktu do posiłku w~jadłospisie}{mockup:3mealplans_1new_2calendar_2add-product}
\image{0.55}{../mockup/3mealplans_1new_2calendar_3add-recipe.png}{Prototyp interfejsu - dodawanie przepisu do posiłku w~jadłospisie}{mockup:3mealplans_1new_2calendar_3add-recipe}
\image{0.55}{../mockup/3mealplans_2details_1settings.png}{Prototyp interfejsu - szczegóły jadłospisu - zakładka ustawień}{mockup:3mealplans_2details_1settings}
\image{0.55}{../mockup/3mealplans_2details_2calendar.png}{Prototyp interfejsu - szczegóły jadłospisu - zakładka kalendarza}{mockup:3mealplans_2details_2calendar}
\image{0.55}{../mockup/3mealplans_2details_2calendar_1meal.png}{Prototyp interfejsu - szczegóły posiłku w~jadłospisie}{mockup:3mealplans_2details_2calendar_1meal}
\image{0.55}{../mockup/4appointments.png}{Prototyp interfejsu - lista wizyt}{mockup:4appointments}
\image{0.55}{../mockup/4appointments_1new-patient-card.png}{Prototyp interfejsu - dodawanie nowej karty pacjenta}{mockup:4appointments_1new-patient-card}
\image{0.55}{../mockup/4appointments_2patient-card-details.png}{Prototyp interfejsu - szczegóły karty pacjenta}{mockup:4appointments_2patient-card-details}
\image{0.55}{../mockup/4appointments_3new-appointment.png}{Prototyp interfejsu - dodawanie nowej wizyty}{mockup:4appointments_3new-appointment}
\image{0.55}{../mockup/4appointments_4appointment-details.png}{Prototyp interfejsu - szczegóły wizyty}{mockup:4appointments_4appointment-details}
\image{0.55}{../mockup/4appointments_4appointment-details_1nutritional-interview.png}{Prototyp interfejsu - wizyta - wywiad żywieniowy}{mockup:4appointments_4appointment-details_1nutritional-interview}
\image{0.55}{../mockup/4appointments_4appointment-details_2body-measurement.png}{Prototyp interfejsu - wizyta - pomiary ciała}{mockup:4appointments_4appointment-details_2body-measurement}
\image{0.55}{../mockup/4appointments_4appointment-details_3mealplan.png}{Prototyp interfejsu - wizyta - jadłospis}{mockup:4appointments_4appointment-details_3mealplan}
\image{0.55}{../mockup/5administration.png}{Prototyp interfejsu - widok administratora}{mockup:5administration}
\image{0.4}{../mockup/6mobile.png}{Prototyp interfejsu - układ strony na urządzeniu mobilnym}{mockup:6mobile}

\section{Kategorie}\label{sec:categories}

\begin{enumerate}[label={\textbf{KAT/\protect\threedigits{\theenumi}}}, wide, labelwidth=!, labelindent=0pt, labelsep=0pt, series=reqs]
    \setlength\itemsep{1em}
    \req{User}\label{kat:User} (Użytkownik)

    \textbf{Opis}: Konto użytkownika aplikacji. Każdy zalogowany użytkownik musi mieć konto użytkownika
    \par
    \textbf{Atrybuty}:
    \begin{itemize}[series=atr, wide, align=left, leftmargin=190pt]
        \atr{id}\label{kat:User:id}- identyfikator
        \atr{login}\label{kat:User:login}- login użytkownika
        \atr{passwordHash}\label{kat:User:passwordHash}- reprezentacja hasła stworzona przez nałożenie na hasło funkcji skrótu
        \atr{firstName}\label{kat:User:firstName}- imię użytkownika
        \atr{lastName}\label{kat:User:lastName}- nazwisko użytkownika
        \atr{email}\label{kat:User:email}- adres e-mail
        \atr{image}\label{kat:User:image}- zdjęcie profilowe użytkownika
        \atr{activated}\label{kat:User:activated}- flaga pokazująca czy konto użytkownika zostało aktywowane
        \atr{language}\label{kat:User:language}- język użytkownika w~postaci kodu ISO 639-1
        \atr{activationKey}\label{kat:User:activationKey}- klucz wymagany podczas aktywacji konta użytkownika
        \atr{resetKey}\label{kat:User:resetKey}- klucz wymagany podczas resetowania hasła do konta użytkownika
        \atr{createdDate}\label{kat:User:createdDate}- data utworzenia konta
        \atr{resetDate}\label{kat:User:resetDate}- data ostatniego resetowania hasła do konta
        \atr{lastModifiedDate}\label{kat:User:lastModifiedDate}- data ostatniej modyfikacji konta
    \end{itemize}

    \req{Authority}\label{kat:Authority} (Rola)

    \textbf{Opis}: Rola użytkownika od której zależy zakres uprawnień użytkownika
    \par
    \textbf{Atrybuty}:
    \begin{itemize}[series=atr, wide, align=left, leftmargin=190pt]
        \atr{name}\label{kat:Authority:name}- nazwa roli
    \end{itemize}

    \req{UserExtraInfo}\label{kat:UserExtraInfo} (Dodatkowe Informacje Użytkownika)

    \textbf{Opis}: Dodatkowe informacje o~użytkowniku
    \par
    \textbf{Atrybuty}:
    \begin{itemize}[series=atr, wide, align=left, leftmargin=190pt]
        \atr{id}\label{kat:UserExtraInfo:id}- identyfikator
        \atr{gender}\label{kat:UserExtraInfo:gender}- płeć
        \atr{dateOfBirth}\label{kat:UserExtraInfo:dateOfBirth}- data urodzenia
        \atr{phoneNumber}\label{kat:UserExtraInfo:phoneNumber}- numer telefonu, najlepiej w~formacie (+00) 000-000-000
        \atr{streetAddress}\label{kat:UserExtraInfo:streetAddress}- adres zamieszkania
        \atr{postalCode}\label{kat:UserExtraInfo:postalCode}- kod pocztowy
        \atr{city}\label{kat:UserExtraInfo:city}- miasto
        \atr{country}\label{kat:UserExtraInfo:country}- państwo
        \atr{personalDescription}\label{kat:UserExtraInfo:personalDescription}- krótki opis osobisty. W~przypadku dietetyka może zawierać dodatkowe informacje o~prowadzonej praktyce dietetycznej
    \end{itemize}

    \req{SiteContent}\label{kat:SiteContent} (Treść Strony)

    \textbf{Opis}: Treść strony definiowana przez administratora
    \par
    \textbf{Atrybuty}:
    \begin{itemize}[series=atr, wide, align=left, leftmargin=190pt]
        \atr{id}\label{kat:SiteContent:id}- identyfikator
        \atr{ordinalNumber}\label{kat:SiteContent:ordinalNumber}- numer porządkowy definiujący kolejność w~jakim dane powinny być wyświetlane
        \atr{siteContentType}\label{kat:SiteContent:siteContentType}- typ treści strony
        \atr{title}\label{kat:SiteContent:title}- tytuł treści stron
        \atr{description}\label{kat:SiteContent:description}- opis treści strony
        \atr{image}\label{kat:SiteContent:image}- opcjonalny obrazek, który powinien zostać wyświetlony obok treści strony
    \end{itemize}

    \req{SiteContentTranslation}\label{kat:SiteContentTranslation} (Tłumaczenie Treści Strony)

    \textbf{Opis}: Tłumaczenie treści strony
    \par
    \textbf{Atrybuty}:
    \begin{itemize}[series=atr, wide, align=left, leftmargin=190pt]
        \atr{id}\label{kat:SiteContentTranslation:id}- identyfikator
        \atr{title}\label{kat:SiteContentTranslation:title}- tłumaczenie tytułu
        \atr{description}\label{kat:SiteContentTranslation:description}- tłumaczenie opisu
        \atr{language}\label{kat:SiteContentTranslation:language}- język tłumaczenia w~postaci kodu ISO 639-1
    \end{itemize}

    \req{ContactInfo}\label{kat:ContactInfo} (Informacje Kontaktowe)

    \textbf{Opis}: Informacje kontaktowe witryny
    \par
    \textbf{Atrybuty}:
    \begin{itemize}[series=atr, wide, align=left, leftmargin=190pt]
        \atr{id}\label{kat:ContactInfo:id}- identyfikator
        \atr{contactInfoType}\label{kat:ContactInfo:contactInfoType}- typ informacji kontaktowej
        \atr{description}\label{kat:ContactInfo:description}- opis
    \end{itemize}

    \req{Pricing}\label{kat:Pricing} (Cennik)

    \textbf{Opis}: Pozycja cennika dostępu do usług oferowanych przez system
    \par
    \textbf{Atrybuty}:
    \begin{itemize}[series=atr, wide, align=left, leftmargin=190pt]
        \atr{id}\label{kat:Pricing:id}- identyfikator
        \atr{ordinalNumber}\label{kat:Pricing:ordinalNumber}- numer porządkowy warunkujący kolejność wyświetlania pozycji w~cenniku
        \atr{title}\label{kat:Pricing:title}- nazwa pozycji cennika
        \atr{description}\label{kat:Pricing:description}- opis pozycji cennika
        \atr{price}\label{kat:Pricing:price}- cena
        \atr{currency}\label{kat:Pricing:currency}- waluta przedstawiona jako kod ISO 4217
    \end{itemize}

    \req{PricingTranslation}\label{kat:PricingTranslation} (Tłumaczenie Cennika)

    \textbf{Opis}: Tłumaczenie pozycji cennika
    \par
    \textbf{Atrybuty}:
    \begin{itemize}[series=atr, wide, align=left, leftmargin=190pt]
        \atr{id}\label{kat:PricingTranslation:id}- identyfikator
        \atr{title}\label{kat:PricingTranslation:title}- tłumaczenie nazwy
        \atr{description}\label{kat:PricingTranslation:description}- tłumaczenie opisu
        \atr{language}\label{kat:PricingTranslation:language}- język tłumaczenia w~postaci kodu ISO 639-1
    \end{itemize}

    \req{Product}\label{kat:Product} (Produkt)

    \textbf{Opis}: Produkt żywieniowy
    \par
    \textbf{Atrybuty}:
    \begin{itemize}[series=atr, wide, align=left, leftmargin=190pt]
        \atr{id}\label{kat:Product:id}- identyfikator
        \atr{source}\label{kat:Product:source}- źródło produktu, jeśli produkt jest importowany; preferowany format adresu URL
        \atr{isPublic}\label{kat:Product:isPublic}- flaga określająca czy produkt jest widoczny publicznie
        \atr{language}\label{kat:Product:language}- język tłumaczenia w~postaci kodu ISO 639-1
    \end{itemize}

    \req{ProductVersion}\label{kat:ProductVersion} (Wersja Produktu)

    \textbf{Opis}: Wersja produktu. Podczas każdej edycji powstaje nowa wersja. Wersji nie można usunąć, można ją jedynie zastąpić nową wersją
    \par
    \textbf{Atrybuty}:
    \begin{itemize}[series=atr, wide, align=left, leftmargin=190pt]
        \atr{id}\label{kat:ProductVersion:id}- identyfikator
        \atr{createdDate}\label{kat:ProductVersion:createdDate}- czas utworzenia wersji
        \atr{description}\label{kat:ProductVersion:description}- krótki opis produktu w~języku produktu
    \end{itemize}

    \req{ProductBasicNutritionData}\label{kat:ProductBasicNutritionData} (Podstawowe Składniki Odżywcze Produktu)

    \textbf{Opis}: Podstawowe składniki odżywcze produktu
    \par
    \textbf{Atrybuty}:
    \begin{itemize}[series=atr, wide, align=left, leftmargin=190pt]
        \atr{id}\label{kat:ProductBasicNutritionData:id}- identyfikator
        \atr{energy}\label{kat:ProductBasicNutritionData:energy}- energia w~kilokaloriach (kcal) na 100 gramów produktu
        \atr{protein}\label{kat:ProductBasicNutritionData:protein}- białko w~gramach na 100 gramów produktu
        \atr{fat}\label{kat:ProductBasicNutritionData:fat}- tłuszcz w~gramach na 100 gramów produktu
        \atr{carbohydrates}\label{kat:ProductBasicNutritionData:carbohydrates}- węglowodany w~gramach na 100 gramów produktu
    \end{itemize}

    \req{NutritionData}\label{kat:NutritionData} (Wartość Odżywcza)

    \textbf{Opis}: Wartość wartości odżywczej dla konkretnego produktu
    \par
    \textbf{Atrybuty}:
    \begin{itemize}[series=atr, wide, align=left, leftmargin=190pt]
        \atr{id}\label{kat:NutritionData:id}- identyfikator
        \atr{nutritionValue}\label{kat:NutritionData:nutritionValue}- ilość składnika odżywczego w~jednostkach wyspecyfikowanych w~definicji wartości odżywczej
    \end{itemize}

    \req{NutritionDefinition}\label{kat:NutritionDefinition} (Definicja Wartości Odżywczej)

    \textbf{Opis}: Definicja wartości odżywczej
    \par
    \textbf{Atrybuty}:
    \begin{itemize}[series=atr, wide, align=left, leftmargin=190pt]
        \atr{id}\label{kat:NutritionDefinition:id}- identyfikator
        \atr{tag}\label{kat:NutritionDefinition:tag}- krótki znacznik reprezentujący wartość odżywczą
        \atr{description}\label{kat:NutritionDefinition:description}- krótki opis wartości odżywczej w~języku angielskim
        \atr{units}\label{kat:NutritionDefinition:units}- jednostki wykorzystywane do pomiaru wartości odżywczej, np. "g", "kcal", "ml"
        \atr{decimalPlaces}\label{kat:NutritionDefinition:decimalPlaces}- liczba miejsc dziesiętnych do których wartość składnika odżywczego powinna być zaokrąglana
    \end{itemize}

    \req{NutritionDefinitionTranslation}\label{kat:NutritionDefinitionTranslation} (Tłumaczenie Definicji Wartości Odżywczej)

    \textbf{Opis}: Tłumaczenie definicji wartości odżywczej
    \par
    \textbf{Atrybuty}:
    \begin{itemize}[series=atr, wide, align=left, leftmargin=190pt]
        \atr{id}\label{kat:NutritionDefinitionTranslation:id}- identyfikator
        \atr{translation}\label{kat:NutritionDefinitionTranslation:translation}- przetłumaczony opis definicji wartości odżywczej
        \atr{language}\label{kat:NutritionDefinitionTranslation:language}- język tłumaczenia w~postaci kodu ISO 639-1
    \end{itemize}

    \req{HouseholdMeasure}\label{kat:HouseholdMeasure} (Miara Domowa)

    \textbf{Opis}: Miara domowa produktu z~masą w~gramach
    \par
    \textbf{Atrybuty}:
    \begin{itemize}[series=atr, wide, align=left, leftmargin=190pt]
        \atr{id}\label{kat:HouseholdMeasure:id}- identyfikator
        \atr{description}\label{kat:HouseholdMeasure:description}- krótki opis miary domowej w~języku produktu, np. "szklanka" lub "łyżeczka"
        \atr{gramsWeight}\label{kat:HouseholdMeasure:gramsWeight}- masa w~gramach 1~jednostki miary domowej
        \atr{isVisible}\label{kat:HouseholdMeasure:isVisible}- flaga określająca czy miara jest widoczna podczas wyświetlania produktu
    \end{itemize}

    \req{ProductSubcategory}\label{kat:ProductSubcategory} (Podkategoria Produktu)

    \textbf{Opis}: Podkategoria produktu
    \par
    \textbf{Atrybuty}:
    \begin{itemize}[series=atr, wide, align=left, leftmargin=190pt]
        \atr{id}\label{kat:ProductSubcategory:id}- identyfikator
        \atr{description}\label{kat:ProductSubcategory:description}- krótki opis podkategorii w~języku produktu
    \end{itemize}

    \req{ProductCategory}\label{kat:ProductCategory} (Kategoria Produktu)

    \textbf{Opis}: Główna kategoria produktu
    \par
    \textbf{Atrybuty}:
    \begin{itemize}[series=atr, wide, align=left, leftmargin=190pt]
        \atr{id}\label{kat:ProductCategory:id}- identyfikator
        \atr{description}\label{kat:ProductCategory:description}- krótki opis kategorii produktu w~języku angielskim
    \end{itemize}

    \req{ProductCategoryTranslation}\label{kat:ProductCategoryTranslation} (Tłumaczenie Kategorii Produktu)

    \textbf{Opis}: Tłumaczenie kategorii produktu
    \par
    \textbf{Atrybuty}:
    \begin{itemize}[series=atr, wide, align=left, leftmargin=190pt]
        \atr{id}\label{kat:ProductCategoryTranslation:id}- identyfikator
        \atr{translation}\label{kat:ProductCategoryTranslation:translation}- przetłumaczona nazwa kategorii produktu
        \atr{language}\label{kat:ProductCategoryTranslation:language}- język tłumaczenia w~postaci kodu ISO 639-1
    \end{itemize}

    \req{DietType}\label{kat:DietType} (Typ Diety)

    \textbf{Opis}: Typ diety
    \par
    \textbf{Atrybuty}:
    \begin{itemize}[series=atr, wide, align=left, leftmargin=190pt]
        \atr{id}\label{kat:DietType:id}- identyfikator
        \atr{name}\label{kat:DietType:name}- krótki opis typu diety w~języku angielskim
    \end{itemize}

    \req{DietTypeTranslation}\label{kat:DietTypeTranslation} (Tłumaczenie Typu Diety)

    \textbf{Opis}: Tłumaczenie typu diety
    \par
    \textbf{Atrybuty}:
    \begin{itemize}[series=atr, wide, align=left, leftmargin=190pt]
        \atr{id}\label{kat:DietTypeTranslation:id}- identyfikator
        \atr{translation}\label{kat:DietTypeTranslation:translation}- tłumaczenie nazwy typu diety
        \atr{language}\label{kat:DietTypeTranslation:language}- język tłumaczenia w~postaci kodu ISO 639-1
    \end{itemize}

    \req{Recipe}\label{kat:Recipe} (Przepis)

    \textbf{Opis}: Przepis
    \par
    \textbf{Atrybuty}:
    \begin{itemize}[series=atr, wide, align=left, leftmargin=190pt]
        \atr{id}\label{kat:Recipe:id}- identyfikator
        \atr{isPublic}\label{kat:Recipe:isPublic}- flaga określająca czy przepis jest widoczny publicznie
        \atr{language}\label{kat:Recipe:language}- język tłumaczenia w~postaci kodu ISO 639-1
    \end{itemize}

    \req{RecipeVersion}\label{kat:RecipeVersion} (Wersja Przepisu)

    \textbf{Opis}: Wersja przepisu. Podczas każdej edycji powstaje nowa wersja. Wersji nie można usunąć, można ją jedynie zastąpić nową wersją
    \par
    \textbf{Atrybuty}:
    \begin{itemize}[series=atr, wide, align=left, leftmargin=190pt]
        \atr{id}\label{kat:RecipeVersion:id}- identyfikator
        \atr{editTimestamp}\label{kat:RecipeVersion:editTimestamp}- czas utworzenia wersji
        \atr{name}\label{kat:RecipeVersion:name}- nazwa przepisu w~języku przepisu
        \atr{preparationTimeMinutes}\label{kat:RecipeVersion:preparationTimeMinutes}- średni czas potrzebny na całkowite przygotowanie przepisu, zdefiniowany w~minutach
        \atr{numberOfPortions}\label{kat:RecipeVersion:numberOfPortions}- liczba porcji dla których przepis jest zdefiniowany
        \atr{image}\label{kat:RecipeVersion:image}- opcjonalne zdjęcie dania przygotowanego na podstawie przepisu
        \atr{totalGramsWeight}\label{kat:RecipeVersion:totalGramsWeight}- całkowita masa w~gramach dania przygotowanego z~przepisu
    \end{itemize}

    \req{RecipeBasicNutritionData}\label{kat:RecipeBasicNutritionData} (Podstawowe Wartości Odżywcze Przepisu)

    \textbf{Opis}: Podstawowe wartości odżywcze przepisu
    \par
    \textbf{Atrybuty}:
    \begin{itemize}[series=atr, wide, align=left, leftmargin=190pt]
        \atr{id}\label{kat:RecipeBasicNutritionData:id}- identyfikator
        \atr{energy}\label{kat:RecipeBasicNutritionData:energy}- energia w~kilokaloriach (kcal) na 100 gramów posiłku przygotowanego z~użyciem produktów w~przepisie
        \atr{protein}\label{kat:RecipeBasicNutritionData:protein}- białko w~gramach na 100 gramów posiłku przygotowanego z~użyciem produktów w~przepisie
        \atr{fat}\label{kat:RecipeBasicNutritionData:fat}- tłuszcz w~gramach na 100 gramów posiłku przygotowanego z~użyciem produktów w~przepisie
        \atr{carbohydrates}\label{kat:RecipeBasicNutritionData:carbohydrates}- węglowodany w~gramach na 100 gramów posiłku przygotowanego z~użyciem produktów w~przepisie
    \end{itemize}

    \req{RecipeSection}\label{kat:RecipeSection} (Sekcja Przepisu)

    \textbf{Opis}: Sekcja przepisu, np. sernik może mieć 3~sekcje odpowiednio dla ciasta, masy i~polewy
    \par
    \textbf{Atrybuty}:
    \begin{itemize}[series=atr, wide, align=left, leftmargin=190pt]
        \atr{id}\label{kat:RecipeSection:id}- identyfikator
        \atr{sectionName}\label{kat:RecipeSection:sectionName}- nazwa przepisu w~języku przepisu
    \end{itemize}

    \req{ProductPortion}\label{kat:ProductPortion} (Porcja Produktu)

    \textbf{Opis}: Porcja produktu wykorzystywana w~przepisie
    \par
    \textbf{Atrybuty}:
    \begin{itemize}[series=atr, wide, align=left, leftmargin=190pt]
        \atr{id}\label{kat:ProductPortion:id}- identyfikator
        \atr{amount}\label{kat:ProductPortion:amount}- ilość produktu w~jednostkach miary domowej lub w~gramach jeśli miara domowa nie jest zdefiniowana
    \end{itemize}

    \req{PreparationStep}\label{kat:PreparationStep} (Krok Przygotowania)

    \textbf{Opis}: Krok przygotowania w~przepisie
    \par
    \textbf{Atrybuty}:
    \begin{itemize}[series=atr, wide, align=left, leftmargin=190pt]
        \atr{id}\label{kat:PreparationStep:id}- identyfikator
        \atr{ordinalNumber}\label{kat:PreparationStep:ordinalNumber}- liczba porządkowa kroku przygotowania
        \atr{stepDescription}\label{kat:PreparationStep:stepDescription}- w~miarę możliwości krótki opis kroku przygotowania
    \end{itemize}

    \req{KitchenAppliance}\label{kat:KitchenAppliance} (Sprzęt Kuchenny)

    \textbf{Opis}: Definicja sprzętu kuchennego
    \par
    \textbf{Atrybuty}:
    \begin{itemize}[series=atr, wide, align=left, leftmargin=190pt]
        \atr{id}\label{kat:KitchenAppliance:id}- identyfikator
        \atr{name}\label{kat:KitchenAppliance:name}- nazwa sprzętu kuchennego w~języku angielskim
    \end{itemize}

    \req{KitchenApplianceTranslation}\label{kat:KitchenApplianceTranslation} (Tłumaczenie Sprzętu Kuchennego)

    \textbf{Opis}: Tłumaczenie sprzętu kuchennego
    \par
    \textbf{Atrybuty}:
    \begin{itemize}[series=atr, wide, align=left, leftmargin=190pt]
        \atr{id}\label{kat:KitchenApplianceTranslation:id}- identyfikator
        \atr{translation}\label{kat:KitchenApplianceTranslation:translation}- przetłumaczona nazwa sprzętu kuchennego
        \atr{language}\label{kat:KitchenApplianceTranslation:language}- język tłumaczenia w~postaci kodu ISO 639-1
    \end{itemize}

    \req{DishType}\label{kat:DishType} (Typ Dania)

    \textbf{Opis}: Typ dania, np. sałatka lub zupa
    \par
    \textbf{Atrybuty}:
    \begin{itemize}[series=atr, wide, align=left, leftmargin=190pt]
        \atr{id}\label{kat:DishType:id}- identyfikator
        \atr{description}\label{kat:DishType:description}- opis typu dania w~języku angielskim
    \end{itemize}

    \req{DishTypeTranslation}\label{kat:DishTypeTranslation} (Tłumaczenie Typu Dania)

    \textbf{Opis}: Tłumaczenie typu dania
    \par
    \textbf{Atrybuty}:
    \begin{itemize}[series=atr, wide, align=left, leftmargin=190pt]
        \atr{id}\label{kat:DishTypeTranslation:id}- identyfikator
        \atr{translation}\label{kat:DishTypeTranslation:translation}- przetłumaczona nazwa typu dania
        \atr{language}\label{kat:DishTypeTranslation:language}- język tłumaczenia w~postaci kodu ISO 639-1
    \end{itemize}

    \req{MealType}\label{kat:MealType} (Typ Posiłku)

    \textbf{Opis}: Typ posiłku, np. śniadanie lub obiad
    \par
    \textbf{Atrybuty}:
    \begin{itemize}[series=atr, wide, align=left, leftmargin=190pt]
        \atr{id}\label{kat:MealType:id}- identyfikator
        \atr{name}\label{kat:MealType:name}- nazwa typu posiłku w~języku angielskim
    \end{itemize}

    \req{MealTypeTranslation}\label{kat:MealTypeTranslation} (Tłumaczenie Typu Posiłku)

    \textbf{Opis}: Meal type translation
    \par
    \textbf{Atrybuty}:
    \begin{itemize}[series=atr, wide, align=left, leftmargin=190pt]
        \atr{id}\label{kat:MealTypeTranslation:id}- identyfikator
        \atr{translation}\label{kat:MealTypeTranslation:translation}- przetłumaczona nazwa typu posiłku
        \atr{language}\label{kat:MealTypeTranslation:language}- język tłumaczenia w~postaci kodu ISO 639-1
    \end{itemize}


    \req{MealPlan}\label{kat:MealPlan} (Jadłospis)

    \textbf{Opis}: Jadłospis; plan posiłków z~podziałem na dnie i~posiłki
    \par
    \textbf{Atrybuty}:
    \begin{itemize}[series=atr, wide, align=left, leftmargin=190pt]
        \atr{id}\label{kat:MealPlan:id}- identyfikator
        \atr{creationTimestamp}\label{kat:MealPlan:creationTimestamp}- czas utworzenia jadłospisu
        \atr{editTimestamp}\label{kat:MealPlan:editTimestamp}- czas ostatniej edycji jadłospisu
        \atr{name}\label{kat:MealPlan:name}- nazwa jadłospisu
        \atr{isVisible}\label{kat:MealPlan:isVisible}- flaga określająca czy jadłospis jest widoczny na liście jadłospisów autora
        \atr{language}\label{kat:MealPlan:language}- język tłumaczenia w~postaci kodu ISO 639-1
        \atr{numberOfDays}\label{kat:MealPlan:numberOfDays}- liczba dni planu
        \atr{numberOfMealsPerDay}\label{kat:MealPlan:numberOfMealsPerDay}- liczba posiłków w~ciągu dnia
        \atr{totalDailyEnergy}\label{kat:MealPlan:totalDailyEnergy}- całkowita oczekiwana energia w~ciągu dnia w~kilokaloriach
        \atr{percentOfProtein}\label{kat:MealPlan:percentOfProtein}- procent białka w~całkowitej dziennej energii
        \atr{percentOfFat}\label{kat:MealPlan:percentOfFat}- procent tłuszczu w~całkowitej dziennej energii
        \atr{percentOfCarbohydrates}\label{kat:MealPlan:percentOfCarbohydrates}- procent węglowodanów w~całkowitej dziennej energii
    \end{itemize}

    \req{MealPlanDay}\label{kat:MealPlanDay} (Dzień Jadłospisu)

    \textbf{Opis}: Dzień w~jadłospisie
    \par
    \textbf{Atrybuty}:
    \begin{itemize}[series=atr, wide, align=left, leftmargin=190pt]
        \atr{id}\label{kat:MealPlanDay:id}- identyfikator
        \atr{ordinalNumber}\label{kat:MealPlanDay:ordinalNumber}- numer porządkowy dnia
    \end{itemize}

    \req{Meal}\label{kat:Meal} (Posiłek)

    \textbf{Opis}: Posiłek
    \par
    \textbf{Atrybuty}:
    \begin{itemize}[series=atr, wide, align=left, leftmargin=190pt]
        \atr{id}\label{kat:Meal:id}- identyfikator
        \atr{ordinalNumber}\label{kat:Meal:ordinalNumber}- numer porządkowy posiłku
    \end{itemize}

    \req{MealRecipe}\label{kat:MealRecipe} (Przepis Posiłku)

    \textbf{Opis}: Przepis przypisany do posiłku
    \par
    \textbf{Atrybuty}:
    \begin{itemize}[series=atr, wide, align=left, leftmargin=190pt]
        \atr{id}\label{kat:MealRecipe:id}- identyfikator
        \atr{amount}\label{kat:MealRecipe:amount}- ilość przepisu w~gramach
    \end{itemize}

    \req{MealProduct}\label{kat:MealProduct} (Produkt Posiłku)

    \textbf{Opis}: Produkt przypisany do posiłku
    \par
    \textbf{Atrybuty}:
    \begin{itemize}[series=atr, wide, align=left, leftmargin=190pt]
        \atr{id}\label{kat:MealProduct:id}- identyfikator
        \atr{amount}\label{kat:MealProduct:amount}- ilość produktu w~jednostkach miary domowej lub w~gramach jeśli miara domowa nie jest zdefiniowana
    \end{itemize}

    \req{MealDefinition}\label{kat:MealDefinition} (Definicja Posiłku)

    \textbf{Opis}: Definicja posiłku wykorzystywana do określenia właściwości każdego posiłku w~ciągu dnia
    \par
    \textbf{Atrybuty}:
    \begin{itemize}[series=atr, wide, align=left, leftmargin=190pt]
        \atr{id}\label{kat:MealDefinition:id}- identyfikator
        \atr{ordinalNumber}\label{kat:MealDefinition:ordinalNumber}- dzienny numer porządkowy posiłku
        \atr{timeOfMeal}\label{kat:MealDefinition:timeOfMeal}- typowy czas posiłku w~formacie 24h w~postaci: HH:mm
        \atr{percentOfEnergy}\label{kat:MealDefinition:percentOfEnergy}- część dziennej całkowitej dziennej energii w~procentach
    \end{itemize}

    \req{Appointment}\label{kat:Appointment} (Wizyta)

    \textbf{Opis}: Wizyta dietetyczna
    \par
    \textbf{Atrybuty}:
    \begin{itemize}[series=atr, wide, align=left, leftmargin=190pt]
        \atr{id}\label{kat:Appointment:id}- identyfikator
        \atr{appointmentDate}\label{kat:Appointment:appointmentDate}- data i~godzina wizyty
        \atr{appointmentState}\label{kat:Appointment:appointmentState}- stan wizyty
        \atr{generalAdvice}\label{kat:Appointment:generalAdvice}- ogólna porada po wizycie
    \end{itemize}

    \req{PatientCard}\label{kat:PatientCard} (Karta Pacjenta)

    \textbf{Opis}: Karta pacjenta
    \par
    \textbf{Atrybuty}:
    \begin{itemize}[series=atr, wide, align=left, leftmargin=190pt]
        \atr{id}\label{kat:PatientCard:id}- identyfikator
        \atr{creationDate}\label{kat:PatientCard:creationDate}- data rejestracji pacjenta do dietetyka
    \end{itemize}

    \req{AppointmentEvaluation}\label{kat:AppointmentEvaluation} (Ewaluacja Wizyty)

    \textbf{Opis}: Ocena wizyty przez pacjenta
    \par
    \textbf{Atrybuty}:
    \begin{itemize}[series=atr, wide, align=left, leftmargin=190pt]
        \atr{id}\label{kat:AppointmentEvaluation:id}- identyfikator
        \atr{overallSatisfaction}\label{kat:AppointmentEvaluation:overallSatisfaction}- ogólne zadowolenie z~wizyty
        \atr{dietitianServiceSatisfaction}\label{kat:AppointmentEvaluation:dietitianServiceSatisfaction}- zadowolenie z~obsługi dietetyka
        \atr{mealPlanOverallSatisfaction}\label{kat:AppointmentEvaluation:mealPlanOverallSatisfaction}- ogólne zadowolenie z~jadłospisu
        \atr{mealCostSatisfaction}\label{kat:AppointmentEvaluation:mealCostSatisfaction}- zadowolenie z~kosztów posiłków
        \atr{mealPreparationTimeSatisfaction}\label{kat:AppointmentEvaluation:mealPreparationTimeSatisfaction}- zadowolenie z~czasu przygotowania posiłków
        \atr{mealComplexityLevelSatisfaction}\label{kat:AppointmentEvaluation:mealComplexityLevelSatisfaction}- zadowolenie z~poziomu skomplikowania posiłków
        \atr{mealTastefulnessSatisfaction}\label{kat:AppointmentEvaluation:mealTastefulnessSatisfaction}- zadowolenie ze smaku posiłków
        \atr{dietaryResultSatisfaction}\label{kat:AppointmentEvaluation:dietaryResultSatisfaction}- zadowolenie z~rezultatów dietetycznych
        \atr{comment}\label{kat:AppointmentEvaluation:comment}- opcjonalny komentarz do wizyty
    \end{itemize}

    \req{BodyMeasurement}\label{kat:BodyMeasurement} (Pomiar Ciała)

    \textbf{Opis}: Pomiar ciała
    \par
    \textbf{Atrybuty}:
    \begin{itemize}[series=atr, wide, align=left, leftmargin=190pt]
        \atr{id}\label{kat:BodyMeasurement:id}- identyfikator
        \atr{completionDate}\label{kat:BodyMeasurement:completionDate}- data przeprowadzenia pomiaru
        \atr{height}\label{kat:BodyMeasurement:height}- wzrost pacjenta; razem z~wagą wykorzystywany do obliczania współczynnika BMI
        \atr{weight}\label{kat:BodyMeasurement:weight}- waga pacjenta; razem z~wzrostem wykorzystywany do obliczania współczynnika BMI
        \atr{waist}\label{kat:BodyMeasurement:waist}- obwód pasa pacjenta
        \atr{percentOfFatTissue}\label{kat:BodyMeasurement:percentOfFatTissue}- procent tkanki tłuszczowej w~ciele pacjenta; norma dla kobiet: 16-20; norma dla mężczyzn: 15-18
        \atr{percentOfWater}\label{kat:BodyMeasurement:percentOfWater}- procent wody w~ciele pacjenta; norma dla kobiet: 45-60; norma dla mężczyzn: 50-65
        \atr{muscleMass}\label{kat:BodyMeasurement:muscleMass}- masa tkanki mięśniowej w~ciele pacjenta w~kilogramach
        \atr{physicalMark}\label{kat:BodyMeasurement:physicalMark}- ocena fizyczna; norma: 5
        \atr{calciumInBones}\label{kat:BodyMeasurement:calciumInBones}- poziom wapnia w~kościach pacjenta w~kilogramach; norma: ~2.4kg
        \atr{basicMetabolism}\label{kat:BodyMeasurement:basicMetabolism}- podstawowy metabolizm w~kilokaloriach
        \atr{metabolicAge}\label{kat:BodyMeasurement:metabolicAge}- wiek metaboliczny w~latach
        \atr{visceralFatLevel}\label{kat:BodyMeasurement:visceralFatLevel}- poziom tłuszczu trzewnego; norma: 1-12
    \end{itemize}

    \req{NutritionalInterview}\label{kat:NutritionalInterview} (Wywiad Żywieniowy)

    \textbf{Opis}: Wywiad żywieniowy
    \par
    \textbf{Atrybuty}:
    \begin{itemize}[series=atr, wide, align=left, leftmargin=190pt]
        \atr{id}\label{kat:NutritionalInterview:id}- identyfikator
        \atr{completionDate}\label{kat:NutritionalInterview:completionDate}- czas przeprowadzenia wywiadu
        \atr{targetWeight}\label{kat:NutritionalInterview:targetWeight}- docelowa waga pacjenta w~kilogramach
        \atr{advicePurpose}\label{kat:NutritionalInterview:advicePurpose}- cel wizyty podsumowujący co pacjent pragnie osiągnąć poprzez terapię dietetyczną
        \atr{physicalActivity}\label{kat:NutritionalInterview:physicalActivity}- poziom aktywności fizycznej pacjenta
        \atr{diseases}\label{kat:NutritionalInterview:diseases}- choroby pacjenta
        \atr{medicines}\label{kat:NutritionalInterview:medicines}- leki przyjmowane przez pacjenta
        \atr{jobType}\label{kat:NutritionalInterview:jobType}- typ pracy pacjenta
        \atr{likedProducts}\label{kat:NutritionalInterview:likedProducts}- produkty spożywcze, które pacjent lubi
        \atr{dislikedProducts}\label{kat:NutritionalInterview:dislikedProducts}- produkty spożywcze, których pacjent nie lubi
        \atr{foodAllergies}\label{kat:NutritionalInterview:foodAllergies}- produkty spożywcze na które pacjent jest uczulony
        \atr{foodIntolerances}\label{kat:NutritionalInterview:foodIntolerances}- nietolerancje pokarmowe pacjenta
    \end{itemize}

    \req{CustomNutritionalInterviewQuestion}\label{kat:CustomNutritionalInterviewQuestion} (Niestandardowe Pytanie Wywiadu Żywieniowego)

    \textbf{Opis}: Niestandardowe pytanie wywiadu żywieniowego
    \par
    \textbf{Atrybuty}:
    \begin{itemize}[series=atr, wide, align=left, leftmargin=190pt]
        \atr{id}\label{kat:CustomNutritionalInterviewQuestion:id}- identyfikator
        \atr{ordinalNumber}\label{kat:CustomNutritionalInterviewQuestion:ordinalNumber}- numer porządkowy pytania
        \atr{question}\label{kat:CustomNutritionalInterviewQuestion:question}- pytanie
        \atr{answer}\label{kat:CustomNutritionalInterviewQuestion:answer}- odpowiedź na pytanie
    \end{itemize}

    \req{CustomNutritionalInterviewQuestionTemplate}\label{kat:CustomNutritionalInterviewQuestionTemplate} (Szablon Niestandardowego Pytania Wywiadu Żywieniowego)

    \textbf{Opis}: Szablon niestandardowego pytania do wywiadu żywieniowego
    \par
    \textbf{Atrybuty}:
    \begin{itemize}[series=atr, wide, align=left, leftmargin=190pt]
        \atr{id}\label{kat:CustomNutritionalInterviewQuestionTemplate:id}- identyfikator
        \atr{question}\label{kat:CustomNutritionalInterviewQuestionTemplate:question}- pytanie
        \atr{language}\label{kat:CustomNutritionalInterviewQuestionTemplate:language}- język tłumaczenia w~postaci kodu ISO 639-1
    \end{itemize}

    \req{AssignedMealPlan}\label{kat:AssignedMealPlan} (Przypisany Jadłospis)

    \textbf{Opis}: Przypisany jadłospis
    \par
    \textbf{Atrybuty}:
    \begin{itemize}[series=atr, wide, align=left, leftmargin=190pt]
        \atr{id}\label{kat:AssignedMealPlan:id}- identyfikator
        \atr{assigmentTime}\label{kat:AssignedMealPlan:assigmentTime}- czas przypisania jadłospisu
    \end{itemize}

\end{enumerate}

\section {Reguły funkcjonowania}\label{sec:functionalRules}

\begin{itemize}[label={\textbf{Reguły dla}}, wide, labelwidth=!, labelindent=0pt]
    \setlength\itemsep{1em}
    \item[\textbf{Reguły}] \textbf{ogólne}
    \begin{enumerate}[label={\textbf{REG/\protect\threedigits{\arabic{enumi}}}}, wide, labelwidth=!, align=left, leftmargin=3cm]
        \item Przedmiot kompozycji podlega takim samym zasadom dostępu co właściciel kompozycji pod warunkiem, że przedmiot kompozycji nie definiuje własnych reguł dostępu
    \end{enumerate}
    \item\ref{kat:User}
    \begin{enumerate}[label={\textbf{REG/\protect\threedigits{\arabic{enumi}}}}, wide, labelwidth=!, align=left, leftmargin=3cm, resume]
        %Relacje
        \item Użytkownik (\ref{kat:User}) nie musi mieć musi mieć żadnych dodatkowych informacji (\ref{kat:UserExtraInfo})
        \item Użytkownik (\ref{kat:User}) może mieć maksymalnie jedne dodatkowe informacje (\ref{kat:UserExtraInfo})
        \item Użytkownik (\ref{kat:User}) musi mieć musi mieć przynajmniej jedną rolę (\ref{kat:Authority})
        \item Użytkownik (\ref{kat:User}) może mieć wiele ról (\ref{kat:Authority})
        \item Użytkownik (\ref{kat:User}) nie musi mieć autora (\ref{kat:User})
        \item Użytkownik (\ref{kat:User}) może mieć maksymalnie jednego autora (\ref{kat:User})
        \item Użytkownik (\ref{kat:User}) nie musi mieć ostatniego edytora (\ref{kat:User})
        \item Użytkownik (\ref{kat:User}) może mieć maksymalnie jednego ostatniego edytora (\ref{kat:User})
        %CRUD
        \item \role{Gość} może dodawać nowego użytkownika (\ref{kat:User})
        \item \role{Użytkownik} może wyświetlać, edytować i~usuwać swoje dane użytkownika (\ref{kat:User})
        \item \role{Dietetyk} może wyświetlać dane \role{Pacjenta}, którego kartotekę prowadzi
        \item \role{Administrator} może wyświetlać i~usuwać dane użytkownika (\ref{kat:User})
    \end{enumerate}
    \item\ref{kat:Authority}
    \begin{enumerate}[label={\textbf{REG/\protect\threedigits{\arabic{enumi}}}}, wide, labelwidth=!, align=left, leftmargin=3cm, resume]
        %Relacje
        %CRUD
        \item \role{Administrator} może dodawać, wyświetlać, edytować i~usuwać dane roli (\ref{kat:Authority})
    \end{enumerate}
    \item\ref{kat:UserExtraInfo}
    \begin{enumerate}[label={\textbf{REG/\protect\threedigits{\arabic{enumi}}}}, wide, labelwidth=!, align=left, leftmargin=3cm, resume]
        %Relacje
        \item Dodatkowe informacje (\ref{kat:UserExtraInfo}) muszą być przypisane do dokładnie jednego użytkownika (\ref{kat:User})
        %CRUD
        \item Dodatkowe informacje (\ref{kat:UserExtraInfo}) są przedmiotem kompozycji ze strony użytkownika (\ref{kat:User})
        \item \role{Administrator} nie może wyświetlać i~usuwać dodatkowych informacji (\ref{kat:UserExtraInfo}) innych użytkowników
    \end{enumerate}
    \item\ref{kat:SiteContent}
    \begin{enumerate}[label={\textbf{REG/\protect\threedigits{\arabic{enumi}}}}, wide, labelwidth=!, align=left, leftmargin=3cm, resume]
        %Relacje
        \item Treść strony (\ref{kat:SiteContent}) nie musi mieć żadnego tłumaczenia (\ref{kat:SiteContentTranslation})
        \item Treść strony (\ref{kat:SiteContent}) może mieć wiele tłumaczeń (\ref{kat:SiteContentTranslation})
        %CRUD
        \item \role{Gość} może wyświetlać dane treści strony (\ref{kat:SiteContent})
        \item \role{Użytkownik} może wyświetlać dane treści strony (\ref{kat:SiteContent})
        \item \role{Administrator} może dodawać, wyświetlać, edytować i~usuwać dane treści strony (\ref{kat:SiteContent})
    \end{enumerate}
    \item\ref{kat:SiteContentTranslation}
    \begin{enumerate}[label={\textbf{REG/\protect\threedigits{\arabic{enumi}}}}, wide, labelwidth=!, align=left, leftmargin=3cm, resume]
        %Relacje
        \item Tłumaczenie treści strony (\ref{kat:SiteContentTranslation}) musi być przypisane do dokładnie jednej treści strony (\ref{kat:SiteContent})
        %CRUD
        \item Tłumaczenie treści strony (\ref{kat:SiteContentTranslation}) jest przedmiotem kompozycji ze strony treści strony (\ref{kat:SiteContent})
    \end{enumerate}
    \item\ref{kat:ContactInfo}
    \begin{enumerate}[label={\textbf{REG/\protect\threedigits{\arabic{enumi}}}}, wide, labelwidth=!, align=left, leftmargin=3cm, resume]
        %CRUD
        \item \role{Gość} może wyświetlać dane informacji kontaktowych (\ref{kat:ContactInfo})
        \item \role{Użytkownik} może wyświetlać dane informacji kontaktowych (\ref{kat:ContactInfo})
        \item \role{Administrator} może dodawać, wyświetlać, edytować i~usuwać dane informacji kontaktowych (\ref{kat:ContactInfo})
    \end{enumerate}
    \item\ref{kat:Pricing}
    \begin{enumerate}[label={\textbf{REG/\protect\threedigits{\arabic{enumi}}}}, wide, labelwidth=!, align=left, leftmargin=3cm, resume]
        %Relacje
        \item Cennik (\ref{kat:Pricing}) nie musi mieć przypisanych żadnych tłumaczeń (\ref{kat:PricingTranslation})
        \item Cennik (\ref{kat:Pricing}) może mieć przypisane wiele tłumaczeń (\ref{kat:PricingTranslation})
        %CRUD
        \item \role{Gość} może wyświetlać dane cennika (\ref{kat:Pricing})
        \item \role{Użytkownik} może wyświetlać dane cennika (\ref{kat:Pricing})
        \item \role{Administrator} może dodawać, wyświetlać, edytować i~usuwać dane cennika (\ref{kat:Pricing})
    \end{enumerate}
    \item\ref{kat:PricingTranslation}
    \begin{enumerate}[label={\textbf{REG/\protect\threedigits{\arabic{enumi}}}}, wide, labelwidth=!, align=left, leftmargin=3cm, resume]
        %Relacje
        \item Tłumaczenie cennika (\ref{kat:PricingTranslation}) musi być przypisane do dokładnie jednego cennika (\ref{kat:Pricing})
        %CRUD
        \item Tłumaczenie cennika (\ref{kat:PricingTranslation}) jest przedmiotem kompozycji ze strony cennika (\ref{kat:Pricing})
    \end{enumerate}
    \item\ref{kat:Product}
    \begin{enumerate}[label={\textbf{REG/\protect\threedigits{\arabic{enumi}}}}, wide, labelwidth=!, align=left, leftmargin=3cm, resume]
        %Relacje
        \item Produkt (\ref{kat:Product}) musi mieć przynajmniej jedną wersję (\ref{kat:ProductVersion})
        \item Produkt (\ref{kat:Product}) może mieć wiele wersji (\ref{kat:ProductVersion})
        \item Produkt (\ref{kat:Product}) nie musi mieć zdefiniowanego autora (\ref{kat:User})
        \item Produkt (\ref{kat:Product}) może mieć maksymalnie jednego autora (\ref{kat:User})
        %CRUD
        \item \role{Dietetyk} może wyświetlać publiczne produkty (\ref{kat:Product})
        \item \role{Dietetyk} może dodawać, wyświetlać, edytować i~usuwać własne produkty (\ref{kat:Product})
        \item \role{Administrator} może wyświetlać i~usuwać produkty (\ref{kat:Product})
    \end{enumerate}
    \item\ref{kat:ProductVersion}
    \begin{enumerate}[label={\textbf{REG/\protect\threedigits{\arabic{enumi}}}}, wide, labelwidth=!, align=left, leftmargin=3cm, resume]
        %Relacje
        \item Wersja produktu (\ref{kat:ProductVersion}) musi być przypisana do dokładnie jednego produktu  (\ref{kat:Product})
        \item Wersja produktu (\ref{kat:ProductVersion}) musi być przypisana do dokładnie jednych podstawowych wartości odżywczych (\ref{kat:ProductBasicNutritionData})
        \item Wersja produktu (\ref{kat:ProductVersion}) nie musi mieć zdefiniowanych żadnych wartości odżywczych (\ref{kat:NutritionData})
        \item Wersja produktu (\ref{kat:ProductVersion}) może mieć zdefiniowane wiele wartości odżywczych (\ref{kat:NutritionData})
        \item Wersja produktu (\ref{kat:ProductVersion}) nie musi mieć zdefiniowanych żadnych miar domowych (\ref{kat:HouseholdMeasure})
        \item Wersja produktu (\ref{kat:ProductVersion}) może mieć zdefiniowane wiele miar domowych (\ref{kat:HouseholdMeasure})
        \item Wersja produktu (\ref{kat:ProductVersion}) musi należeć do dokładnie jednej podkategorii (\ref{kat:ProductSubcategory})
        \item Wersja produktu (\ref{kat:ProductVersion}) nie musi mieć przypisanego żadnego odpowiedniego typu diety (\ref{kat:DietType})
        \item Wersja produktu (\ref{kat:ProductVersion}) może mieć przypisanych wiele odpowiednich typów diety (\ref{kat:DietType})
        \item Wersja produktu (\ref{kat:ProductVersion}) nie musi mieć przypisanego żadnego nieodpowiedniego typu diety (\ref{kat:DietType})
        \item Wersja produktu (\ref{kat:ProductVersion}) może mieć przypisanych wiele nieodpowiednich typów diety (\ref{kat:DietType})
        %CRUD
        \item Wersja produktu (\ref{kat:ProductVersion}) jest przedmiotem kompozycji ze strony produktu (\ref{kat:Product})
        \item Wersja produktu (\ref{kat:ProductVersion}) po utworzeniu nie może być edytowana ani usuwana
    \end{enumerate}
    \item\ref{kat:ProductBasicNutritionData}
    \begin{enumerate}[label={\textbf{REG/\protect\threedigits{\arabic{enumi}}}}, wide, labelwidth=!, align=left, leftmargin=3cm, resume]
        %Relacje
        \item Podstawowe wartości odżywcze produktu(\ref{kat:ProductBasicNutritionData}) muszą być przypisane do dokładnie jednej wersji produktu (\ref{kat:ProductVersion})
        %CRUD
        \item Podstawowe wartości odżywcze produktu (\ref{kat:ProductBasicNutritionData}) są przedmiotem kompozycji ze strony wersji produktu (\ref{kat:ProductVersion})
    \end{enumerate}
    \item\ref{kat:NutritionData}
    \begin{enumerate}[label={\textbf{REG/\protect\threedigits{\arabic{enumi}}}}, wide, labelwidth=!, align=left, leftmargin=3cm, resume]
        %Relacje
        \item Wartość odżywcza (\ref{kat:NutritionData}) musi być przypisana do dokładnie jednej wersji produktu (\ref{kat:ProductVersion})
        \item Wartość odżywcza (\ref{kat:NutritionData}) musi być przypisana do dokładnie jednej definicji wartości odżywczej (\ref{kat:NutritionDefinition})
        %CRUD
        \item Wartość odżywcza (\ref{kat:NutritionData}) jest przedmiotem kompozycji ze strony wersji produktu (\ref{kat:ProductVersion})
    \end{enumerate}
    \item\ref{kat:NutritionDefinition}
    \begin{enumerate}[label={\textbf{REG/\protect\threedigits{\arabic{enumi}}}}, wide, labelwidth=!, align=left, leftmargin=3cm, resume]
        %Relacje
        \item Definicja wartości odżywczej (\ref{kat:NutritionDefinition}) nie musi mieć zdefiniowanego żadnego tłumaczenia (\ref{kat:NutritionDefinitionTranslation})
        \item Definicja wartości odżywczej (\ref{kat:NutritionDefinition}) może mieć zdefiniowanych wiele tłumaczeń (\ref{kat:NutritionDefinitionTranslation})
        %CRUD
        \item \role{Dietetyk} może wyświetlać definicję wartości odżywczej (\ref{kat:NutritionDefinition})
        \item \role{Administrator} może dodawać, wyświetlać, edytować i~usuwać definicję wartości odżywczej (\ref{kat:NutritionDefinition})
    \end{enumerate}
    \item\ref{kat:NutritionDefinitionTranslation}
    \begin{enumerate}[label={\textbf{REG/\protect\threedigits{\arabic{enumi}}}}, wide, labelwidth=!, align=left, leftmargin=3cm, resume]
        %Relacje
        \item Tłumaczenie definicji wartości odżywczej (\ref{kat:NutritionDefinitionTranslation}) musi być przypisane do dokładnie jednej definicji wartości odżywczej  (\ref{kat:NutritionDefinition})
        %CRUD
        \item Tłumaczenie definicji wartości odżywczej (\ref{kat:NutritionDefinitionTranslation}) jest przedmiotem kompozycji ze strony definicji wartości odżywczej (\ref{kat:NutritionDefinition})
    \end{enumerate}
    \item\ref{kat:HouseholdMeasure}
    \begin{enumerate}[label={\textbf{REG/\protect\threedigits{\arabic{enumi}}}}, wide, labelwidth=!, align=left, leftmargin=3cm, resume]
        %Relacje
        \item Miara domowa (\ref{kat:HouseholdMeasure}) musi być przypisana do dokładnie jednej wersji produktu (\ref{kat:ProductVersion})
        %CRUD
        \item Miara domowa (\ref{kat:HouseholdMeasure}) jest przedmiotem kompozycji ze strony wersji produktu (\ref{kat:ProductVersion})
    \end{enumerate}
    \item\ref{kat:ProductSubcategory}
    \begin{enumerate}[label={\textbf{REG/\protect\threedigits{\arabic{enumi}}}}, wide, labelwidth=!, align=left, leftmargin=3cm, resume]
        %Relacje
        \item Podkategoria produktu (\ref{kat:ProductSubcategory}) musi być przypisana do conajmniej jednej wersji produktu (\ref{kat:ProductVersion})
        \item Podkategoria produktu (\ref{kat:ProductSubcategory}) może być przypisana do wielu wersji produktu (\ref{kat:ProductVersion})
        \item Podktagoria produktu (\ref{kat:ProductSubcategory}) musi być przypisana do dokładnie jednej kategorii (\ref{kat:ProductCategory})
        %CRUD
        \item \role{Dietetyk} może dodawać i~wyświetlać podkategorię produktu (\ref{kat:ProductSubcategory})
        \item \role{Administrator} może wyświetlać podkategorię produktu (\ref{kat:ProductSubcategory})
    \end{enumerate}
    \item\ref{kat:ProductCategory}
    \begin{enumerate}[label={\textbf{REG/\protect\threedigits{\arabic{enumi}}}}, wide, labelwidth=!, align=left, leftmargin=3cm, resume]
        %Relacje
        \item Kategoria produktu (\ref{kat:ProductCategory}) nie musi mieć przypisanego żadnego tłumaczenia (\ref{kat:ProductCategoryTranslation})
        \item Kategoria produktu (\ref{kat:ProductCategory}) może mieć przypisanych wiele tłumaczeń (\ref{kat:ProductCategoryTranslation})
        %CRUD
        \item \role{Dietetyk} może wyświetlać kategorię produktu (\ref{kat:ProductCategory})
        \item \role{Administrator} może dodawać, wyświetlać, edytować i~usuwać kategorię produktu (\ref{kat:ProductCategory})
    \end{enumerate}
    \item\ref{kat:ProductCategoryTranslation}
    \begin{enumerate}[label={\textbf{REG/\protect\threedigits{\arabic{enumi}}}}, wide, labelwidth=!, align=left, leftmargin=3cm, resume]
        %Relacje
        \item Tłumaczenie kategorii produktu (\ref{kat:ProductCategoryTranslation}) musi być przypisane do dokładnie jednej kategorii (\ref{kat:ProductCategory})
        %CRUD
        \item Tłumaczenie kategorii produktu (\ref{kat:ProductCategoryTranslation}) jest przedmiotem kompozycji ze strony kategorii (\ref{kat:ProductCategory})
    \end{enumerate}
    \item\ref{kat:DietType}
    \begin{enumerate}[label={\textbf{REG/\protect\threedigits{\arabic{enumi}}}}, wide, labelwidth=!, align=left, leftmargin=3cm, resume]
        %Relacje
        \item Typ diety (\ref{kat:DietType}) nie musi mieć zdefiniowanego żadnego tłumaczenia (\ref{kat:DietTypeTranslation})
        \item Typ diety (\ref{kat:DietType}) może mieć zdefiniowanych wiele tłumaczeń (\ref{kat:DietTypeTranslation})
        %CRUD
        \item \role{Dietetyk} może wyświetlać typ diety (\ref{kat:DietType})
        \item \role{Administrator} może dodawać, wyświetlać, edytować i~usuwać typ diety (\ref{kat:DietType})
    \end{enumerate}
    \item\ref{kat:DietTypeTranslation}
    \begin{enumerate}[label={\textbf{REG/\protect\threedigits{\arabic{enumi}}}}, wide, labelwidth=!, align=left, leftmargin=3cm, resume]
        %Relacje
        \item Tłumaczenie typu diety (\ref{kat:DietTypeTranslation}) musi być przypisane do dokładnie jednego typu diety (\ref{kat:DietType})
        %CRUD
        \item Tłumaczenie typu diety (\ref{kat:DietTypeTranslation}) jest przedmiotem kompozycji ze strony typu diety (\ref{kat:DietType})
    \end{enumerate}
    \item\ref{kat:Recipe}
    \begin{enumerate}[label={\textbf{REG/\protect\threedigits{\arabic{enumi}}}}, wide, labelwidth=!, align=left, leftmargin=3cm, resume]
        %Relacje
        \item Przepis (\ref{kat:Recipe}) nie musi mieć zdefiniowanego żadnego przepisu źródłowego (\ref{kat:Recipe})
        \item Przepis (\ref{kat:Recipe}) może mieć zdefiniowany maksymalnie jeden przepis źródłowy (\ref{kat:Recipe})
        \item Przepis (\ref{kat:Recipe}) musi mieć przynajmniej jedną wersję (\ref{kat:RecipeVersion})
        \item Przepis (\ref{kat:Recipe}) może mieć wiele wersji (\ref{kat:RecipeVersion})
        \item Przepis (\ref{kat:Recipe}) nie musi mieć zdefiniowanego autora (\ref{kat:User})
        \item Przepis (\ref{kat:Recipe}) może mieć maksymalnie jednego autora (\ref{kat:User})
        %CRUD
        \item \role{Dietetyk} może wyświetlać publiczne przepisy (\ref{kat:Recipe})
        \item \role{Dietetyk} może dodawać, wyświetlać, edytować i~usuwać własne przepisy (\ref{kat:Recipe})
        \item \role{Administrator} może wyświetlać i~usuwać przepisy (\ref{kat:Recipe})
    \end{enumerate}
    \item\ref{kat:RecipeVersion}
    \begin{enumerate}[label={\textbf{REG/\protect\threedigits{\arabic{enumi}}}}, wide, labelwidth=!, align=left, leftmargin=3cm, resume]
        %Relacje
        \item Wersja przepisu (\ref{kat:RecipeVersion}) musi mieć dokładnie jedne podstawowe wartości odżywcze przepisu (\ref{kat:RecipeBasicNutritionData})
        \item Wersja przepisu (\ref{kat:RecipeVersion}) musi mieć przynajmniej jedną sekcję (\ref{kat:RecipeSection})
        \item Wersja przepisu (\ref{kat:RecipeVersion}) może mieć wiele sekcji (\ref{kat:RecipeSection})
        \item Wersja przepisu (\ref{kat:RecipeVersion}) nie musi mieć przypisanego żadnego sprzętu kuchennego (\ref{kat:KitchenAppliance})
        \item Wersja przepisu (\ref{kat:RecipeVersion}) może mieć przypisanych wiele sprzętów kuchennych (\ref{kat:KitchenAppliance})
        \item Wersja przepisu (\ref{kat:RecipeVersion}) nie musi mieć przypisanego żadnego typu dania (\ref{kat:DishType})
        \item Wersja przepisu (\ref{kat:RecipeVersion}) może mieć przypisanych wiele typów dań (\ref{kat:DishType})
        \item Wersja przepisu (\ref{kat:RecipeVersion}) nie musi mieć przypisanego żadnego typu posiłku (\ref{kat:MealType})
        \item Wersja przepisu (\ref{kat:RecipeVersion}) może mieć przypisanych wiele typów posiłków (\ref{kat:MealType})
        \item Wersja przepisu (\ref{kat:RecipeVersion}) nie musi mieć przypisanego żadnego odpowiedniego typu diety (\ref{kat:DietType})
        \item Wersja przepisu (\ref{kat:RecipeVersion}) może mieć przypisanych wiele odpowiednich typów diety  (\ref{kat:DietType})
        \item Wersja przepisu (\ref{kat:RecipeVersion}) nie musi mieć przypisanego żadnego nieodpowiedniego typu diety (\ref{kat:DietType})
        \item Wersja przepisu (\ref{kat:RecipeVersion}) może mieć przypisanych wiele nieodpowiednich typów diety (\ref{kat:DietType})
        %CRUD
        \item Wersja przepisu (\ref{kat:RecipeVersion}) jest przedmiotem kompozycji ze strony przepisu (\ref{kat:Recipe})
        \item Wersja przepisu (\ref{kat:RecipeVersion}) po utworzeniu nie może być edytowana ani usuwana
    \end{enumerate}
    \item\ref{kat:RecipeBasicNutritionData}
    \begin{enumerate}[label={\textbf{REG/\protect\threedigits{\arabic{enumi}}}}, wide, labelwidth=!, align=left, leftmargin=3cm, resume]
        %Relacje
        \item Podstawowe wartości odżywcze przepisu (\ref{kat:RecipeBasicNutritionData}) muszą być przypisane do dokładnie jednej wersji przepisu (\ref{kat:RecipeVersion})
        %CRUD
        \item Podstawowe wartości odżywcze przepisu (\ref{kat:RecipeBasicNutritionData}) są przedmiotem kompozycji ze strony wersji przepisu (\ref{kat:RecipeVersion})
    \end{enumerate}
    \item\ref{kat:RecipeSection}
    \begin{enumerate}[label={\textbf{REG/\protect\threedigits{\arabic{enumi}}}}, wide, labelwidth=!, align=left, leftmargin=3cm, resume]
        %Relacje
        \item Sekcja przepisu (\ref{kat:RecipeSection}) musi być przypisana do dokładniej jednej wersji przepisu (\ref{kat:RecipeVersion})
        \item Sekcja przepisu (\ref{kat:RecipeSection}) musi mieć przypisaną przynajmniej jedną porcję produktu (\ref{kat:ProductPortion})
        \item Sekcja przepisu (\ref{kat:RecipeSection}) może mieć przypisanych wiele porcji produktu (\ref{kat:ProductPortion})
        \item Sekcja przepisu (\ref{kat:RecipeSection}) musi mieć przypisany przynajmniej jeden krok przygotowania (\ref{kat:PreparationStep})
        \item Sekcja przepisu (\ref{kat:RecipeSection}) może mieć zdefiniowanych wiele kroków przygotowania (\ref{kat:PreparationStep})
        %CRUD
        \item Sekcja przepisu (\ref{kat:RecipeSection}) jest przedmiotem kompozycji ze strony wersji przepisu (\ref{kat:RecipeVersion})
    \end{enumerate}
    \item\ref{kat:ProductPortion}
    \begin{enumerate}[label={\textbf{REG/\protect\threedigits{\arabic{enumi}}}}, wide, labelwidth=!, align=left, leftmargin=3cm, resume]
        %Relacje
        \item Porcja produktu (\ref{kat:ProductPortion}) musi być przypisana do dokładnie jednej sekcji przepisu (\ref{kat:RecipeSection})
        \item Porcja produktu (\ref{kat:ProductPortion}) musi mieć przypisany dokładnie jeden produkt (\ref{kat:Product})
        \item Porcja produktu (\ref{kat:ProductPortion}) nie musi mieć przypisanej miary domowej (\ref{kat:HouseholdMeasure})
        \item Porcja produktu (\ref{kat:ProductPortion}) może mieć przypisaną maksymalnie jedną miarę domową (\ref{kat:HouseholdMeasure})
        %CRUD
        \item Porcja produktu (\ref{kat:ProductPortion}) jest przedmiotem kompozycji ze strony sekcji przepisu (\ref{kat:RecipeSection})
    \end{enumerate}
    \item\ref{kat:PreparationStep}
    \begin{enumerate}[label={\textbf{REG/\protect\threedigits{\arabic{enumi}}}}, wide, labelwidth=!, align=left, leftmargin=3cm, resume]
        %Relacje
        \item Krok przygotowania (\ref{kat:PreparationStep}) musi być przypisany do dokładnie jednej sekcji przepisu (\ref{kat:RecipeSection})
        %CRUD
        \item Krok przygotowania (\ref{kat:PreparationStep}) jest przedmiotem kompozycji ze strony sekcji przepisu (\ref{kat:RecipeSection})
    \end{enumerate}
    \item\ref{kat:KitchenAppliance}
    \begin{enumerate}[label={\textbf{REG/\protect\threedigits{\arabic{enumi}}}}, wide, labelwidth=!, align=left, leftmargin=3cm, resume]
        %Relacje
        \item Sprzęt kuchenny (\ref{kat:KitchenAppliance}) nie musi mieć zdefiniowanego żadnego tłumaczenia (\ref{kat:KitchenApplianceTranslation})
        \item Sprzęt kuchenny (\ref{kat:KitchenAppliance}) może mieć zdefiniowanych wiele tłumaczeń (\ref{kat:KitchenApplianceTranslation})
        %CRUD
        \item \role{Pacjent} może wyświetlać sprzęt kuchenny (\ref{kat:KitchenAppliance})
        \item \role{Dietetyk} może wyświetlać sprzęt kuchenny (\ref{kat:KitchenAppliance})
        \item \role{Administrator} może dodawać, wyświetlać, edytować i~usuwać sprzęt kuchenny (\ref{kat:KitchenAppliance})
    \end{enumerate}
    \item\ref{kat:KitchenApplianceTranslation}
    \begin{enumerate}[label={\textbf{REG/\protect\threedigits{\arabic{enumi}}}}, wide, labelwidth=!, align=left, leftmargin=3cm, resume]
        %Relacje
        \item Tłumaczenie sprzętu kuchennego (\ref{kat:KitchenApplianceTranslation}) musi być przypisane do dokładnie jednego sprzętu kuchennego (\ref{kat:KitchenAppliance})
        %CRUD
        \item Tłumaczenie sprzętu kuchennego (\ref{kat:KitchenApplianceTranslation}) jest przedmiotem kompozycji ze strony sprzętu kuchennego (\ref{kat:KitchenAppliance})
    \end{enumerate}
    \item\ref{kat:DishType}
    \begin{enumerate}[label={\textbf{REG/\protect\threedigits{\arabic{enumi}}}}, wide, labelwidth=!, align=left, leftmargin=3cm, resume]
        %Relacje
        \item Typ dania (\ref{kat:DishType}) nie musi mieć zdefiniowanego żadnego tłumaczenia (\ref{kat:DishTypeTranslation})
        \item Typ dania (\ref{kat:DishType}) może mieć zdefiniowanych wiele tłumaczeń (\ref{kat:DishTypeTranslation})
        %CRUD
        \item \role{Dietetyk} może wyświetlać typ dania (\ref{kat:DishType})
        \item \role{Administrator} może dodawać, wyświetlać, edytować i~usuwać typ dania (\ref{kat:DishType})
    \end{enumerate}
    \item\ref{kat:DishTypeTranslation}
    \begin{enumerate}[label={\textbf{REG/\protect\threedigits{\arabic{enumi}}}}, wide, labelwidth=!, align=left, leftmargin=3cm, resume]
        %Relacje
        \item Tłumaczenie typu dania (\ref{kat:DishTypeTranslation}) musi być przypisane do dokładnie jednego typu dania (\ref{kat:DishType})
        %CRUD
        \item Tłumaczenie typu dania (\ref{kat:DishTypeTranslation}) jest przedmiotem kompozycji ze strony typu dania (\ref{kat:DishType})
    \end{enumerate}
    \item\ref{kat:MealType}
    \begin{enumerate}[label={\textbf{REG/\protect\threedigits{\arabic{enumi}}}}, wide, labelwidth=!, align=left, leftmargin=3cm, resume]
        %Relacje
        \item Typ posiłku (\ref{kat:MealType}) nie musi mieć zdefiniowanego żadnego tłumaczenia (\ref{kat:MealTypeTranslation})
        \item Typ posiłku (\ref{kat:MealType}) może mieć zdefiniowanych wiele tłumaczeń (\ref{kat:MealTypeTranslation})
        %CRUD
        \item \role{Pacjent} może wyświetlać typ posiłku (\ref{kat:MealType})
        \item \role{Dietetyk} może wyświetlać typ posiłku (\ref{kat:MealType})
        \item \role{Administrator} może dodawać, wyświetlać, edytować i~usuwać typ posiłku (\ref{kat:MealType})
    \end{enumerate}
    \item\ref{kat:MealTypeTranslation}
    \begin{enumerate}[label={\textbf{REG/\protect\threedigits{\arabic{enumi}}}}, wide, labelwidth=!, align=left, leftmargin=3cm, resume]
        %Relacje
        \item Tłumaczenie typu posiłku (\ref{kat:MealTypeTranslation}) musi być przypisane do dokładnie jednego typu posiłku (\ref{kat:MealType})
        %CRUD
        \item Tłumaczenie typu posiłku (\ref{kat:MealTypeTranslation}) jest przedmiotem kompozycji ze strony typu posiłku (\ref{kat:MealType})
    \end{enumerate}
    \item\ref{kat:MealPlan}
    \begin{enumerate}[label={\textbf{REG/\protect\threedigits{\arabic{enumi}}}}, wide, labelwidth=!, align=left, leftmargin=3cm, resume]
        %Relacje
        \item Jadłospis (\ref{kat:MealPlan}) musi mieć przypisany przynajmniej jeden dzień (\ref{kat:MealPlanDay})
        \item Jadłospis (\ref{kat:MealPlan}) może mieć przypisanych maksymalnie 31 dni (\ref{kat:MealPlanDay})
        \item Jadłospis (\ref{kat:MealPlan}) musi mieć przypisaną przynajmniej jedną definicję posiłku (\ref{kat:MealDefinition})
        \item Jadłospis (\ref{kat:MealPlan}) może mieć przypisanych maksymalnie 10 definicji posiłków (\ref{kat:MealDefinition})
        \item Jadłospis (\ref{kat:MealPlan}) nie musi mieć przypisanego żadnego odpowiedniego typu diety (\ref{kat:DietType})
        \item Jadłospis (\ref{kat:MealPlan}) może mieć przypisanych wiele odpowiednich typów diety (\ref{kat:DietType})
        \item Jadłospis (\ref{kat:MealPlan}) nie musi mieć przypisanego żadnego nieodpowiedniego typu diety (\ref{kat:DietType})
        \item Jadłospis (\ref{kat:MealPlan}) może mieć przypisanych wiele nieodpowiednich typów diety (\ref{kat:DietType})
        \item Jadłospis (\ref{kat:MealPlan}) musi mieć dokładnie jednego autora (\ref{kat:User})
        %CRUD
        \item \role{Dietetyk} może wyświetlać publiczne jadłospisy (\ref{kat:MealPlan})
        \item \role{Dietetyk} może dodawać, wyświetlać, edytować i~usuwać własne jadłospisy (\ref{kat:MealPlan})
        \item \role{Administrator} może wyświetlać i~usuwać jadłospisy (\ref{kat:MealPlan})
    \end{enumerate}
    \item\ref{kat:MealPlanDay}
    \begin{enumerate}[label={\textbf{REG/\protect\threedigits{\arabic{enumi}}}}, wide, labelwidth=!, align=left, leftmargin=3cm, resume]
        %Relacje
        \item Dzień jadłospisu (\ref{kat:MealPlanDay}) musi być przypisany do dokładnie jednego jadłospisu (\ref{kat:MealPlan})
        \item Dzień jadłospisu (\ref{kat:MealPlanDay}) nie musi mieć przypisanego żadnego posiłku (\ref{kat:Meal})
        \item Dzień jadłospisu (\ref{kat:MealPlanDay}) może mieć przypisanych maksymalnie 10 posiłków (\ref{kat:Meal})
        %CRUD
        \item Dzień jadłospisu (\ref{kat:MealPlanDay}) jest przedmiotem kompozycji ze strony jadłospisu (\ref{kat:MealPlan})
    \end{enumerate}
    \item\ref{kat:Meal}
    \begin{enumerate}[label={\textbf{REG/\protect\threedigits{\arabic{enumi}}}}, wide, labelwidth=!, align=left, leftmargin=3cm, resume]
        %Relacje
        \item Posiłek (\ref{kat:Meal}) musi być przypisany do dokładnie jednego dnia jadłospisu (\ref{kat:MealPlanDay})
        \item Posiłek (\ref{kat:Meal}) nie musi mieć przypisanego żadnego produktu (\ref{kat:MealProduct})
        \item Posiłek (\ref{kat:Meal}) może mieć przypisanych wiele produktów (\ref{kat:MealProduct})
        \item Posiłek (\ref{kat:Meal}) nie musi mieć przypisanego żadnego przepisu (\ref{kat:MealRecipe})
        \item Posiłek (\ref{kat:Meal}) może mieć przypisanych wiele przepisów (\ref{kat:MealRecipe})
        %CRUD
        \item Posiłek (\ref{kat:Meal}) jest przedmiotem kompozycji ze strony dnia jadłospisu (\ref{kat:MealPlanDay})
    \end{enumerate}
    \item\ref{kat:MealRecipe}
    \begin{enumerate}[label={\textbf{REG/\protect\threedigits{\arabic{enumi}}}}, wide, labelwidth=!, align=left, leftmargin=3cm, resume]
        %Relacje
        \item Przepis posiłku (\ref{kat:MealRecipe}) musi być przypisany do dokładnie jednego posiłku (\ref{kat:Meal})
        \item Przepis posiłku (\ref{kat:MealRecipe}) musi mieć przypisany dokładnie jeden przepis (\ref{kat:Recipe})
        %CRUD
        \item Przepis posiłku (\ref{kat:MealRecipe}) jest przedmiotem kompozycji ze strony posiłku (\ref{kat:Meal})
    \end{enumerate}
    \item\ref{kat:MealProduct}
    \begin{enumerate}[label={\textbf{REG/\protect\threedigits{\arabic{enumi}}}}, wide, labelwidth=!, align=left, leftmargin=3cm, resume]
        %Relacje
        \item Produkt posiłku (\ref{kat:MealProduct}) musi być przypisany do dokładnie jednego posiłku (\ref{kat:Meal})
        \item Produkt posiłku (\ref{kat:MealProduct}) musi mieć przypisany dokładnie jeden produkt (\ref{kat:Product})
        \item Produkt posiłku (\ref{kat:MealProduct}) nie musi mieć przypisanej żadnej miary domowej (\ref{kat:HouseholdMeasure})
        \item Produkt posiłku (\ref{kat:MealProduct}) musi mieć przypisaną maksymalnie jedną miarę domową (\ref{kat:HouseholdMeasure})
        %CRUD
        \item Produkt posiłku (\ref{kat:MealProduct}) jest przedmiotem kompozycji ze strony posiłku (\ref{kat:Meal})
    \end{enumerate}
    \item\ref{kat:MealDefinition}
    \begin{enumerate}[label={\textbf{REG/\protect\threedigits{\arabic{enumi}}}}, wide, labelwidth=!, align=left, leftmargin=3cm, resume]
        %Relacje
        \item Definicja posiłku (\ref{kat:MealDefinition}) musi być przypisana do dokładnie jednego jadłospisu (\ref{kat:MealPlan})
        \item Definicja posiłku (\ref{kat:MealDefinition}) musi mieć przypisany dokładnie jeden typ posiłku (\ref{kat:MealType})
        %CRUD
        \item Definicja posiłku (\ref{kat:MealDefinition}) jest przedmiotem kompozycji ze strony jadłospisu (\ref{kat:MealPlan})
    \end{enumerate}
    \item\ref{kat:Appointment}
    \begin{enumerate}[label={\textbf{REG/\protect\threedigits{\arabic{enumi}}}}, wide, labelwidth=!, align=left, leftmargin=3cm, resume]
        %Relacje
        \item Wizyta (\ref{kat:Appointment}) musi być przypisana do dokładnie jednej karty pacjenta (\ref{kat:PatientCard})
        \item Wizyta (\ref{kat:Appointment}) nie musi mieć przypisanej żadnej ewaluacji (\ref{kat:AppointmentEvaluation})
        \item Wizyta (\ref{kat:Appointment}) może mieć przypisaną maksymalnie jedną ewaluację (\ref{kat:AppointmentEvaluation})
        \item Wizyta (\ref{kat:Appointment}) nie musi mieć przypisanego żadnych pomiarów ciała (\ref{kat:BodyMeasurement})
        \item Wizyta (\ref{kat:Appointment}) może mieć przypisane maksymalnie jedne pomiary ciała (\ref{kat:BodyMeasurement})
        \item Wizyta (\ref{kat:Appointment}) nie musi mieć przypisanego żadnego wywiadu żywieniowego (\ref{kat:NutritionalInterview})
        \item Wizyta (\ref{kat:Appointment}) może mieć przypisany maksymalnie jeden wywiad żywieniowy (\ref{kat:NutritionalInterview})
        \item Wizyta (\ref{kat:Appointment}) nie musi mieć przypisanego żadnego jadłospisu (\ref{kat:AssignedMealPlan})
        \item Wizyta (\ref{kat:Appointment}) może mieć przypisanych wiele jadłospisów (\ref{kat:AssignedMealPlan})
        %CRUD
        \item \role{Pacjent} może wyświetlać swoją wizytę (\ref{kat:Appointment})
        \item \role{Dietetyk} może dodawać nową wizytę (\ref{kat:Appointment})
        \item \role{Dietetyk} może wyświetlać i~edytować swoją wizytę (\ref{kat:Appointment})
    \end{enumerate}
    \item\ref{kat:PatientCard}
    \begin{enumerate}[label={\textbf{REG/\protect\threedigits{\arabic{enumi}}}}, wide, labelwidth=!, align=left, leftmargin=3cm, resume]
        %Relacje
        \item Karta pacjenta (\ref{kat:PatientCard}) nie musi mieć przypisanej żadnej wizyty (\ref{kat:Appointment})
        \item Karta pacjenta (\ref{kat:PatientCard}) może mieć przypisanych wiele wizyt (\ref{kat:Appointment})
        \item Karta pacjenta (\ref{kat:PatientCard}) musi mieć przypisanego dokładnie jednego pacjenta (\ref{kat:User})
        \item Karta pacjenta (\ref{kat:PatientCard}) musi mieć przypisanego dokładnie jednego dietetyka (\ref{kat:User})
        %CRUD
        \item \role{Pacjent} może wyświetlać swoją kartę pacjenta (\ref{kat:PatientCard})
        \item \role{Dietetyk} może dodawać nową kartę pacjenta (\ref{kat:PatientCard})
        \item \role{Dietetyk} może wyświetlać i~edytować karty pacjenta (\ref{kat:PatientCard}), którymi zarządza
    \end{enumerate}
    \item\ref{kat:AppointmentEvaluation}
    \begin{enumerate}[label={\textbf{REG/\protect\threedigits{\arabic{enumi}}}}, wide, labelwidth=!, align=left, leftmargin=3cm, resume]
        %Relacje
        \item Ewaluacja wizyty (\ref{kat:AppointmentEvaluation}) musi być przypisana do dokładnie jednej wizyty (\ref{kat:Appointment})
        %CRUD
        \item \role{Użytkownik} może wyświetlać ewaluację wizyty (\ref{kat:AppointmentEvaluation})
        \item \role{Pacjent} może dodawać ewaluację (\ref{kat:AppointmentEvaluation}) do swojej wizyty (\ref{kat:Appointment})
    \end{enumerate}
    \item\ref{kat:BodyMeasurement}
    \begin{enumerate}[label={\textbf{REG/\protect\threedigits{\arabic{enumi}}}}, wide, labelwidth=!, align=left, leftmargin=3cm, resume]
        %Relacje
        \item Pomiary ciała (\ref{kat:BodyMeasurement}) muszą być przypisane do dokładnie jednej wizyty (\ref{kat:Appointment})
        %CRUD
        \item Pomiary ciała (\ref{kat:BodyMeasurement}) są przedmiotem kompozycji ze strony wizyty (\ref{kat:Appointment})
    \end{enumerate}
    \item\ref{kat:NutritionalInterview}
    \begin{enumerate}[label={\textbf{REG/\protect\threedigits{\arabic{enumi}}}}, wide, labelwidth=!, align=left, leftmargin=3cm, resume]
        %Relacje
        \item Wywiad żywieniowy (\ref{kat:NutritionalInterview}) musi być przypisany do dokładnie jednej wizyty (\ref{kat:Appointment})
        \item Wywiad żywieniowy (\ref{kat:NutritionalInterview}) nie musi mieć przypisanego żadnego niestandardowego pytania (\ref{kat:CustomNutritionalInterviewQuestion})
        \item Wywiad żywiniowy (\ref{kat:NutritionalInterview}) może mieć przypisanych wiele niestandardowych pytań (\ref{kat:CustomNutritionalInterviewQuestion})
        \item Wywiad żywieniowy (\ref{kat:NutritionalInterview}) nie musi mieć przypisanych żadnych posiadanych sprzętów kuchennych (\ref{kat:KitchenAppliance})
        \item Wywiad żywieniowy (\ref{kat:NutritionalInterview}) może mieć przypisanych wiele posiadanych sprzętów kuchennych (\ref{kat:KitchenAppliance})
        %CRUD
        \item Wywiad żywieniowy (\ref{kat:NutritionalInterview}) jest przedmiotem kompozycji ze strony wizyty (\ref{kat:Appointment})
    \end{enumerate}
    \item\ref{kat:CustomNutritionalInterviewQuestion}
    \begin{enumerate}[label={\textbf{REG/\protect\threedigits{\arabic{enumi}}}}, wide, labelwidth=!, align=left, leftmargin=3cm, resume]
        %Relacje
        \item Niestandardowe pytanie żywieniowe (\ref{kat:CustomNutritionalInterviewQuestion}) musi być przypisane do dokładnie jednego wywiadu żywieniowego (\ref{kat:NutritionalInterview})
        %CRUD
        \item  Niestandardowe pytanie żywieniowe (\ref{kat:CustomNutritionalInterviewQuestion}) jest przedmiotem kompozycji ze strony wywiadu żywieniowego (\ref{kat:NutritionalInterview})
    \end{enumerate}
    \item\ref{kat:CustomNutritionalInterviewQuestionTemplate}
    \begin{enumerate}[label={\textbf{REG/\protect\threedigits{\arabic{enumi}}}}, wide, labelwidth=!, align=left, leftmargin=3cm, resume]
        %Relacje
        \item Szablon niestandardowego pytania żywieniowego (\ref{kat:CustomNutritionalInterviewQuestionTemplate}) musi mieć dokaldnie jednego autora (\ref{kat:User})
        %CRUD
        \item \role{Dietetyk} może dodawać szablon niestandardowego pytania żywieniowego (\ref{kat:CustomNutritionalInterviewQuestionTemplate})
        \item \role{Dietetyk} może wyświetlać, edytować i~usuwać swój szablon niestandardowego pytania żywieniowego (\ref{kat:CustomNutritionalInterviewQuestionTemplate})
    \end{enumerate}
    \item\ref{kat:AssignedMealPlan}
    \begin{enumerate}[label={\textbf{REG/\protect\threedigits{\arabic{enumi}}}}, wide, labelwidth=!, align=left, leftmargin=3cm, resume]
        %Relacje
        \item Przypisany jadłospis (\ref{kat:AssignedMealPlan}) musi mieć przydzieloną dokładnie jedną wizytę (\ref{kat:Appointment})
        \item Przypisany jadłospis (\ref{kat:AssignedMealPlan}) musi mieć przydzielony dokładnie jeden jadłospis (\ref{kat:MealPlan})
        %CRUD
        \item Przypisany jadłospis (\ref{kat:AssignedMealPlan}) jest przedmiotem kompozycji ze strony wizyty (\ref{kat:Appointment})
    \end{enumerate}
\end{itemize}

\section{Ograniczenia dziedzinowe}\label{sec:restrictions}

\begin{itemize}[label={\textbf{Ograniczenia dla}}, wide, labelwidth=!, labelindent=0pt]
    \setlength\itemsep{1em}
    \item[\textbf{Ograniczenia}] \textbf{ogólne}
    \begin{enumerate}[label={\textbf{OGR/\protect\threedigits{\arabic{enumi}}}}, wide, labelwidth=!, align=left, leftmargin=3cm]
        \item Wszystkie \textbf{id} muszą być być unikalne
        \item Wszystkie \textbf{id} są wymagane
        \item Wszystkie \textbf{id} są liczbami całkowitymi dodatnimi tworzonymi przez SZBD za pomocą autonumerowania
        \item Wszystkie atrybuty \textbf{language} są wymagane
        \item Wszystkie \textbf{language} są ciągami znaków o~długości 2~znaków spełniającymi normę ISO 639-1
        \item Wszystkie \textbf{stemple czasowe} są w~formacie YYYY:MM:DD HH:MI:SS
        \item Wszystkie \textbf{daty} są w~formacie YYYY:MM:DD
        \item Ciągi znaków bez dodatkowych ograniczeń mogą zawierać dowolne znaki dopuszczalne w~systemie kodowania UTF-8
    \end{enumerate}
    \item\ref{kat:User}
    \begin{enumerate}[label={\textbf{OGR/\protect\threedigits{\arabic{enumi}}}}, wide, labelwidth=!, align=left, leftmargin=3cm, resume]
        %Required
        \item Atrybut \ref{kat:User:login} jest wymagany
        \item Atrybut \ref{kat:User:passwordHash} jest wymagany
        \item Atrybut flagę \ref{kat:User:activated} jest wymagany
        \item Atrybut \ref{kat:User:createdDate} jest wymagany
        %Unique
        \item Atrybut \ref{kat:User:login} ma unikalną wartość
        \item Atrybut \ref{kat:User:email} ma unikalną wartość
        %Type
        \item Atrybut \ref{kat:User:login} jest ciągiem znaków składającym się z~liter, cyfr i~dodatkowo mogącym zawierać znaki ".", "\_", "-", "@" o~długości od 1~do 50 znaków
        \item Atrybut \ref{kat:User:passwordHash} jest ciągiem znaków o~długości 60 znaków
        \item Atrybut \ref{kat:User:firstName} jest ciagiem znaków o~długości do 50 znaków
        \item Atrybut \ref{kat:User:lastName} jest ciagiem znaków o~długości do 50 znaków
        \item Atrybut \ref{kat:User:email} jest ciagiem znaków o~długości od 5~do 254 znaków
        \item Atrybut \ref{kat:User:activated} jest typem logicznym
        \item Atrybut \ref{kat:User:image} jest ciągiem znaków o~długości do 256 znaków tworzącym poprawny adres URL
        \item Atrybut \ref{kat:User:activationKey} jest ciągiem znaków o~długości 20 znaków
        \item Atrybut \ref{kat:User:resetKey} jest ciągiem znaków o~długości 20 znaków
        \item Atrybut \ref{kat:User:resetDate} jest stemplem czasowym
        \item Atrybut \ref{kat:User:createdDate} jest stemplem czasowym
        \item Atrybut \ref{kat:User:lastModifiedDate} jest stemplem czasowym
    \end{enumerate}
    \item\ref{kat:Authority}
    \begin{enumerate}[label={\textbf{OGR/\protect\threedigits{\arabic{enumi}}}}, wide, labelwidth=!, align=left, leftmargin=3cm, resume]
        \item Atrybut \ref{kat:Authority:name} jest wymagany
        \item Atrybut \ref{kat:Authority:name} ma unikalną wartość
        \item Atrybut \ref{kat:Authority:name} jest ciągiem znaków składającym się z~liter i~znaków "\_" o~długości od 1~do 255 znaków
    \end{enumerate}
    \item\ref{kat:UserExtraInfo}
    \begin{enumerate}[label={\textbf{OGR/\protect\threedigits{\arabic{enumi}}}}, wide, labelwidth=!, align=left, leftmargin=3cm, resume]
        \item Atrybut \ref{kat:UserExtraInfo:gender} jest typu wyliczeniowego i~może przyjmować wartości "FEMALE", "MALE", "OTHER"
        \item Atrybut \ref{kat:UserExtraInfo:dateOfBirth} jest datą
        \item Atrybut \ref{kat:UserExtraInfo:phoneNumber} jest ciągiem znaków o~długości od 1~do 50 znaków
        \item Atrybut \ref{kat:UserExtraInfo:streetAddress} jest ciągiem znaków o~długości od 1~do 255 znaków
        \item Atrybut \ref{kat:UserExtraInfo:postalCode} jest ciągiem znaków o~długości od 1~do 20 znaków
        \item Atrybut \ref{kat:UserExtraInfo:city} jest ciągiem znaków o~długości od 1~do 50 znaków
        \item Atrybut \ref{kat:UserExtraInfo:country} jest ciągiem znaków o~długości od 1~do 50 znaków
        \item Atrybut \ref{kat:UserExtraInfo:personalDescription} jest ciągiem znaków
    \end{enumerate}

    \item\ref{kat:SiteContent}
    \begin{enumerate}[label={\textbf{OGR/\protect\threedigits{\arabic{enumi}}}}, wide, labelwidth=!, align=left, leftmargin=3cm, resume]
        \item Atrybut \ref{kat:SiteContent:ordinalNumber} jest wymagany
        \item Atrybut \ref{kat:SiteContent:siteContentType} jest wymagany
        \item Atrybut \ref{kat:SiteContent:description} jest wymagany

        \item Atrybut \ref{kat:SiteContent:ordinalNumber} jest liczbą całkowitą
        \item Atrybut \ref{kat:SiteContent:siteContentType} jest typu wyliczeniowego i~może przyjmować wartości "LANDING\_PAGE\_CARD", "TERMS\_OF\_SERVICE", "PRIVACY\_POLICY", "FREQUENTLY\_ASKED\_QUESTION"
        \item Atrybut \ref{kat:SiteContent:title} jest ciągiem znaków o~długości od 1~do 255 znaków
        \item Atrybut \ref{kat:SiteContent:description} jest ciągiem znaków
        \item Atrybut \ref{kat:SiteContent:image} jest zdjęciem o~maksymalnym rozmiarze 5000000 bajtów
    \end{enumerate}

    \item\ref{kat:SiteContentTranslation}
    \begin{enumerate}[label={\textbf{OGR/\protect\threedigits{\arabic{enumi}}}}, wide, labelwidth=!, align=left, leftmargin=3cm, resume]
        \item Atrybut \ref{kat:SiteContentTranslation:description} jest wymagany

        \item Atrybut \ref{kat:SiteContentTranslation:title} jest ciągiem znaków o~długości od 1~do 255 znaków
        \item Atrybut \ref{kat:SiteContentTranslation:description} jest ciągiem znaków
    \end{enumerate}

    \item\ref{kat:ContactInfo}
    \begin{enumerate}[label={\textbf{OGR/\protect\threedigits{\arabic{enumi}}}}, wide, labelwidth=!, align=left, leftmargin=3cm, resume]
        \item Atrybut \ref{kat:ContactInfo:contactInfoType} jest wymagany
        \item Atrybut \ref{kat:ContactInfo:description} jest wymagany

        \item Atrybut \ref{kat:ContactInfo:contactInfoType} jest typu wyliczeniowego i~może przyjmować wartości "PHONE", "EMAIL", "ADDRESS", "FACEBOOK", "TWITTER", "INSTAGRAM", "ANDROID", "IOS", "WORKING\_HOURS", "OTHER"
        \item Atrybut \ref{kat:ContactInfo:description} jest ciągiem znaków o~długości od 1~do 255 znaków
    \end{enumerate}

    \item\ref{kat:Pricing}
    \begin{enumerate}[label={\textbf{OGR/\protect\threedigits{\arabic{enumi}}}}, wide, labelwidth=!, align=left, leftmargin=3cm, resume]
        \item Atrybut \ref{kat:Pricing:ordinalNumber} jest wymagany
        \item Atrybut \ref{kat:Pricing:description} jest wymagany
        \item Atrybut \ref{kat:Pricing:price} jest wymagany
        \item Atrybut \ref{kat:Pricing:currency} jest wymagany

        \item Atrybut \ref{kat:Pricing:ordinalNumber} jest liczbą całkowitą większą od 0
        \item Atrybut \ref{kat:Pricing:title} jest ciągiem znaków o~długości od 1~do 255 znaków
        \item Atrybut \ref{kat:Pricing:description} jest ciągiem znaków
        \item Atrybut \ref{kat:Pricing:price} jest ciągiem znaków o~długości od 1~do 8~znaków
        \item Atrybut \ref{kat:Pricing:currency} jest ciągiem znaków o~długości 3~znaków spełniającym normę ISO 4217
        \item Jeżeli w~\ref{kat:Pricing:price} występuje znak "." to muszą po nim występować dokładnie 2~cyfry
    \end{enumerate}

    \item\ref{kat:PricingTranslation}
    \begin{enumerate}[label={\textbf{OGR/\protect\threedigits{\arabic{enumi}}}}, wide, labelwidth=!, align=left, leftmargin=3cm, resume]
        \item Atrybut \ref{kat:PricingTranslation:description} jest wymagany

        \item Atrybut \ref{kat:PricingTranslation:title} jest ciągiem znaków o~długości od 1~do 255 znaków
        \item Atrybut \ref{kat:PricingTranslation:description} jest ciągiem znaków
    \end{enumerate}

    \item\ref{kat:Product}
    \begin{enumerate}[label={\textbf{OGR/\protect\threedigits{\arabic{enumi}}}}, wide, labelwidth=!, align=left, leftmargin=3cm, resume]
        \item Atrybut \ref{kat:Product:isPublic} jest wymagany

        \item Atrybut \ref{kat:Product:source} jest ciągiem znaków o~długości od 1~do 255 znaków
        \item Atrybut \ref{kat:Product:isPublic} jest typu logicznego
    \end{enumerate}

    \item\ref{kat:ProductVersion}
    \begin{enumerate}[label={\textbf{OGR/\protect\threedigits{\arabic{enumi}}}}, wide, labelwidth=!, align=left, leftmargin=3cm, resume]
        \item Atrybut \ref{kat:ProductVersion:createdDate} jest wymagany
        \item Atrybut \ref{kat:ProductVersion:description} jest wymagany

        \item Atrybut \ref{kat:ProductVersion:createdDate} jest stemplem czasowym
        \item Atrybut \ref{kat:ProductVersion:description} jest ciągiem znaków o~długości od 1~do 255 znaków
    \end{enumerate}

    \item\ref{kat:ProductBasicNutritionData}
    \begin{enumerate}[label={\textbf{OGR/\protect\threedigits{\arabic{enumi}}}}, wide, labelwidth=!, align=left, leftmargin=3cm, resume]
        \item Atrybut \ref{kat:ProductBasicNutritionData:energy} jest wymagany
        \item Atrybut \ref{kat:ProductBasicNutritionData:protein} jest wymagany
        \item Atrybut \ref{kat:ProductBasicNutritionData:fat} jest wymagany
        \item Atrybut \ref{kat:ProductBasicNutritionData:carbohydrates} jest wymagany

        \item Atrybut \ref{kat:ProductBasicNutritionData:energy} jest liczbą rzeczywistą nie mniejszą niż 0
        \item Atrybut \ref{kat:ProductBasicNutritionData:protein} jest liczbą rzeczywistą nie mniejszą niż 0
        \item Atrybut \ref{kat:ProductBasicNutritionData:fat} jest liczbą rzeczywistą nie mniejszą niż 0
        \item Atrybut \ref{kat:ProductBasicNutritionData:carbohydrates} jest liczbą rzeczywistą nie mniejszą niż 0
    \end{enumerate}

    \item\ref{kat:NutritionData}
    \begin{enumerate}[label={\textbf{OGR/\protect\threedigits{\arabic{enumi}}}}, wide, labelwidth=!, align=left, leftmargin=3cm, resume]
        \item Atrybut \ref{kat:NutritionData:nutritionValue} jest wymagany

        \item Atrybut \ref{kat:NutritionData:nutritionValue} jest liczbą rzeczywistą nie mniejszą niż 0
    \end{enumerate}

    \item\ref{kat:NutritionDefinition}
    \begin{enumerate}[label={\textbf{OGR/\protect\threedigits{\arabic{enumi}}}}, wide, labelwidth=!, align=left, leftmargin=3cm, resume]
        \item Atrybut \ref{kat:NutritionDefinition:tag} jest wymagany
        \item Atrybut \ref{kat:NutritionDefinition:description} jest wymagany
        \item Atrybut \ref{kat:NutritionDefinition:units} jest wymagany
        \item Atrybut \ref{kat:NutritionDefinition:decimalPlaces} jest wymagany

        \item Atrybut \ref{kat:NutritionDefinition:tag} ma unikalną wartość

        \item Atrybut \ref{kat:NutritionDefinition:tag} jest ciągiem znaków o~długości od 1~do 20 znaków
        \item Atrybut \ref{kat:NutritionDefinition:description} jest ciągiem znaków o~długości od 1~do 255 znaków
        \item Atrybut \ref{kat:NutritionDefinition:units} jest ciągiem znaków o~długości od 1~do 10 znaków
        \item Atrybut \ref{kat:NutritionDefinition:decimalPlaces} jest liczbą całkowitą nie mniejszą niż 0
    \end{enumerate}

    \item\ref{kat:NutritionDefinitionTranslation}
    \begin{enumerate}[label={\textbf{OGR/\protect\threedigits{\arabic{enumi}}}}, wide, labelwidth=!, align=left, leftmargin=3cm, resume]
        \item Atrybut \ref{kat:NutritionDefinitionTranslation:translation} jest wymagany

        \item Atrybut \ref{kat:NutritionDefinitionTranslation:translation} jest ciągiem znaków o~długości od 1~do 255 znaków
    \end{enumerate}

    \item\ref{kat:HouseholdMeasure}
    \begin{enumerate}[label={\textbf{OGR/\protect\threedigits{\arabic{enumi}}}}, wide, labelwidth=!, align=left, leftmargin=3cm, resume]
        \item Atrybut \ref{kat:HouseholdMeasure:description} jest wymagany
        \item Atrybut \ref{kat:HouseholdMeasure:gramsWeight} jest wymagany
        \item Atrybut \ref{kat:HouseholdMeasure:isVisible} jest wymagany

        \item Atrybut \ref{kat:HouseholdMeasure:description} jest ciągiem znaków o~długości od 1~do 255 znaków
        \item Atrybut \ref{kat:HouseholdMeasure:gramsWeight} jest liczbą rzeczywistą nie mniejszą niż 0
        \item Atrybut \ref{kat:HouseholdMeasure:isVisible} jest typu logicznego
    \end{enumerate}

    \item\ref{kat:ProductSubcategory}
    \begin{enumerate}[label={\textbf{OGR/\protect\threedigits{\arabic{enumi}}}}, wide, labelwidth=!, align=left, leftmargin=3cm, resume]
        \item Atrybut \ref{kat:ProductSubcategory:description} jest wymagany

        \item Atrybut \ref{kat:ProductSubcategory:description} jest ciągiem znaków o~długości od 1~do 255 znaków
    \end{enumerate}

    \item\ref{kat:ProductCategory}
    \begin{enumerate}[label={\textbf{OGR/\protect\threedigits{\arabic{enumi}}}}, wide, labelwidth=!, align=left, leftmargin=3cm, resume]
        \item Atrybut \ref{kat:ProductCategory:description} jest wymagany

        \item Atrybut \ref{kat:ProductCategory:description} ma unikalną wartość

        \item Atrybut \ref{kat:ProductCategory:description} jest ciągiem znaków o~długości od 1~do 255 znaków
    \end{enumerate}

    \item\ref{kat:ProductCategoryTranslation}
    \begin{enumerate}[label={\textbf{OGR/\protect\threedigits{\arabic{enumi}}}}, wide, labelwidth=!, align=left, leftmargin=3cm, resume]
        \item Atrybut \ref{kat:ProductCategoryTranslation:translation} jest wymagany

        \item Atrybut \ref{kat:ProductCategoryTranslation:translation} jest ciągiem znaków o~długości od 1~do 255 znaków
    \end{enumerate}

    \item\ref{kat:DietType}
    \begin{enumerate}[label={\textbf{OGR/\protect\threedigits{\arabic{enumi}}}}, wide, labelwidth=!, align=left, leftmargin=3cm, resume]
        \item Atrybut \ref{kat:DietType:name} jest wymagany

        \item Atrybut \ref{kat:DietType:name} ma unikalną wartość

        \item Atrybut \ref{kat:DietType:name} jest ciągiem znaków o~długości od 1~do 255 znaków
    \end{enumerate}

    \item\ref{kat:DietTypeTranslation}
    \begin{enumerate}[label={\textbf{OGR/\protect\threedigits{\arabic{enumi}}}}, wide, labelwidth=!, align=left, leftmargin=3cm, resume]
        \item Atrybut \ref{kat:DietTypeTranslation:translation} jest wymagany

        \item Atrybut \ref{kat:DietTypeTranslation:translation} jest ciągiem znaków o~długości od 1~do 255 znaków
    \end{enumerate}

    \item\ref{kat:Recipe}
    \begin{enumerate}[label={\textbf{OGR/\protect\threedigits{\arabic{enumi}}}}, wide, labelwidth=!, align=left, leftmargin=3cm, resume]
        \item Atrybut \ref{kat:Recipe:isPublic} jest wymagany

        \item Atrybut \ref{kat:Recipe:isPublic} jest typu logicznego
    \end{enumerate}

    \item\ref{kat:RecipeVersion}
    \begin{enumerate}[label={\textbf{OGR/\protect\threedigits{\arabic{enumi}}}}, wide, labelwidth=!, align=left, leftmargin=3cm, resume]
        \item Atrybut \ref{kat:RecipeVersion:name} jest wymagany
        \item Atrybut \ref{kat:RecipeVersion:preparationTimeMinutes} jest wymagany
        \item Atrybut \ref{kat:RecipeVersion:numberOfPortions} jest wymagany
        \item Atrybut \ref{kat:RecipeVersion:totalGramsWeight} jest wymagany

        \item Atrybut \ref{kat:RecipeVersion:editTimestamp} jest stemplem czasowym
        \item Atrybut \ref{kat:RecipeVersion:name} jest ciągiem znaków o~długości od 1~do 255 znaków
        \item Atrybut \ref{kat:RecipeVersion:preparationTimeMinutes} jest liczbą całkowitą nie mniejszą niż 0
        \item Atrybut \ref{kat:RecipeVersion:numberOfPortions} jest liczbą rzeczywistą nie mniejszą niż 0
        \item Atrybut \ref{kat:RecipeVersion:image} jest zdjęciem o~maksymalnym rozmiarze 5000000 bajtów
        \item Atrybut \ref{kat:RecipeVersion:totalGramsWeight} jest liczbą rzeczywistą nie mniejszą niż 0
    \end{enumerate}

    \item\ref{kat:RecipeBasicNutritionData}
    \begin{enumerate}[label={\textbf{OGR/\protect\threedigits{\arabic{enumi}}}}, wide, labelwidth=!, align=left, leftmargin=3cm, resume]
        \item Atrybut \ref{kat:RecipeBasicNutritionData:energy} jest wymagany
        \item Atrybut \ref{kat:RecipeBasicNutritionData:protein} jest wymagany
        \item Atrybut \ref{kat:RecipeBasicNutritionData:fat} jest wymagany
        \item Atrybut \ref{kat:RecipeBasicNutritionData:carbohydrates} jest wymagany

        \item Atrybut \ref{kat:RecipeBasicNutritionData:energy} jest liczbą całkowitą nie mniejszą niż 0
        \item Atrybut \ref{kat:RecipeBasicNutritionData:protein} jest liczbą całkowitą nie mniejszą niż 0
        \item Atrybut \ref{kat:RecipeBasicNutritionData:fat} jest liczbą całkowitą nie mniejszą niż 0
        \item Atrybut \ref{kat:RecipeBasicNutritionData:carbohydrates} jest liczbą całkowitą nie mniejszą niż 0
    \end{enumerate}

    \item\ref{kat:RecipeSection}
    \begin{enumerate}[label={\textbf{OGR/\protect\threedigits{\arabic{enumi}}}}, wide, labelwidth=!, align=left, leftmargin=3cm, resume]
        \item Atrybut \ref{kat:RecipeSection:sectionName} jest ciągiem znaków o~długości od 1~do 255 znaków
    \end{enumerate}

    \item\ref{kat:ProductPortion}
    \begin{enumerate}[label={\textbf{OGR/\protect\threedigits{\arabic{enumi}}}}, wide, labelwidth=!, align=left, leftmargin=3cm, resume]
        \item Atrybut \ref{kat:ProductPortion:amount} jest wymagany

        \item Atrybut \ref{kat:ProductPortion:amount} jest liczbą rzeczywistą nie mniejszą niż 0
    \end{enumerate}

    \item\ref{kat:PreparationStep}
    \begin{enumerate}[label={\textbf{OGR/\protect\threedigits{\arabic{enumi}}}}, wide, labelwidth=!, align=left, leftmargin=3cm, resume]
        \item Atrybut \ref{kat:PreparationStep:ordinalNumber} jest wymagany

        \item Atrybut \ref{kat:PreparationStep:ordinalNumber} jest liczbą całkowitą nie mniejszą niż 1
        \item Atrybut \ref{kat:PreparationStep:stepDescription} jest ciągiem znaków
    \end{enumerate}

    \item\ref{kat:KitchenAppliance}
    \begin{enumerate}[label={\textbf{OGR/\protect\threedigits{\arabic{enumi}}}}, wide, labelwidth=!, align=left, leftmargin=3cm, resume]
        \item Atrybut \ref{kat:KitchenAppliance:name} jest wymagany

        \item Atrybut \ref{kat:KitchenAppliance:name} ma unikalną wartość

        \item Atrybut \ref{kat:KitchenAppliance:name} jest ciągiem znaków o~długości od 1~do 255 znaków
    \end{enumerate}

    \item\ref{kat:KitchenApplianceTranslation}
    \begin{enumerate}[label={\textbf{OGR/\protect\threedigits{\arabic{enumi}}}}, wide, labelwidth=!, align=left, leftmargin=3cm, resume]
        \item Atrybut \ref{kat:KitchenApplianceTranslation:translation} jest wymagany

        \item Atrybut \ref{kat:KitchenApplianceTranslation:translation} jest ciągiem znaków o~długości od 1~do 255 znaków
    \end{enumerate}

    \item\ref{kat:DishType}
    \begin{enumerate}[label={\textbf{OGR/\protect\threedigits{\arabic{enumi}}}}, wide, labelwidth=!, align=left, leftmargin=3cm, resume]
        \item Atrybut \ref{kat:DishType:description} jest wymagany

        \item Atrybut \ref{kat:DishType:description} ma unikalną wartość

        \item Atrybut \ref{kat:DishType:description} jest ciągiem znaków o~długości od 1~do 255 znaków
    \end{enumerate}

    \item\ref{kat:DishTypeTranslation}
    \begin{enumerate}[label={\textbf{OGR/\protect\threedigits{\arabic{enumi}}}}, wide, labelwidth=!, align=left, leftmargin=3cm, resume]
        \item Atrybut \ref{kat:DishTypeTranslation:translation} jest wymagany

        \item Atrybut \ref{kat:DishTypeTranslation:translation} jest ciągiem znaków o~długości od 1~do 255 znaków
    \end{enumerate}

    \item\ref{kat:MealType}
    \begin{enumerate}[label={\textbf{OGR/\protect\threedigits{\arabic{enumi}}}}, wide, labelwidth=!, align=left, leftmargin=3cm, resume]
        \item Atrybut \ref{kat:MealType:name} jest wymagany

        \item Atrybut \ref{kat:MealType:name} ma unikalną wartość

        \item Atrybut \ref{kat:MealType:name} jest ciągiem znaków o~długości od 1~do 255 znaków
    \end{enumerate}

    \item\ref{kat:MealTypeTranslation}
    \begin{enumerate}[label={\textbf{OGR/\protect\threedigits{\arabic{enumi}}}}, wide, labelwidth=!, align=left, leftmargin=3cm, resume]
        \item Atrybut \ref{kat:MealTypeTranslation:translation} jest wymagany

        \item Atrybut \ref{kat:MealTypeTranslation:translation} jest ciągiem znaków o~długości od 1~do 255 znaków
    \end{enumerate}

    \item\ref{kat:MealPlan}
    \begin{enumerate}[label={\textbf{OGR/\protect\threedigits{\arabic{enumi}}}}, wide, labelwidth=!, align=left, leftmargin=3cm, resume]
        \item Atrybut \ref{kat:MealPlan:creationTimestamp} jest wymagany
        \item Atrybut \ref{kat:MealPlan:editTimestamp} jest wymagany
        \item Atrybut \ref{kat:MealPlan:isVisible} jest wymagany
        \item Atrybut \ref{kat:MealPlan:numberOfDays}  jest wymagany
        \item Atrybut \ref{kat:MealPlan:numberOfMealsPerDay} jest wymagany
        \item Atrybut \ref{kat:MealPlan:totalDailyEnergy} jest wymagany
        \item Atrybut \ref{kat:MealPlan:percentOfProtein} jest wymagany
        \item Atrybut \ref{kat:MealPlan:percentOfFat} jest wymagany
        \item Atrybut \ref{kat:MealPlan:percentOfCarbohydrates} jest wymagany

        \item Atrybut \ref{kat:MealPlan:creationTimestamp} jest stemplem czasowym
        \item Atrybut \ref{kat:MealPlan:editTimestamp} jest stemplem czasowym
        \item Atrybut \ref{kat:MealPlan:name} jest ciągiem znaków o~długości od 1~do 255 znaków
        \item Atrybut \ref{kat:MealPlan:isVisible} jest typu logicznego
        \item Atrybut \ref{kat:MealPlan:numberOfDays} jest liczbą całkowitą nie mniejszą niż 1~i nie większą niż 30
        \item Atrybut \ref{kat:MealPlan:numberOfMealsPerDay} jest liczbą całkowitą nie mniejszą niż 1~i nie większą niż 10
        \item Atrybut \ref{kat:MealPlan:totalDailyEnergy} jest liczbą całkowitą nie mniejszą niż 1
        \item Atrybut \ref{kat:MealPlan:percentOfProtein} jest liczbą całkowitą nie mniejszą niż 0~i nie większą niż 100
        \item Atrybut \ref{kat:MealPlan:percentOfFat} jest liczbą całkowitą nie mniejszą niż 0~i nie większą niż 100
        \item Atrybut \ref{kat:MealPlan:percentOfCarbohydrates} jest liczbą całkowitą nie mniejszą niż 0~i nie większą niż 100
        \item Suma wartości atrybutów \ref{kat:MealPlan:percentOfProtein}, \ref{kat:MealPlan:percentOfFat}, \ref{kat:MealPlan:percentOfCarbohydrates} nie może przekraczać 100
    \end{enumerate}

    \item\ref{kat:MealPlanDay}
    \begin{enumerate}[label={\textbf{OGR/\protect\threedigits{\arabic{enumi}}}}, wide, labelwidth=!, align=left, leftmargin=3cm, resume]
        \item Atrybut \ref{kat:MealPlanDay:ordinalNumber} jest wymagany

        \item Atrybut \ref{kat:MealPlanDay:ordinalNumber} jest liczbą całkowitą nie mniejszą niż 1
    \end{enumerate}

    \item\ref{kat:Meal}
    \begin{enumerate}[label={\textbf{OGR/\protect\threedigits{\arabic{enumi}}}}, wide, labelwidth=!, align=left, leftmargin=3cm, resume]
        \item Atrybut \ref{kat:Meal:ordinalNumber} jest wymagany

        \item Atrybut \ref{kat:Meal:ordinalNumber} jest liczbą całkowitą nie mniejszą niż 1
    \end{enumerate}

    \item\ref{kat:MealRecipe}
    \begin{enumerate}[label={\textbf{OGR/\protect\threedigits{\arabic{enumi}}}}, wide, labelwidth=!, align=left, leftmargin=3cm, resume]
        \item Atrybut \ref{kat:MealRecipe:amount} jest wymagany

        \item Atrybut \ref{kat:MealRecipe:amount} jest liczbą całkowitą nie mniejszą niż 0
    \end{enumerate}

    \item\ref{kat:MealProduct}
    \begin{enumerate}[label={\textbf{OGR/\protect\threedigits{\arabic{enumi}}}}, wide, labelwidth=!, align=left, leftmargin=3cm, resume]
        \item Atrybut \ref{kat:MealProduct:amount} jest wymagany

        \item Atrybut \ref{kat:MealProduct:amount} jest liczbą rzeczywistą nie mniejszą niż 0
    \end{enumerate}

    \item\ref{kat:MealDefinition}
    \begin{enumerate}[label={\textbf{OGR/\protect\threedigits{\arabic{enumi}}}}, wide, labelwidth=!, align=left, leftmargin=3cm, resume]
        \item Atrybut \ref{kat:MealDefinition:ordinalNumber} jest wymagany
        \item Atrybut \ref{kat:MealDefinition:timeOfMeal} jest wymagany
        \item Atrybut \ref{kat:MealDefinition:percentOfEnergy} jest wymagany

        \item Atrybut \ref{kat:MealDefinition:ordinalNumber} jest liczbą całkowitą nie mniejszą niż 1
        \item Atrybut \ref{kat:MealDefinition:timeOfMeal} jest ciągiem znaków w~postaci HH:MI
        \item Atrybut \ref{kat:MealDefinition:percentOfEnergy} jest liczbą całkowitą nie mniejszą niż 0~i nie większą niż 100
        \item Suma wartości wszystkich atrybutów \ref{kat:MealDefinition:percentOfEnergy} w~jednym jadłospisie musi być równa 100
    \end{enumerate}

    \item\ref{kat:Appointment}
    \begin{enumerate}[label={\textbf{OGR/\protect\threedigits{\arabic{enumi}}}}, wide, labelwidth=!, align=left, leftmargin=3cm, resume]
        \item Atrybut \ref{kat:Appointment:appointmentDate} jest wymagany
        \item Atrybut \ref{kat:Appointment:appointmentState} jest wymagany

        \item Atrybut \ref{kat:Appointment:appointmentDate} jest stemplem czasowym
        \item Atrybut \ref{kat:Appointment:appointmentState} jest typu wyliczeniowego i~może przyjmować wartości "PLANNED", "CANCELED", "TOOK\_PLACE", "COMPLETED"
        \item Atrybut \ref{kat:Appointment:generalAdvice} jest ciągiem znaków
    \end{enumerate}

    \item\ref{kat:PatientCard}
    \begin{enumerate}[label={\textbf{OGR/\protect\threedigits{\arabic{enumi}}}}, wide, labelwidth=!, align=left, leftmargin=3cm, resume]
        \item Atrybut \ref{kat:PatientCard:creationDate} jest wymagany

        \item Atrybut \ref{kat:PatientCard:creationDate} jest stemplem czasowym
    \end{enumerate}

    \item\ref{kat:AppointmentEvaluation}
    \begin{enumerate}[label={\textbf{OGR/\protect\threedigits{\arabic{enumi}}}}, wide, labelwidth=!, align=left, leftmargin=3cm, resume]
        \item Atrybut \ref{kat:AppointmentEvaluation:overallSatisfaction} jest wymagany
        \item Atrybut \ref{kat:AppointmentEvaluation:dietitianServiceSatisfaction} jest wymagany
        \item Atrybut \ref{kat:AppointmentEvaluation:mealPlanOverallSatisfaction} jest wymagany
        \item Atrybut \ref{kat:AppointmentEvaluation:mealCostSatisfaction} jest wymagany
        \item Atrybut \ref{kat:AppointmentEvaluation:mealPreparationTimeSatisfaction} jest wymagany
        \item Atrybut \ref{kat:AppointmentEvaluation:mealComplexityLevelSatisfaction} jest wymagany
        \item Atrybut \ref{kat:AppointmentEvaluation:mealTastefulnessSatisfaction} jest wymagany
        \item Atrybut \ref{kat:AppointmentEvaluation:dietaryResultSatisfaction} jest wymagany

        \item Atrybut \ref{kat:AppointmentEvaluation:overallSatisfaction} jest typu wyliczeniowego i~może przyjmować wartości "VERY\_DISSATISFIED", "DISSATISFIED", "NEUTRAL", "SATISFIED", "VERY\_SATISFIED"
        \item Atrybut \ref{kat:AppointmentEvaluation:dietitianServiceSatisfaction} jest typu wyliczeniowego i~może przyjmować wartości "VERY\_DISSATISFIED", "DISSATISFIED", "NEUTRAL", "SATISFIED", "VERY\_SATISFIED"
        \item Atrybut \ref{kat:AppointmentEvaluation:mealPlanOverallSatisfaction} jest typu wyliczeniowego i~może przyjmować wartości "VERY\_DISSATISFIED", "DISSATISFIED", "NEUTRAL", "SATISFIED", "VERY\_SATISFIED"
        \item Atrybut \ref{kat:AppointmentEvaluation:mealCostSatisfaction} jest typu wyliczeniowego i~może przyjmować wartości "VERY\_DISSATISFIED", "DISSATISFIED", "NEUTRAL", "SATISFIED", "VERY\_SATISFIED"
        \item Atrybut \ref{kat:AppointmentEvaluation:mealPreparationTimeSatisfaction} jest typu wyliczeniowego i~może przyjmować wartości "VERY\_DISSATISFIED", "DISSATISFIED", "NEUTRAL", "SATISFIED", "VERY\_SATISFIED"
        \item Atrybut \ref{kat:AppointmentEvaluation:mealComplexityLevelSatisfaction} jest typu wyliczeniowego i~może przyjmować wartości "VERY\_DISSATISFIED", "DISSATISFIED", "NEUTRAL", "SATISFIED", "VERY\_SATISFIED"
        \item Atrybut \ref{kat:AppointmentEvaluation:mealTastefulnessSatisfaction} jest typu wyliczeniowego i~może przyjmować wartości "VERY\_DISSATISFIED", "DISSATISFIED", "NEUTRAL", "SATISFIED", "VERY\_SATISFIED"
        \item Atrybut \ref{kat:AppointmentEvaluation:dietaryResultSatisfaction} jest typu wyliczeniowego i~może przyjmować wartości "VERY\_DISSATISFIED", "DISSATISFIED", "NEUTRAL", "SATISFIED", "VERY\_SATISFIED"
        \item Atrybut \ref{kat:AppointmentEvaluation:comment} jest ciągiem znaków
    \end{enumerate}

    \item\ref{kat:BodyMeasurement}
    \begin{enumerate}[label={\textbf{OGR/\protect\threedigits{\arabic{enumi}}}}, wide, labelwidth=!, align=left, leftmargin=3cm, resume]
        \item Atrybut \ref{kat:BodyMeasurement:completionDate} jest wymagany
        \item Atrybut \ref{kat:BodyMeasurement:height} jest wymagany
        \item Atrybut \ref{kat:BodyMeasurement:weight} jest wymagany
        \item Atrybut \ref{kat:BodyMeasurement:waist} jest wymagany

        \item Atrybut \ref{kat:BodyMeasurement:completionDate} jest stemplem czasowym
        \item Atrybut \ref{kat:BodyMeasurement:height} jest liczbą całkowitą
        \item Atrybut \ref{kat:BodyMeasurement:weight} jest liczbą całkowitą
        \item Atrybut \ref{kat:BodyMeasurement:waist} jest liczbą rzeczywistą
        \item Atrybut \ref{kat:BodyMeasurement:percentOfFatTissue} jest liczbą rzeczywistą nie mniejszą niż 0~i nie większą niż 100
        \item Atrybut \ref{kat:BodyMeasurement:percentOfWater} jest liczbą rzeczywistą nie mniejszą niż 0~i nie większą niż 100
        \item Atrybut \ref{kat:BodyMeasurement:muscleMass} jest liczbą rzeczywistą
        \item Atrybut \ref{kat:BodyMeasurement:physicalMark} jest liczbą rzeczywistą
        \item Atrybut \ref{kat:BodyMeasurement:calciumInBones} jest liczbą rzeczywistą
        \item Atrybut \ref{kat:BodyMeasurement:basicMetabolism} jest liczbą całkowitą
        \item Atrybut \ref{kat:BodyMeasurement:metabolicAge} jest liczbą rzeczywistą
        \item Atrybut \ref{kat:BodyMeasurement:visceralFatLevel} jest liczbą rzeczywistą
    \end{enumerate}

    \item\ref{kat:NutritionalInterview}
    \begin{enumerate}[label={\textbf{OGR/\protect\threedigits{\arabic{enumi}}}}, wide, labelwidth=!, align=left, leftmargin=3cm, resume]
        \item Atrybut \ref{kat:NutritionalInterview:completionDate} jest wymagany
        \item Atrybut \ref{kat:NutritionalInterview:targetWeight} jest wymagany
        \item Atrybut \ref{kat:NutritionalInterview:advicePurpose} jest wymagany
        \item Atrybut \ref{kat:NutritionalInterview:physicalActivity} jest wymagany

        \item Atrybut \ref{kat:NutritionalInterview:completionDate} jest stemplem czasowym
        \item Atrybut \ref{kat:NutritionalInterview:targetWeight} jest liczbą całkowitą
        \item Atrybut \ref{kat:NutritionalInterview:advicePurpose} jest ciągiem znaków
        \item Atrybut \ref{kat:NutritionalInterview:physicalActivity} jest typu wyliczeniowego i~może przyjmować wartości "EXTREMELY\_INACTIVE", "SEDENTARY", "MODERATELY\_ACTIVE", "VIGOROUSLY\_ACTIVE", "EXTREMELY\_ACTIVE"
        \item Atrybut \ref{kat:NutritionalInterview:diseases} jest ciągiem znaków
        \item Atrybut \ref{kat:NutritionalInterview:medicines} jest ciągiem znaków
        \item Atrybut \ref{kat:NutritionalInterview:jobType} jest typu wyliczeniowego i~może przyjmować wartości "SITTING", "STANDING", "MIXED"
        \item Atrybut \ref{kat:NutritionalInterview:likedProducts} jest ciągiem znaków
        \item Atrybut \ref{kat:NutritionalInterview:dislikedProducts} jest ciągiem znaków
        \item Atrybut \ref{kat:NutritionalInterview:foodAllergies} jest ciągiem znaków
        \item Atrybut \ref{kat:NutritionalInterview:foodIntolerances} jest ciągiem znaków
    \end{enumerate}

    \item\ref{kat:CustomNutritionalInterviewQuestion}
    \begin{enumerate}[label={\textbf{OGR/\protect\threedigits{\arabic{enumi}}}}, wide, labelwidth=!, align=left, leftmargin=3cm, resume]
        \item Atrybut \ref{kat:CustomNutritionalInterviewQuestion:question} jest wymagany

        \item Atrybut \ref{kat:CustomNutritionalInterviewQuestion:ordinalNumber} jest liczbą całkowitą nie mniejszą niż 1
        \item Atrybut \ref{kat:CustomNutritionalInterviewQuestion:question} jest ciągiem znaków
        \item Atrybut \ref{kat:CustomNutritionalInterviewQuestion:answer} jest ciągiem znaków
    \end{enumerate}

    \item\ref{kat:CustomNutritionalInterviewQuestionTemplate}
    \begin{enumerate}[label={\textbf{OGR/\protect\threedigits{\arabic{enumi}}}}, wide, labelwidth=!, align=left, leftmargin=3cm, resume]
        \item Atrybut \ref{kat:CustomNutritionalInterviewQuestionTemplate:question} jest wymagany

        \item Atrybut \ref{kat:CustomNutritionalInterviewQuestionTemplate:question} jest ciągiem znaków
    \end{enumerate}

    \item\ref{kat:AssignedMealPlan}
    \begin{enumerate}[label={\textbf{OGR/\protect\threedigits{\arabic{enumi}}}}, wide, labelwidth=!, align=left, leftmargin=3cm, resume]
        \item Atrybut \ref{kat:AssignedMealPlan:assigmentTime} jest wymagany

        \item Atrybut \ref{kat:AssignedMealPlan:assigmentTime} jest stemplem czasowym
    \end{enumerate}
\end{itemize}

\section{Model domenowy}\label{sec:domainModel}
\noindent
\todo{opisać dodane atrybuty i~pomocnicze tabele asocjacyjne}
\image{0.7}{../uml/class_diagrams/dataTypes.png}{Diagram klas - typy danych}{class-diagram:data-types}
\image{0.7}{../uml/class_diagrams/gateway.png}{Diagram klas - brama aplikacji}{class-diagram:gateway}
\image{0.7}{../uml/class_diagrams/products.png}{Diagram klas - produkty}{class-diagram:products}
\image{0.7}{../uml/class_diagrams/recipes.png}{Diagram klas - przepisy}{class-diagram:recipes}
\image{0.7}{../uml/class_diagrams/mealplans.png}{Diagram klas - jadłospisy}{class-diagram:mealplans}
\image{0.7}{../uml/class_diagrams/appointments.png}{Diagram klas - wizyty}{class-diagram:appointments}

\section{Opis podstawowej architektury systemu}\label{sec:basicArchitecture}
\todo{Opisać, że to aplikacja webowa w~architekturze mikroserwisów
Wyszczególnienie modułów;
Diagram rozmieszczenia, wzorce projektowe}

%https://martinfowler.com/eaaDev/TemporalObject.html
%https://microservices.io/patterns/decomposition/decompose-by-subdomain.html
%https://dzone.com/articles/design-patterns-for-microservices
\thispagestyle{normal}
