% !TeX spellcheck = pl_PL
% --- Strona ze streszczeniem i~abstraktem ------------------------------------------------------------------
\addtocontents{toc}{\protect\setcounter{tocdepth}{-1}}
\chapter*{Streszczenie} % po polsku
Celem pracy było opracowanie systemu do zarządzania dietą w~architekturze mikroserwisów.
Aby osiągnąć ten cel przeprowadzono analizę istniejących rozwiązań konkurencyjnych, przedstawiono niezbędną wiedzę domenową oraz porównano popularne style architektury aplikacji.
Na podstawie zgromadzonej wiedzy wyszczególniono niezbędne założenia projektowe, zaprojektowano interfejs oraz zdefiniowano kategorie danych wraz z~regułami i~ograniczeniami ich dotyczącymi.
Następnie przedstawiono opis implementacji stworzonej na podstawie opracowanego projektu.
W implementacji kluczową rolę odegrały języki Java i~TypeScript, platforma deweloperska JHipster oraz stos technologii Netflix OSS dla architektury mikroserwisów.
Stworzone rozwiązanie może zostać wykorzystane przez dietetyków w~celu przeprowadzania kompleksowej obsługi wizyty pacjenta z~położeniem szczególnego nacisku na układanie jadłospisów i~udostępnianie go pacjentom.


% Kilka sztuczek, żeby:
%~- Abstract pojawił się na tej samej stronie co Streszczenie
%~- Abstract nie pojawił się w~spisie treści
\addtocontents{toc}{\protect\setcounter{tocdepth}{-1}}
\begingroup
\renewcommand{\cleardoublepage}{}
\renewcommand{\clearpage}{}
\chapter*{Abstract} % ...i to samo po angielsku
The aim of this work was to develop a~diet management system based on microservice architecture.
To achieve that goal, an analysis of existing competitive solutions was performed, the necessary domain knowledge was presented, and popular application architecture styles were compared.
Based on the accumulated knowledge, the necessary design assumptions were specified, the interface was designed, and categories of data were defined along with the rules and restrictions concerning them.
Then a~description of the implementation based on the developed project was presented.
The key role in the implementation was played by languages Java and TypeScript, the JHipster development platform and Netflix OSS technology stack for a~microservices architecture.
The created solution can be used by dietitians in order to conduct comprehensive service of the patient's visit with particular emphasis on designing the meal plans and sharing them with patients.
\endgroup
\addtocontents{toc}{\protect\setcounter{tocdepth}{2}}
% --- Koniec strony ze streszczeniem i~abstraktem -----------------------------------------------------------
