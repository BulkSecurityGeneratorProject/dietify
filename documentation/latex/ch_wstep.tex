% !TeX spellcheck = pl_PL
\chapter*{Wstęp}\label{ch:admission}

\section*{Opis problemu}\label{sec:problem-description}

Obserwując trendy występujące we współczesnym społeczeństwie można zauważyć, że zdrowy styl życia stał się modny, a~czasem nawet utożsamiany ze statusem społecznym.
W związku z~tą tendencją coraz więcej ludzi regularnie uprawia sport, rezygnuje z~używek, a~także dba o~dietę.
Analizując dane wyszukiwania hasła "dietetyk" rys. \ref{fig:dietetyk-trend} poprzez narzędzie Google Trends\cite{url:google-trends} można zauważyć, że popularność wyszukiwania tego hasła w~latach 2016-2019 jest ponad 5~krotnie większa niż w~roku 2004.
\imagewide[\cite{url:google-trends}]{img/dietetyk-trend.jpg}{Zainteresowanie hasłem "dietetyk" w~ujęciu czasowym}{dietetyk-trend}

\todo{zacytować: http://www.ejournals.eu/pliki/art/5960/}

Zwiększone zainteresowanie usługami dietetycznymi powoduje zwiększone zapotrzebowanie na wysokiej jakości, nowoczesne narzędzia wspomagające pracę dietetyka.
W chwili pisania niniejszej pracy w~Polsce popularność zyskało jedynie kilka programów oferujących kompleksowe funkcjonalności potrzebne w~codziennej praktyce dietetyka, można więc sądzić, że rynek aplikacji tego typu nie został jeszcze nasycony.

\par
Na podstawie wywiadu z~dietetykiem ustalono kluczowe aspekty pracy w~tej profesji dotyczące przede wszystkim współpracy z~pacjentami i~układania dla nich jadłospisów.
Przede wszystkim wyszczególniono następujące domeny:
\begin{itemize}
    \item zarządzanie produktami spożywczymi, ich wartościami odżywczymi i~miarami domowymi
    \item zarządzanie przepisami wraz z~przypisanymi do nich produktami
    \item zarządzanie jadłospisami wraz z~przypisanymi do nich przepisami i~produktami
    \item zarządzanie kartami pacjentów i~wizytami oraz pomiarami ciała, wywiadami żywieniowymi i~jadłospisami przydzielanymi do wizyty
\end{itemize}

Dodatkowo podczas wstępnej analizy wymagań technicznych zauważono, potrzebę zdefiniowane dodatkowej domeny technicznej agregującej akcje wykonywane przez administratora systemu/

\section*{Cel pracy}\label{sec:thesis-goal}

Celem pracy jest projekt i~budowa platformy do zarządzania dietą w~oparciu o~architekturę mikroserwisów.
Tworzona platforma będzie obejmowała cały cykl życia diety, czyli przede wszystkim:
zebranie przez dietetyka wywiadu żywieniowego od pacjenta,
stworzenie przez dietetyka jadłospisu,
udostępnienie jadłospisu pacjentowi
i elementy pomagające pacjentowi stosować dietę.

\section*{Zakres pracy}\label{sec:scope-of-work}

Praca w~swoim zakresie będzie zawierała
opracowanie projektu systemu, w~ramach którego, między innymi, przygotowane zostaną diagramy UML takie jak diagram przypadków użycia, diagram klas i~diagram rozmieszczenia.
Przygotowana zostanie również implementacja w~oparciu o~języki Java\cite{tech:java} i~TypeScript\cite{tech:typescript} oraz o~stos technologii Netflix OSS\cite{tech:netflix-oss} dla architektury mikroserwisów.
Na koniec zostanie przedstawiona dokumentacja kodu oraz pokrótce omówiony zostanie sposób instalacji i~korzystania z~systemu.

\thispagestyle{normal}
