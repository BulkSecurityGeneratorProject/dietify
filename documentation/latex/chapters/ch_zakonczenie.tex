% !TeX spellcheck = pl_PL
\chapter*{Zakończenie}\label{ch:ending}
Rezultatem wykonanych prac jest platforma wspomagająca zarządzanie dietą spełniająca kryteria biznesowe postawione w fazie analizy wymagań.
Dietetycy mogą wykorzystać opracowane rozwiązanie do ułatwienia układania i udostępniania swoim pacjentom jadłospisów,
a także do kontroli rezultatów stosowania diety przez pacjentów.

\par
Należy jednak wziąć pod uwagę, że oprogramowanie to ma na celu przede wszystkim ułatwienie pracy dietetyka
i zakłada się, że użytkownicy korzystający z aplikacji będą mieli specjalistyczną wiedzę w dziedzinie dietetyki.

\par
Oprogramowanie zostało zaprojektowane w sposób ułatwiający skalowanie,
a zastosowane wzorce i dobre praktyki programistyczne sprawią, że wdrażanie nowych funkcjonalności w systemie nie będzie stanowiło problemu.
Do funkcjonalności, które mogłyby zostać wdrożone w systemie w przyszłości można zaliczyć przede wszystkim;
\begin{itemize}
    \item zapewnienie możliwości korzystania z aplikacji bez połączenia z internetem poprzez dodanie funkcjonalności progresywnej aplikacji webowej (ang. Progressive Web App - PWA)\cite{url:pwa}
    \item zapewnienie możliwości instalacji aplikacji na urządzeniach mobilnych poprzez integrację z platformą Ionic\cite{tech:ionic}
    \item umożliwienie pacjentom korzystania z aplikacji
    \item zapewnienie możliwości komunikacji pacjenta z dietetykiem z wykorzystaniem komunikatora zintegrowanego z opracowywanym systemem
\end{itemize}

\par
Wdrożenie aplikacji komercyjnie i szersza współpraca z ekspertami domenowymi z dziedziny dietetyki z pewnością pokazałyby wiele innych możliwości rozwoju opracowanej platformy,
jednakże może to wymagać zainwestowania znacznych środków finansowych.
\thispagestyle{normal}
