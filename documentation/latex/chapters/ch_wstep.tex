% !TeX spellcheck = pl_PL
\chapter*{Wstęp}\label{ch:admission}

\section*{Opis problemu}\label{sec:problem-description}

Przed przejściem do analizy problemu, warto zdefiniować co można rozumieć pod pojęciem diety.
Według definicji z~Encyklopedii Powszechnej PWN, dieta to "system odżywiania z~ustaleniem jakości i~ilości pokarmów,
dostosowany do potrzeb organizmu"\cite{book:encyklopedia-dieta}.
Opierając się na tej definicji, można opisać system do zarządzania dietą jako system,
który będzie ułatwiał dietetykowi układanie jadłospisów dostosowanych do potrzeb żywieniowych organizmu pacjenta, pozwalał na zarządzanie ułożonymi jadłospisami
i udostępnianie ich pacjentowi w~formie umożliwiającej w~jak najprostszy sposób zastosowanie przez pacjenta przygotowanej dla niego diety,
a także umożliwiał kontrolę rezultatów stosowania jadłospisu przez pacjenta.

\par
Obserwując trendy występujące we współczesnym społeczeństwie można zauważyć, że zdrowy styl życia stał się modny, a~czasem nawet utożsamiany ze statusem społecznym.
W związku z~tą tendencją coraz więcej osób regularnie uprawia sport, rezygnuje z~używek, a~także dba o~dietę.
Analizując przedstawione na rysunku \ref{fig:dietetyk-trend} dane dotyczące wyszukiwania hasła "dietetyk" w~latach 2004-2019 poprzez narzędzie Google Trends\cite{url:google-trends} można zauważyć,
że popularność wyszukiwania tego hasła w~latach 2016-2019 jest ponad 5~krotnie większa niż w~roku 2004.

\imagewide[\cite{url:google-trends}]{img/dietetyk-trend.jpg}{Zainteresowanie hasłem "dietetyk" w~ujęciu czasowym}{dietetyk-trend}

Zwiększone zainteresowanie usługami dietetycznymi powoduje zwiększone zapotrzebowanie na wysokiej jakości, nowoczesne narzędzia wspomagające pracę dietetyka.
W chwili pisania niniejszej pracy w~Polsce popularność zyskało jedynie kilka programów oferujących kompleksowe funkcjonalności potrzebne w~codziennej praktyce dietetyka,
można więc sądzić, że rynek aplikacji tego typu nie został jeszcze nasycony.

\par
Biorąc pod uwagę powyższe spostrzeżenia, warto w~tym miejscu podkreślić, że Światowa Organizacja Zdrowia (ang. World Health Organization~- WHO) określiła otyłość
(i bardziej ogólnie choroby dietozależne) jako jeden z~głównych problemów zdrowia publicznego\cite{article:dietetyk-na-rynku-uslug-medycznych}.
Fakt ten podkreśla jak duże brzemię odpowiedzialności spoczywa na ramionach dietetyków,
a~co za tym idzie jak istotne jest dostarczenie specjalistom narzędzi ułatwiających niesienie pomocy pacjentom.

\section*{Cel pracy}\label{sec:thesis-goal}

Celem pracy jest projekt i~budowa platformy do zarządzania dietą w~oparciu o~architekturę mikroserwisów.
Opracowywana platforma będzie obejmowała cały proces zarządzania dietą, czyli przede wszystkim:
zebranie przez dietetyka wywiadu żywieniowego od pacjenta,
ułożenie przez dietetyka jadłospisu,
udostępnienie jadłospisu pacjentowi,
ułatwienie pacjentowi stosowania diety poprzez generowanie listy zakupów
oraz kontrolę rezultatów diety.

\section*{Zakres pracy}\label{sec:scope-of-work}

Praca w~swoim zakresie zawiera opracowanie projektu systemu, w~ramach którego, między innymi,
omówieni zostaną użytkownicy systemu wraz z~ich potrzebami,
przeprowadzona zostanie dekompozycja problemu w~oparciu o~poddziedziny,
przygotowane zostaną prototypy interfejsu i~diagramy UML takie jak diagram przypadków użycia i~diagram klas.
Omówiona zostanie również podstawowa architektura systemu.
\par
W części praktycznej pracy analizie poddane będą wybrane narzędzia i~technologie wykorzystane do implementacji projektu,
omówiony zostanie zakres implementacji, a~także przedstawiony będzie rezultat tejże implementacji w~formie prezentacji działania aplikacji.
Przedstawiona zostanie również dokumentacja kodu i~testy oraz pokrótce omówiony będzie sposób instalacji systemu z~uwzględnieniem wymagań wstępnych.

\thispagestyle{normal}
