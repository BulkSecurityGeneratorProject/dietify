% !TeX spellcheck = pl_PL
\chapter*{Wstęp}\label{ch:admission}

\section*{Opis problemu}\label{sec:problem-description}

Zanim będzie można rozpocząć analizę problemu, warto zdefiniować co można rozumieć pod pojęciem diety.
Według definicji z~Encyklopedii PWN, dieta to "system odżywiania z~ustaleniem jakości i~ilości pokarmów, 
dostosowany do potrzeb organizmu"\cite{book:pwn-dietetyk}. \inlinetodo{fix reference to pwn}
Opierając się na tej definicji, można opisać system do zarządzania dietą jako system, 
który będzie ułatwiał dietetykowi tworzenie jadłospisów dostosowanych do potrzeb żywieniowych organizmu pacjenta, zarządzanie stworzonymi jadłospisami 
oraz pozwalał udostępniać utworzony jadłospis pacjentowi w~formie umożliwiającej w~jak najprostszy sposób zastosowanie przez pacjenta przygotowanej dla niego diety.

\par
Obserwując trendy występujące we współczesnym społeczeństwie można zauważyć, że zdrowy styl życia stał się modny, a~czasem nawet utożsamiany ze statusem społecznym.
W związku z~tą tendencją coraz więcej ludzi regularnie uprawia sport, rezygnuje z~używek, a~także dba o~dietę.
Analizując dane wyszukiwania hasła "dietetyk" rys. \ref{fig:dietetyk-trend} poprzez narzędzie Google Trends\cite{url:google-trends} można zauważyć, 
że popularność wyszukiwania tego hasła w~latach 2016-2019 jest ponad 5~krotnie większa niż w~roku 2004.
\imagewide[\cite{url:google-trends}]{img/dietetyk-trend.jpg}{Zainteresowanie hasłem "dietetyk" w~ujęciu czasowym}{dietetyk-trend}

Zwiększone zainteresowanie usługami dietetycznymi powoduje zwiększone zapotrzebowanie na wysokiej jakości, nowoczesne narzędzia wspomagające pracę dietetyka.
W chwili pisania niniejszej pracy w~Polsce popularność zyskało jedynie kilka programów oferujących kompleksowe funkcjonalności potrzebne w~codziennej praktyce dietetyka, 
można więc sądzić, że rynek aplikacji tego typu nie został jeszcze nasycony.

\par
Biorąc pod uwagę powyższe spostrzeżenia, warto w~tym miejscu podkreślić, że Światowa Organizacja Zdrowia określiła otyłość 
(i bardziej ogólnie choroby dietozależne) jako jeden z~głównych problemów zdrowia publicznego\cite{article:dietetyk-na-rynku-uslug-medycznych}.
Fakt ten podkreśla jak duże brzemię odpowiedzialności spoczywa na ramionach dietetyków, 
a~co za tym idzie jak istotne jest dostarczenie specjalistom dietetyki narzędzi ułatwiających niesienie pomocy pacjentom.

\section*{Cel pracy}\label{sec:thesis-goal}

Celem pracy jest projekt i~budowa platformy do zarządzania dietą w~oparciu o~architekturę mikroserwisów.
Tworzona platforma będzie obejmowała cały cykl życia diety, czyli przede wszystkim:
zebranie przez dietetyka wywiadu żywieniowego od pacjenta,
stworzenie przez dietetyka jadłospisu,
udostępnienie jadłospisu pacjentowi
i elementy pomagające pacjentowi stosować dietę.

\section*{Zakres pracy}\label{sec:scope-of-work}

Praca w~swoim zakresie będzie zawierała
opracowanie projektu systemu, w~ramach którego, między innymi, przygotowane zostaną diagramy UML takie jak diagram przypadków użycia, diagram klas i~diagram rozmieszczenia.
Przygotowana zostanie również implementacja w~oparciu o~języki Java i~TypeScript oraz o~stos technologii Netflix OSS dla architektury mikroserwisów.
Na koniec zostanie przedstawiona dokumentacja kodu oraz pokrótce omówiony zostanie sposób instalacji i~korzystania z~systemu.

\thispagestyle{normal}
