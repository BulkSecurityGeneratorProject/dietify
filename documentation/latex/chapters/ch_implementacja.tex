% !TeX spellcheck = pl_PL
\chapter{Implementacja}\label{ch:implementation}
\section{Wykorzystywane środowiska i~narzędzia programistyczne}\label{sec:dev-tools}
\todo{uzupełnić, opisać, zacytować narzędzia}
Podczas wyboru języków programowania, z~użyciem których zostanie zaimplementowany system, postawiono następujące kryteria:

\begin{itemize}
    \item Ścisła kontrola typów
    \item Dobre wsparcie dla paradygmatu programowania obiektowego
    \item Niezależność języka od platformy
\end{itemize}

Wybrane języki spełniające te kryteria to:

\begin{itemize}
    \item W~warstwie backendu Java\cite{tech:java}
    \item W~warstwie frontendu Typescript\cite{tech:typescript}
\end{itemize}

Powyższy wybór zaowocował decyzją o~zastosowaniu Angulara\cite{tech:angular} jako wiodącego frameworka frontendowego
i~Springa\cite{tech:spring} jako wiodącego frameworka backendowego.
Frameworki te cieszą się bardzo dużą popularnością, a~ich dojrzałość sprawia,
że nadają się zarówno przy tworzeniu niewielkich aplikacji jak i~systemów klasy enterprise.
\todo{angular, spring}

\par
\todo{html, bootstrap, bootswatch}

\par
System został zaprojektowany tak, żeby wykorzystać cechy relacyjnych baz danych,
więc podczas wyboru systemu zarządzania bazą danych pod uwagę wzięto tylko relacyjne bazy danych.
Rozważano przede wszystkim systemy PostegreSQL\cite{tech:postgresql} i~MySQL\cite{tech:mysql}.
Z~punktu widzenia funkcjonalności potrzebnych w~implementowanej aplikacji oba systemy systemy zarządzania relacyjną bazą danych (ang. Relational Database Management System - RDBMS) wypadają równie dobrze,
jednakże ostatecznie wybrano PostgreSQL ze względu na mniej restrykcyjną licencję wykorzystania systemu nawet w~rozwiązaniach komercyjnych o~zamkniętym kodzie.
\todo{liquibase}
\par
\todo{Netflix oss, jhipster}

\par
\todo{docker, docker compose, gitlab pipelines}
%\begin{enumerate}
%    \item Backend
%    \begin{itemize}
%        \item Java\cite{tech:java}
%        \item Spring\cite{tech:spring}
%        \item Gradle\cite{tech:gradle}
%    \end{itemize}
%
%    \item Frontend
%    \begin{itemize}
%        \item Typescript\cite{tech:typescript}
%        \item Angular\cite{tech:angular}
%        \item HTML\cite{tech:html}
%        \item Sass\cite{tech:sass}
%        \item Bootstrap\cite{tech:bootstrap}
%        \item Bootswatch Flatly\cite{tech:bootswatch}
%        \item Font Awesome\cite{tech:font-awesome}
%    \end{itemize}
%
%    \item Baza danych
%    \begin{itemize}
%        \item PostgreSQL\cite{tech:postgresql}
%        \item Hibernate\cite{tech:hibernate}
%        \item Liquibase\cite{tech:liquibase}
%    \end{itemize}
%
%    \item Testy
%    \begin{itemize}
%        \item JUnit\cite{tech:junit}
%        \item Mockito\cite{tech:mockito}
%        \item Jest\cite{tech:jest}
%        \item Protractor\cite{tech:protractor}
%    \end{itemize}
%
%    \item Netflix OSS\cite{tech:netflix-oss}
%    \begin{itemize}
%        \item Eureka Service Discovery\cite{tech:netflix-eureka}
%        \item Zuul Proxy\cite{tech:netflix-zuul}
%        \item Ribbon load balancer\cite{tech:netflix-ribbon}
%        \item Hystrix circuit breaker\cite{tech:netflix-hystrix}
%    \end{itemize}
%    \item Inne
%    \begin{itemize}
%        \item JHipster\cite{tech:jhipster}
%        \item Gitlab\cite{tech:gitlab}
%        \item Docker\cite{tech:docker}
%        \item Docker Compose\cite{tech:docker-compose}
%        \item Elasticsearch\cite{tech:elasticsearch}
%    \end{itemize}
%\end{enumerate}
\section{Architektura systemu}\label{sec:system-architecture}
\todo{opisać stack netflix oss}
\todo{diagram rozmieszczenia}
\section{Instalacja oprogramowania}\label{sec:software-installation}
\subsection{Wymagania wstępne}\label{subsec:prerequirements}
Przed przystąpieniem do wykonywania kolejny kroków należy się upewnić, że następujące narzędzia są zainstalowane:
\begin{itemize}
    \item Open JDK 11 (https://adoptopenjdk.net/?variant=openjdk11)
    \item Node.js 10 lub nowsza wersja LTS (https://nodejs.org/en/)
    \item Docker 19.03 + Docker Compose 2 (https://docs.docker.com/install/)
\end{itemize}

\subsection{Instalacja}\label{subsec:installation}

Aby zbudować i~uruchomić projekt na Dockerze należy z~poziomu głównego katalogu projektu
wykonać polecenia przedstawione na listingu \ref{listing:komilacja-i-uruchomienie}.
\begin{listing}[h!]
    \begin{minted}{bash}
        cd gateway
        npm install
        sh gradlew bootJar -Pprod jibDockerBuild
        cd ../products
        sh gradlew bootJar -Pprod jibDockerBuild
        cd ../recipes
        sh gradlew bootJar -Pprod jibDockerBuild
        cd ../mealplans
        sh gradlew bootJar -Pprod jibDockerBuild
        cd ../appointments
        sh gradlew bootJar -Pprod jibDockerBuild
        cd ../docker-compose
        sh docker-compose up
    \end{minted}
    \caption{Skrypt kompilujący wszystkie mikroserwisy i~uruchamiający aplikację na Dockerze (opr. wł.)} \label{listing:komilacja-i-uruchomienie}
\end{listing}

Alternatywnie, dla celów deweloperskich można zastosować uproszczony proces nie wykorzystujący Dockera.
W tym celu należy najpierw uruchomić JHipster Registry wykonując polecenie z~poziomu głównego katalogu projektu
wykonać polecenia przedstawione na listingu \ref{listing:service-discovery}.
\begin{listing}[h!]
    \begin{minted}{bash}
        cd service-discovery
        java -jar ./jhipster-registry-5.0.2.jar --spring.profiles.active=dev --spring.security.user.password=admin --spring.cloud.config.server.composite.0.type=git --spring.cloud.config.server.composite.0.uri= https://github.com/jhipster/jhipster-registry-sample-config
    \end{minted}
    \caption{Uruchamianie JHipster Registry (opr. wł.)} \label{listing:service-discovery}
\end{listing}

Następnie z~poziomu katalogu każdego z~serwisów (gateway, products, recipes, mealplans, appointments)
należy wykonać polecenie uruchamiające Gradle Wrapper przedstawione na listingu \ref{listing:run-gradle-wrapper}.
\begin{listing}[h!]
    \begin{minted}{bash}
        ./gradlew
    \end{minted}
    \caption{Uruchamianie Gradle Wrapper (opr. wł.)} \label{listing:run-gradle-wrapper}
\end{listing}

Po uruchomieniu wszystkich serwisów aplikacja będzie dostępna pod adresem \textit{localhost:8080}.

\section{Prezentacja aplikacji}\label{sec:app-presentation}
\todo{podstawowy opis poruszania się po aplikacji, zrzuty ekranu z~kilku najważniejszych widoków}
\todo{implementacja w~kodzie: obliczanie podstawowych wartości odżywczych w~przepisie, wyświetlanie spełnienia norm odżywczych w~jadłospisie, wersjonowanie produktów i~przepisów, generowanie listy zakupów i~jadłospisu do wydruku, wykres BMI}

\section{Dokumentacja kodu}\label{sec:code-documentation}
\todo{javadoc, swagger, jdl}
\section{Testy}\label{sec:tests}
\todo{przykładowe testy jednostkowe i~integracyjne}
\todo{testy użyteczności}
\thispagestyle{normal}
