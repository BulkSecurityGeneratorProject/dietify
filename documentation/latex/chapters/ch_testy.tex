% !TeX spellcheck = pl_PL
\chapter{Testy}
\section{Wprowadzenie}
Tworząc oprogramowanie istotne jest żeby akcje biznesowe i~inne operacje wykonywane w~systemie były realizowane w~określony sposób i~przynosiły oczekiwane rezultaty.
Wiąże się z~tym pojęcie zapewniania jakości, czyli przede wszystkim "spełnienie lub przekroczenie wymagań klienta"\cite{book:jakosc-projektow-informatycznych}.
Podstawowym narzędziem pozwalającym na zapewnienie jakości oprogramowania jest przeprowadzanie testów.
Według sylabusa Międzynarodowej Rady Kwalifikacji Testów Oprogramowania (ang. International Software Testing Qualifications Board~- ISQB)\cite{url:istqb-syllabus} testowanie przeprowadza się w~celu:
\begin{itemize}
    \item znajdowania i~zapobiegania błędom,
    \item zdobywania pewności odnośnie poziomu jakości,
    \item sprawdzania czy akcje biznesowe realizowane są w~oczekiwany sposób,
    \item zapewniania informacji dotyczących bieżącego stanu implementacji.
\end{itemize}

\par
W ramach niniejszej pracy opisane zostaną następujące rodzaje testów:
\begin{itemize}
    \item \textbf{jednostkowe}~- testujące logikę biznesową,
    \item \textbf{integracyjne}~- testujące punkty końcowe API,
    \item \textbf{użyteczności}~- testy przeprowadzane z~użytkownikami końcowymi sprawdzające użyteczność systemu.
\end{itemize}

\par
Pierwsze dwa rodzaje testów mogą być przeprowadzane automatycznie, dlatego postanowiono wdrożyć rozwiązanie zapewniające ciągłą integrację (ang. Continuous Integration~- CI) z~wykorzystaniem platformy Gitlab CI\cite{tech:gitlab-pipelines}.
Dzięki temu w~momencie zwracania kodu do repozytorium automatycznie wykonywany jest zdefiniowany proces CI (ang. pipeline), zawierający następujące etapy:
\begin{itemize}
    \item \textbf{build}~- kompilacja kodu,
    \item \textbf{unit tests}~- wykonanie testów jednostkowych,
    \item \textbf{integration tests}~- wykonanie testów integracyjnych,
    \item \textbf{package}~- pakowanie skompilowanego kodu do plików *.jar.
\end{itemize}

\par
Rezultat wykonania procesu ciągłej integracji na platformie Gitlab CI został przedstawiony na rysunku \ref{fig:gitlab-pipeline}.

\imagewide[\cite{tech:gitlab-pipelines-dietify}]{img/pipeline.png}{Gitlab Pipeline}{gitlab-pipeline}

\section{Testy jednostkowe}

Robert C. Martin stwierdził, że dobrze napisane testy jednostkowe zwiększają elastyczność kodu produkcyjnego,
ułatwiają wprowadzanie zmian w~kodzie i~pozwalają szybko wykryć zaistniałe błędy\cite{book:czysty-kod}.

\par
Przedmiotem testów jednostkowych jest logika akcji biznesowych przeprowadzanych w~obrębie warstwy serwisów.
Należy jednak zauważyć, że wątpliwą wartość przynosi testowanie funkcji,
których jedyną odpowiedzialnością jest przekierowanie żądania z~warstwy punktów końcowych API do warstwy repozytoriów.

\par
Testy jednostkowe, jak nazwa wskazuje, powinny testować jednostkę taką jak funkcja i~powinny daną jednostkę testować w~izolacji od zależności zewnętrznych\cite{book:testy-jednostkowe}.
Aby osiągnąć taką izolację, zastosowano platformę Mockito dzięki czemu możliwe jest tworzenie makiet dla zależności zewnętrznych i~definiowanie rezultatów obcowania z~nimi.
W rezultacie osiągnięta może być ścisła kontrola nad przepływem informacji w~obrębie testowanej jednostki.

\par
Na listingu \ref{listing:unit-test} przedstawiony został przykładowy test jednostkowy.

\noindent\hspace{.075\textwidth}\begin{minipage}{.85\textwidth}
\begin{minted}{java}
@RunWith(MockitoJUnitRunner.class)
public class ExampleUnitTest {
  @Mock private UserService userService;
  @Mock private ProductRepository productRepository;
  @Mock private CacheManager cacheManager;
  @Mock private ProductSubcategoryService productSubcategoryService;
  @Mock EntityManager entityManager;
  @InjectMocks private ProductServiceImpl productService;
  private User user;
  private Product product;

  @Before
  public void setup() {
    long id = 1;
    this.user = UserCreator.createEntity();
    this.user.setId(id);
    this.product = ProductCreator.createEntity(entityManager);
    this.product.setId(id);
    this.product.setAuthor(user);
    when(userService.getCurrentUser())
      .thenReturn(Optional.of(this.user));
    when(productRepository.findOneWithEagerRelationships(any()))
      .thenReturn(Optional.of(this.product));
  }

  @Test
  public void authorShouldBeAbleToEditOwnProduct() {
    //when
    productService.save(product);
    //then
    Mockito.verify(productRepository, times(1)).saveAndFlush(this.product);
  }
}
\end{minted}
\begin{lstlisting}[caption={Przykładowy test jednostkowy \source{\ownwork}}, label={listing:unit-test}]
\end{lstlisting}
\end{minipage}

\section{Testy integracyjne}

W przeciwieństwie do testów jednostkowych testy integracyjne nie są wykonywane w~całkowitej izolacji\cite{book:testy-jednostkowe}.
Testowaniu podlegają punkty końcowe API z~wykorzystaniem rzeczywistego połączenia z~bazą danych.
Na potrzeby testów tworzona jest instancja osadzonej bazy danych H2\cite{tech:h2-db} oraz inicjalizowany jest kontekst aplikacji Spring Boot.

\par
Celem przeprowadzania tego typu testów jest sprawdzenie czy punkty końcowe API w~poprawny sposób obsługują przychodzące żądania
z uwzględnieniem przeprowadzenia akcji biznesowych w~warstwie serwisów i~perzystencji w~warstwie repozytoriów.

\par
Na listingu \ref{listing:integration-test} przedstawiony został przykładowy test integracyjny.

\noindent\hspace{.075\textwidth}\begin{minipage}{.85\textwidth}
\begin{minted}{java}
@SpringBootTest(classes = RecipesApp.class)
public class ExampleIntegrationTest {
  @Autowired private DishTypeRepository dishTypeRepository;
  @Autowired private DishTypeService dishTypeService;
  @Autowired private DishTypeSearchRepository mockDishTypeSearchRepository;
  @Autowired private MappingJackson2HttpMessageConverter jacksonMessageConverter;
  @Autowired private PageableHandlerMethodArgumentResolver pageableArgumentResolver;
  @Autowired private ExceptionTranslator exceptionTranslator;
  @Autowired private EntityManager em;
  @Autowired private Validator validator;

  private MockMvc restDishTypeMockMvc;

  @BeforeEach
  public void setup() {
    MockitoAnnotations.initMocks(this);
    final DishTypeResource dishTypeResource = new DishTypeResource(dishTypeService);
    this.restDishTypeMockMvc = MockMvcBuilders.standaloneSetup(dishTypeResource)
      .setCustomArgumentResolvers(pageableArgumentResolver)
      .setControllerAdvice(exceptionTranslator)
      .setConversionService(createFormattingConversionService())
      .setMessageConverters(jacksonMessageConverter)
      .setValidator(validator).build();
  }

  @Test
  @Transactional
  public void getNonExistingDishType() throws Exception {
    // Get the dishType
    restDishTypeMockMvc.perform(get("/api/dish-types/{id}", Long.MAX_VALUE))
      .andExpect(status().isNotFound());
  }
}
\end{minted}
\begin{lstlisting}[caption={Przykładowy test integracyjny \source{\ownwork}}, label={listing:integration-test}]
\end{lstlisting}
\end{minipage}

\section{Testy użyteczności}

Oprogramowanie zwykle nie jest tworzone, żeby istniało w~próżni.
Należy więc zbadać w~jakim stopniu aplikacja może być używana przez rzeczywistych użytkowników.
W tym celu przeprowadzane są testy użyteczności\cite{book:testowanie-i-jakosc-oprogramowania}.

\par
W ramach realizacji tej grupy testów, 9 potencjalnych użytkowników (tj. osoby studiujące dietetykę lub zajmujące się profesjonalnie dietetyką)
zostało poproszonych o~zasymulowanie przeprowadzania kompleksowej wizyty pacjenta
i wyrażenie opinii o~używalności systemu poprzez odpowiedź na pytania ustandaryzowanego formularza Skali Używalności Systemu (ang. System Usability Scale - SUS)\cite{url:sus}.

\par
Korzystając z~SUS, uczestnik badania udziela odpowiedzi na 10 pytań w~pięciostopniowej skali od "Bardzo się zgadzam" do "Bardzo się nie zgadzam".
Rezultaty następnie są konwertowane na wartość liczbową z~zakresu 0-4, a~suma otrzymanych punktów jest mnożona przez 2.5, żeby otrzymać wynik w~skali 0-100.
Wynik powyżej 68 punktów uznawany jest za ponadprzeciętny.
\par
Badania przeprowadzono za pomocą odpowiednio przygotowanegu formularza Google\cite{tech:google-forms}, czego rezultaty zostały przedstawione na rysunku \ref{fig:sus-results}.
%Na rysunku \ref{fig:sus-form} przedstawiony został wykorzystany kwestionariusz SUS.
%
%\imagewide[\cite{url:sus-form}]{img/kwestionariusz-sus.png}{Kwestionariusz SUS}{sus-form}
%
%\todo{rezultaty sus}
%Na rysunku \ref{fig:sus-results} przedstawione zostały rezultaty przeprowadzonego badania.

\imagewide{img/sus_results.png}{Rezultaty badania SUS}{sus-results}

Punktowa wartość wyników wachała się od 65 pkt do 82,5 pkt, a średni uzyskany wynik wyniósł 73,6 pkt.
Według skali przyjmowanej w SUS jest to wynik ponadprzeciętny.
Osoby badane wyraziły się dość pozytywnie o stworzonym systemie.
Przede wszystkim podkreślano, że jest łatwy w użyciu, a różne funkcje systemu są łatwo dostępne.
Jednakże warto zwrócić uwagę, że większość osób nie czuła się zbyt pewnie podczas korzystania z systemu,
a część osób stwierdziła, że system jest w pewnym stopniu kłopotliwy w użyciu.

\par
Podsumowując, można stwierdzić, że system jest przystępny w użytkowaniu, jednak można by wprowadzić pewne udogodnienia,
takie jak na przykład możliwość przeciągania produktów i przepisów pomiędzy daniami podczas układania jadłospisu.

\thispagestyle{normal}
