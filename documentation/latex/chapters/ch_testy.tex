% !TeX spellcheck = pl_PL
\chapter{Testy}
\section{Wprowadzenie}
Tworząc oprogramowanie istotne jest żeby akcje biznesowe i~inne operacje wykonywane w~systemie były realizowane w~określony sposób i~przynosiły oczekiwane rezultaty.
Wiąże się z~tym pojęcie zapewniania jakości, czyli przede wszystkim "spełnienie lub przekroczenie wymagań klienta"\cite{book:jakosc-projektow-informatycznych}.
Podstawowym narzędziem pozwalającym na zapewnienie jakości oprogramowania jest przeprowadzanie testów.
Według sylabusa Międzynarodowej Rady Kwalifikacji Testów Oprogramowania (ang. International Software Testing Qualifications Board~- ISQB)\cite{url:istqb-syllabus} testowanie przeprowadza się w~celu:
\begin{itemize}
    \item znajdowania i~zapobiegania błędom,
    \item zdobywania pewności odnośnie poziomu jakości,
    \item sprawdzania czy akcje biznesowe realizowane są w~oczekiwany sposób,
    \item zapewniania informacji dotyczących bieżącego stanu implementacji.
\end{itemize}

\par
W ramach niniejszej pracy opisane zostaną następujące rodzaje testów:
\begin{itemize}
    \item \textbf{jednostkowe}~- testujące logikę biznesową,
    \item \textbf{integracyjne}~- testujące punkty końcowe API,
    \item \textbf{użyteczności}~- testy przeprowadzane z~użytkownikami końcowymi sprawdzające użyteczność systemu.
\end{itemize}

\par
Pierwsze dwa rodzaje testów mogą być przeprowadzane automatycznie, dlatego postanowiono wdrożyć rozwiązanie zapewniające ciągłą integrację (ang. Continuous Integration~- CI) z~wykorzystaniem platformy Gitlab CI\cite{tech:gitlab-pipelines}.
Dzięki temu w~momencie zwracania kodu do repozytorium automatycznie wykonywany jest zdefiniowany proces CI (ang. pipeline), zawierający następujące etapy:
\begin{itemize}
    \item \textbf{build}~- kompilacja kodu,
    \item \textbf{unit tests}~- wykonanie testów jednostkowych,
    \item \textbf{integration tests}~- wykonanie testów integracyjnych,
    \item \textbf{package}~- pakowanie skompilowanego kodu do plików *.jar.
\end{itemize}

\par
Rezultat wykonania procesu CI na platformie Gitlab CI został przedstawiony na rysunku \ref{fig:gitlab-pipeline}.

\todo{screenshoot pipeline z~gitlaba}

\section{Testy jednostkowe}

Robert C. Martin stwierdził, że dobrze napisane testy jednostkowe zwiększają elastyczność kodu produkcyjnego, ułatwiają wprowadzanie zmian w kodzie i pozwalają szybko wykryć zaistniałe błędy\cite{book:czysty-kod}.

\todo{testy jednostkowe}\cite{book:testy-jednostkowe}
\section{Testy integracyjne}
\todo{testy integracyjne}\cite{book:testy-jednostkowe}
\section{Testy użyteczności}
\todo{testy użyteczności}\cite{book:testowanie-i-jakosc-oprogramowania}

\thispagestyle{normal}
