% !TeX spellcheck = pl_PL
\chapter{Testy}
\section{Wprowadzenie}
Tworząc oprogramowanie istotne jest żeby akcje biznesowe i~inne operacje wykonywane w~systemie były realizowane w~określony sposób i~przynosiły oczekiwane rezultaty.
Wiąże się z~tym pojęcie zapewniania jakości, czyli przede wszystkim "spełnienie lub przekroczenie wymagań klienta"\cite{book:jakosc-projektow-informatycznych}.
Podstawowym narzędziem pozwalającym na zapewnienie jakości oprogramowania jest przeprowadzanie testów.
Według sylabusa Międzynarodowej Rady Kwalifikacji Testów Oprogramowania (ang. International Software Testing Qualifications Board~- ISQB)\cite{url:istqb-syllabus} testowanie przeprowadza się w~celu:
\begin{itemize}
    \item znajdowania i~zapobiegania błędom,
    \item zdobywania pewności odnośnie poziomu jakości,
    \item sprawdzania czy akcje biznesowe realizowane są w~oczekiwany sposób,
    \item zapewniania informacji dotyczących bieżącego stanu implementacji.
\end{itemize}

\par
W ramach niniejszej pracy opisane zostaną następujące rodzaje testów:
\begin{itemize}
    \item \textbf{jednostkowe}~- testujące logikę biznesową,
    \item \textbf{integracyjne}~- testujące punkty końcowe API,
    \item \textbf{użyteczności}~- testy przeprowadzane z~użytkownikami końcowymi sprawdzające użyteczność systemu.
\end{itemize}

\par
Pierwsze dwa rodzaje testów mogą być przeprowadzane automatycznie, dlatego postanowiono wdrożyć rozwiązanie zapewniające ciągłą integrację (ang. Continuous Integration~- CI) z~wykorzystaniem platformy Gitlab CI\cite{tech:gitlab-pipelines}.
Dzięki temu w~momencie zwracania kodu do repozytorium automatycznie wykonywany jest zdefiniowany proces CI (ang. pipeline), zawierający następujące etapy:
\begin{itemize}
    \item \textbf{build}~- kompilacja kodu,
    \item \textbf{unit tests}~- wykonanie testów jednostkowych,
    \item \textbf{integration tests}~- wykonanie testów integracyjnych,
    \item \textbf{package}~- pakowanie skompilowanego kodu do plików *.jar.
\end{itemize}

\par
Rezultat wykonania procesu CI na platformie Gitlab CI został przedstawiony na rysunku \ref{fig:gitlab-pipeline}.

\todo{screenshoot pipeline z~gitlaba}

\section{Testy jednostkowe}

Robert C. Martin stwierdził, że dobrze napisane testy jednostkowe zwiększają elastyczność kodu produkcyjnego,
ułatwiają wprowadzanie zmian w kodzie i pozwalają szybko wykryć zaistniałe błędy\cite{book:czysty-kod}.

\par
Przedmiotem testów jednostkowych jest logika akcji biznesowych przeprowadzanych w obrębie warstwy serwisów.
Należy jednak zauważyć, że wątpliwą wartość przynosi testowanie funkcji,
których jedyną odpowiedzialnością jest przekierowanie żądania z warstwy punktów końcowych API do warstwy repozytoriów.

\par
Testy jednostkowe, jak nazwa wskazuje, powinny testować jednostkę taką jak funkcja i powinny daną jednostkę testować w izolacji od zależności zewnętrznych\cite{book:testy-jednostkowe}.
Aby osiągnąć taką izolację, zastosowano platformę Mockito dzięki czemu możliwe jest tworzenie makiet dla zależności zewnętrznych i definiowanie rezultatów obcowania z nimi.
W rezultacie osiągnięta może być ścisła kontrola nad przepływem informacji w obrębie testowanej jednostki.

\todo{przykładowy test jednostkowy}

\section{Testy integracyjne}

W przeciwieństwie do testów jednostkowych testy integracyjne nie są wykonywane w całkowitej izolacji\cite{book:testy-jednostkowe}.
Testowaniu podlegają punkty końcowe API z wykorzystaniem rzeczywistego połączenia z bazą danych.
Na potrzeby testów tworzona jest instancja osadzonej bazy danych H2\cite{tech:h2-db} oraz inicjalizowany jest kontekst aplikacji Spring Boot.

\par
Celem przeprowadzania tego typu testów jest sprawdzenie czy punkty końcowe API w poprawny sposób obsługują przychodzące żądania
z uwzględnieniem przeprowadzenia akcji biznesowych w warstwie serwisów i perzystencji w warstwie repozytoriów.

\todo{przykładowy test integracyjny}

\section{Testy użyteczności}

Oprogramowanie zwykle nie jest tworzone, żeby istniało w próżni.
Należy więc zbadać w jakim stopniu aplikacja może być używana przez rzeczywistych użytkowników.
W tym celu przeprowadzane są testy użyteczności\cite{book:testowanie-i-jakosc-oprogramowania}.

\par
W ramach realizacji tej grupy testów, 7 potencjalnych użytkowników (tj. osoby studiujące dietetykę lub zajmujące się profesjonalnie dietetyką)
zostało poproszonych o zasymulowanie przeprowadzania kompleksowej wizyty pacjenta
i wyrażenie opinii o używalności systemu poprzez odpowiedź na pytania ustandaryzowanego formularza Skali Używalności Systemu (ang. System Usability Scale - SUS)\cite{url:sus}.
Korzystając z SUS, uczestnik badania udziela odpowiedzi na 10 pytań w pięciostopniowej skali od "Bardzo się zgadzam" do "Bardzo się nie zgadzam".
Rezultaty następnie są konwertowane na wartość liczbową z zakresu 0-4, a suma otrzymanych punktów jest mnożona przez 2.5, żeby otrzymać wynik w skali 0-100.
Wynik powyżej 68 punktów uznawany jest za ponadprzeciętny.
Na rysunku \ref{fig:sus-form} przedstawiony został wykorzystany kwestionariusz SUS.

\imagewide[\cite{url:sus-form}]{img/kwestionariusz-sus.png}{Kwestionariusz SUS}{sus-form}

\todo{rezultaty sus}

\thispagestyle{normal}
