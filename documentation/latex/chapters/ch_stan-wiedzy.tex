% !TeX spellcheck = pl_PL
\chapter{Stan wiedzy i~techniki w~zakresie tematyki pracy}\label{ch:knowladge-state}
\section{Przegląd istniejących rozwiązań konkurencyjnych}\label{sec:competitive-solutions}
Na rynku Polskim funkcjonuje zaledwie kilka narzędzi w~kompleksowy sposób wspomagających pracę dietetyka.
W niniejszej sekcji zostaną omówione systemy cieszące się największą popularnością oraz przedstawione zostanie zbiorcze porównanie ich najważniejszych cech.
W przypadku większości z~porównywanych programów warunki korzystania z~usługi pozwalają na rejestrację w~systemie jedynie wykwalifikowanym dietetykom.
Z tego względu wszystkie analizowane dane zostały zebrane z~publicznie dostępnych źródeł,
takich jak strony odpowiednich programów, blogi internetowe oraz filmy promocyjne na platformie YouTube\cite{url:youtube}.
Dodatkowo, żeby rozwiać wątpliwości czy dane rozwiązanie konkurencyjne umożliwia korzystanie z~kluczowych funkcjonalności bez dostępu do internetu, przeprowadzono stosowną korespondencję z~obsługą klienta poszczególnych programów.
%\todo{sprawdzić czy korzystanie ze screenshotów z~youtube'a jest legalne}
\begin{itemize}
    \item TiqDiet

        TiqDiet\cite{url:TiqDiet} jest wygodnym w~użyciu programem, który pozwala dietetykom w~intuicyjny sposób projektować jadłospisy i~udostępniać je pacjentom.
        W~celu uproszczenia pracy dietetyka, udostępnianych jest wiele szablonów jadłospisów, które łatwo można dostosować do indywidualnych potrzeb pacjenta.
        Pacjent może odbierać ułożoną dietę poprzez responsywną stronę internetową, aplikację mobilną oraz za pomocą inteligentnego zegarka.
        Aplikacje mobilne pozwalają ponadto na automatyczne przypominanie pacjentowi m.in. o~konieczności zażycia suplementu oraz o~konieczności regularnego picia wody.
        Komunikacja pomiędzy pacjentem, a~dietetykiem może odbywać się w~czasie rzeczywistym za pomocą zintegrowanego komunikatora.
        Ponadto dietetyk ma możliwość obserwowania postępów pacjentów w~stosowaniu diety, a~w razie potrzeby może zalecać wizytę u~lekarza czy też zażycie dodatkowych suplementów.
        Na rysunku \ref{fig:competitive-solution:TiqDiet} przedstawiono przykładowy widok edycji planu dnia w~aplikacji TiqDiet

        \imagewide[\cite{url:TiqDiet-youtube}]{img/competitive-solutions/TiqDiet.png}{TiqDiet}{competitive-solution:TiqDiet}

    \item Kcalmar PRO

        Kcalmar\cite{url:kcalmar} jest systemem, którego głównym założeniem jest maksymalne skrócenie czasu potrzebnego na ułożenie programu żywieniowego dopasowanego do potrzeb pacjenta.
        Zapewnia zaawansowany system podpowiedzi ułatwiający projektowanie zbilansowanej diety z~wyraźnym oznaczeniem alergenów czy zduplikowanych potraw.
        Jadłospisy mogą być automatycznie skalowane z~automatycznym przeliczeniem miar domowych.
        Co ciekawe system pozwala również na wyszukiwanie dietetyków w~wybranych miastach i~filtrowanie ich według typów diet i~jednostek chorobowych, w~których się specjalizują.
        Na rysunku \ref{fig:competitive-solution:kcalmar} pokazano widok wygenerowanej listy zakupów dla jadłospisu zaprojektowanego w~aplikacji Kcalmar.

        \imagewide[\cite{url:kcalmar-youtube}]{img/competitive-solutions/kcalmar.png}{Kcalmar PRO}{competitive-solution:kcalmar}

    \item Dietetyk Pro

        Program Dietetyk Pro\cite{url:dietetyk-pro} na tle konkurencji wyróżnia się tym, że poza główną funkcjonalnością układania jadłospisu,
        abonenci mogą również korzystać ze szkoleń eksperckich i~literatury dietetycznej dostępnej w~ramach platformy.
        Dodatkowo dietetycy po wykupieniu subskrypcji mogą skorzystać ze zdalnej pomocy z~obsługi programu.
        Ciekawym udogodnieniem jest możliwość szerokiej konfiguracji ekranu startowego, np. poprzez dodanie kalkulatora wartości odżywczych czy też wyświetlanie listy zaplanowanych wizyt.
        Spośród porównywanych programów Dietetyk Pro posiada największe bazy produktów i~przepisów wyprzedzając konkurencję niemal dwukrotnie.
        Na rysunku \ref{fig:competitive-solution:dietetyk-pro} przedstawiono widok edycji 4-dniowego jadłospisu w~aplikacji Dietetyk Pro

        \imagewide[\cite{url:dietetyk-pro-youtube}]{img/competitive-solutions/dietetyk-pro.png}{Dietetyk Pro}{competitive-solution:dietetyk-pro}

    \item Aliant

        Do grupy klientów docelowych programu Aliant\cite{url:aliant} należą zarówno dietetycy jak również trenerzy personalni.
        Tak jak inne porównywane aplikacje, główną funkcjonalnością programu Aliant jest układanie jadłospisów,
        jednakże w~przeciwieństwie do konkurencji, aplikacja jest dostępna tylko jako aplikacja na platformę Windows.
        Nie ma możliwości dostępu do systemu przez stronę internetową, jednak możliwe jest automatyczne dodawanie z~internetu zewnętrznych baz produktów po zaakceptowaniu ich licencji wykorzystania.
        Brakuje również zintegrowanego narzędzia do komunikacji z~pacjentami, a~udostępnianie ułożonego jadłospisu musi odbywać się w~całości poza systemem.
        Na rysunku \ref{fig:competitive-solution:aliant} zaprezentowano widok edycji tygodniowego jadłospisu w~programie Aliant.
        Warto zauważyć, że dla każdego dnia w~sposób graficzny przedstawiono czy zostały spełnione zaplanowane dzienne normy ilości podstawowych składników odżywczych.

        \imagewide[\cite{url:aliant-youtube}]{img/competitive-solutions/aliant.png}{Aliant}{competitive-solution:aliant}

    \item Dietico

        Program Dietico\cite{url:dietico} szczyci się stale powiększaną bazą przepisów i~produktów.
        Pozwala na układanie jadłospisu za pomocą wygodnego interfejsu,
        a~przejrzysty system podpowiedzi pozwala szybko wykryć powtarzające się dania oraz produkty, na które pacjent jest uczulony.
        Twórcy programu podkreślają, że ich program wyróżnia możliwość uwzględnienia w~układanej diecie posiadanego przez pacjenta wyposażenia kuchennego i~sezonowych produktów spożywczych.
        Na rysunku \ref{fig:competitive-solution:dietico} przedstawiono widok edycji tygodniowego jadłospisu wraz z~graficznymi oznaczeniami produktów, na które pacjent jest uczulony.

        \imagewide[\cite{url:dietico-youtube}]{img/competitive-solutions/dietico.png}{Dietico}{competitive-solution:dietico}

    \item Vitme

        Program Vitme\cite{url:vitme} umożliwia prowadzenie kart pacjentów, projektowanie jadłospisów oraz generowanie wydruków w~formacie PDF.
        Program w~porównaniu z~konkurencją jest oferowany w~bardzo korzystnej cenie oraz posiada bogatą bazę produktów, jednak stosunkowo niedużą bazę przepisów.
        Do wad produktu można zaliczyć przestarzały i~mało przejrzysty interfejs, który może zniechęcić niektórych potencjalnych klientów.
        Na rysunku \ref{fig:competitive-solution:vitme} pokazano przykładowy widok kalendarza wizyt w~aplikacji Vitme.

        \imagewide[\cite{url:vitme-youtube}]{img/competitive-solutions/vitme.png}{Vitme}{competitive-solution:vitme}

\end{itemize}

W tabeli \ref{tabela:rozwiazania-konkurencyjne-funkcjonalne} przedstawiono porównanie najważniejszych cech funkcjonalnych,
a~w~tabeli \ref{tabela:rozwiazania-konkurencyjne-niefunkcjonalne} cech niefunkcjonalnych
6 istniejących na rynku rozwiązań konkurencyjnych\cite{url:porownanie-programow-dietetycznych}.
Warto zwrócić uwagę, że funkcjonalności takie jak możliwość wykorzystania gotowych szablonów diet, wysyłanie diety do pacjenta,
przeprowadzanie wywiadu żywieniowego czy automatyczne generowanie listy zakupów nie występują w~niektórych spośród analizowanych systemów.
Na tej podstawie można wysunąć hipotezę, że te funkcjonalności~- mino iż istotne~- nie są kluczowe w~systemie wspomagającym pracę dietetyka.
Natomiast możliwość układania jadłospisów z~wykorzystaniem własnych produktów i~przepisów,
zapisywanie ich do plików oraz przypisywanie do stosownych kart pacjenta są oczekiwane w~tego typu aplikacjach.

\begin{minipage}{\textwidth}
    \begin{table}[H]
        \centering\caption{Rozwiązania konkurencyjne~- cechy funkcjonalne (źródło: \ownwork)\label{tabela:rozwiazania-konkurencyjne-funkcjonalne}}
        \begin{tabular}{|P{.22\textwidth}|P{.09\textwidth}|P{.09\textwidth}|P{.09\textwidth}|P{.09\textwidth}|P{.09\textwidth}|P{.09\textwidth}|}
            \hline
                                                           & \cellgray{TiqDiet}    & \cellgray{Kcalmar Pro}    & \cellgray{Dietetyk Pro}  & \cellgray{Aliant}          & \cellgray{Dietico}    & \cellgray{Vitme}    \\ \hline
            \cellgray{Układanie jadłospisów}               & \cellgreen{TAK}       & \cellgreen{TAK}           & \cellgreen{TAK}          & \cellgreen{TAK}            & \cellgreen{TAK}       & \cellgreen{TAK}     \\ \hline
            \cellgray{Gotowe szablony diet}                & \cellgreen{TAK}       & \cellgreen{TAK}           & \cellgreen{TAK}          & \cellgreen{TAK}            & \cellred{NIE}         & \cellred{NIE}       \\ \hline
            \cellgray{Zapis diety do pliku}                & \cellgreen{TAK}       & \cellgreen{TAK}           & \cellgreen{TAK}          & \cellgreen{TAK}            & \cellgreen{TAK}       & \cellgreen{TAK}     \\ \hline
            \cellgray{Wysyłanie diet do pacjenta}          & \cellgreen{TAK}       & \cellgreen{TAK}           & \cellgreen{TAK}          & \cellgreen{TAK}            & \cellred{NIE}         & \cellgreen{TAK}     \\ \hline
            \cellgray{Komunikacja z~pacjentem}             & \cellgreen{TAK}       & \cellgreen{TAK}           & \cellgreen{TAK}          & \cellred{NIE}              & \cellred{NIE}         & \cellgreen{TAK}     \\ \hline
            \cellgray{Karta pacjenta}                      & \cellgreen{TAK}       & \cellgreen{TAK}           & \cellgreen{TAK}          & \cellgreen{TAK}            & \cellgreen{TAK}       & \cellgreen{TAK}     \\ \hline
            \cellgray{Wywiad żywieniowy}                   & \cellgreen{TAK}       & \cellgreen{TAK}           & \cellgreen{TAK}          & \cellred{NIE}              & \cellgreen{TAK}       & \cellgreen{TAK}     \\ \hline
            \cellgray{Lista zakupów}                       & \cellgreen{TAK}       & \cellgreen{TAK}           & \cellgreen{TAK}          & \cellgreen{TAK}            & \cellgreen{TAK}       & \cellred{NIE}       \\ \hline
            \cellgray{Dodawanie własnych produktów}        & \cellgreen{TAK}       & \cellgreen{TAK}           & \cellgreen{TAK}          & \cellgreen{TAK}            & \cellgreen{TAK}       & \cellgreen{TAK}     \\ \hline
            \cellgray{Dodawanie własnych przepisów}        & \cellgreen{TAK}       & \cellgreen{TAK}           & \cellgreen{TAK}          & \cellgreen{TAK}            & \cellgreen{TAK}       & \cellgreen{TAK}     \\ \hline
        \end{tabular}
    \end{table}
\end{minipage}

\begin{minipage}{\textwidth}
    \begin{table}[H]
        \centering\caption{Rozwiązania konkurencyjne~- cechy niefunkcjonalne (źródło: \ownwork)\label{tabela:rozwiazania-konkurencyjne-niefunkcjonalne}}
        \begin{tabular}{|P{.22\textwidth}|P{.09\textwidth}|P{.09\textwidth}|P{.09\textwidth}|P{.09\textwidth}|P{.09\textwidth}|P{.09\textwidth}|}
            \hline
                                                                                & \cellgray{TiqDiet}    & \cellgray{Kcalmar Pro}    & \cellgray{Dietetyk Pro}   & \cellgray{Aliant}         & \cellgray{Dietico}    & \cellgray{Vitme}      \\ \hline
            \cellgray{Liczba produktów w~bazie}                                 & 1000                  & 1400                      & 6000                      & 3500                      & 900                   & 5000                  \\ \hline
            \cellgray{Liczba gotowych przepisów}                                & 200                   & 800                       & 2800                      & 1700                      & 1900                  & 400                   \\ \hline
            \cellgray{Praca offline}                                            & \cellred{NIE}         & \cellred{NIE}             & \cellred{NIE}             & \cellgreen{TAK}           & \cellred{NIE}         & \cellred{NIE}         \\ \hline
            \cellgray{Praca online}                                             & \cellgreen{TAK}       & \cellgreen{TAK}           & \cellgreen{TAK}           & \cellred{NIE}             & \cellgreen{TAK}       & \cellgreen{TAK}       \\ \hline
            \cellgray{Aplikacja mobilna dla dietetyka}                          & \cellgreen{TAK}       & \cellgreen{TAK}           & \cellgreen{TAK}           & \cellred{NIE}             & \cellred{NIE}         & \cellred{NIE}         \\ \hline
            \cellgray{Aplikacja mobilna dla pacjenta}                           & \cellgreen{TAK}       & \cellgreen{TAK}           & \cellred{NIE}             & \cellred{NIE}             & \cellred{NIE}         & \cellgreen{TAK}       \\ \hline
            \cellgray{Dostęp dla pacjenta przez przeglądarkę internetową}       & \cellgreen{TAK}       & \cellgreen{TAK}           & \cellgreen{TAK}           & \cellred{NIE}             & \cellred{NIE}         & \cellgreen{TAK}       \\ \hline
            \cellgray{Darmowy okres testowy}                                    & 14dni                 & 14dni                     & 7dni                      & bezter- minowo            & 14dni                 & 14dni                 \\ \hline
            \cellgray{Cena w~abonamencie rocznym}                               & 199                   & 1188                      & 246                       & 699                       & 546                   & 219                   \\ \hline
        \end{tabular}
    \end{table}
\end{minipage}

\section{Przegląd literatury dietetycznej}\label{sec:domain-literature}

W rozdziale \ref{sec:competitive-solutions} dokonano przeglądu rozwiązań konkurencyjnych.
Na podstawie dokonanej analizy możliwe będzie zdefiniowanie głównych wymagań funkcjonalnych projektowanego systemu,
jednakże konieczne jest odwołanie się do literatury dziedzinowej, żeby potwierdzić zasadność przyjętych założeń istotnych z~punktu widzenia dietetyki.

\par
Pierwszym rozważanym pojęciem jest podstawowa przemiana materii (PPM).
Jest to poziom zapotrzebowania energetycznego organizmu znajdującego się w~stanie spoczynku (czyli minimalny poziom zapotrzebowania energetycznego)
wyznaczany na podstawie wieku i~masy ciała osoby.
Aby obliczyć wartość energetyczną posiłku należy wyznaczyć ekwiwalent metaboliczny (MET) podstawowych wartości odżywczych,
tj. białek, tłuszczy i~węglowodanów\cite{book:dietetyka-zywienie-zdrowego-i-chorego-czlowieka}.

\par
Kolejnym uwzględnianym współczynnikiem jest współczynnik poziomu aktywności fizycznej (ang. Physical Activity Level~- PAL).
Organizacja Narodów Zjednoczonych do spraw Wyżywienia i~Rolnictwa (ang. Food and Agriculture Organization of the United Nations~- FAO) definiuje 5 poziomów aktywności fizycznej\cite{url:fao-pal}:
\begin{itemize}
    \item brak aktywności fizycznej (wartość współczynnika 1,2~- 1,39),
    \item niska aktywność fizyczna (wartość współczynnika 1,4~- 1,69),
    \item umiarkowana aktywność fizyczna (wartość współczynnika 1,7~- 1,99),
    \item wysoka aktywność fizyczna (wartość współczynnika 2~- 2,4),
    \item bardzo wysoka aktywność fizyczna (wartość współczynnika > 2,4).
\end{itemize}

\par
Iloczyn PPM i~PAL określa stopień całkowitej przemiany materii (CPM)\cite{book:normy-zywienia-czlowieka}.
Przy czym dla prawidłowo zaplanowanej diety, dzienna energia powinna być dostarczana w~proporcjach:
\begin{itemize}
    \item ok 10\% z~białek,
    \item ok 60\% z~węglowodanów,
    \item ok 30\% z~tłuszczu.
\end{itemize}

Wymienione wyżej składniki tworzą najważniejszą grupę składników, tzw grupę składników energetycznych, ale należy zwrócić uwagę,
że do prawidłowego funkcjonowania organizmu definiuje się ok 40 innych niezbędnych składników należących do grup składników budulcowych
(głównie jod, wapń, lipidy, fosfor, żelazo i~siarka) i~regulujących (głównie witaminy, błonnik oraz mikro- i~makroelementy),
które również powinny być uwzględnione podczas układania zbilansowanej diety\cite{book:dietetyka-zywienie-zdrowego-i-chorego-czlowieka}.

\par
Podczas układania jadłospisu uwzględnia się podstawowe typy diet\cite{book:dietoterapia}:
\begin{itemize}
    \item podstawowa,
    \item kleikowa,
    \item papkowata,
    \item bogatopotasowa,
    \item niskosodowa,
    \item niskocholesterolowa,
    \item bezglutenowa,
    \item bogatoresztkowa,
    \item łatwostrawna,
    \item ubogotłuszczowa,
    \item bogatobiałkowa,
    \item ubogobiałkowa,
    \item bogatoenergetyczna,
    \item ubogoenergetyczna.
\end{itemize}

\par
Jak zauważono wyżej, podstawowym kryterium potrzebnym do skomponowania odpowiednio zbilansowanej diety jest odpowiednie dobranie wartości odżywczej produktów spożywczych.
Wartości te mogą być uzyskane z~tabel składu i~wartości odżywczej żywności.
Na rynku polskim tabele takie są odpłatnie udostępniane przez polski Instytut Żywności i~Żywienia (IŻŻ)\cite{book:tabele-wartosci-odzywczych},
jednakże licencja wspomnianego opracowania nie zezwala na wykorzystanie danych zawartych w~zestawieniu bez wykupienia odpowiedniego abonamentu\cite{url:izz-dostep-do-bazy}.

\par
Podobne rozwiązanie w~języku angielskim oferuje Departament Rolnictwa Stanów Zjednoczonych (ang. United States Department of Agriculture~- USDA) udostępniając całkowicie za darmo
do dowolnego użytku narodową bazę danych wartości odżywczych dla standardowych odwołań (ang. National Nutrient Database for Standard Reference)\cite{url:usda-sr-db} potocznie nazywana "bazą USDA".
Dane są udostępniane w~formie pliku bazy Microsoft Access. Baza zawiera:
\begin{itemize}
    \item ponad 7 tysięcy produktów spożywczych,
    \item ponad 600 tysięcy wartości odżywczych,
    \item ok 100 definicji wartości odżywczych,
    \item ponad 14 tysięcy miar domowych,
    \item ok 25 kategorii produktów spożywczych.
\end{itemize}

\par
Na koniec należałoby rozważyć wskaźniki pozwalające ocenić wpływ diety.
Podstawowe wyznaczane wartości to wskaźnik masy ciała (ang. Body Mass Index~- BMI), stosunek obwodu talii do obwodu bioder (ang. Waist to Hip Ratio~- WHR) oraz ilość tkanki tłuszczowej w~organizmie.
W celu wyznaczenia ilości tkanki tłuszczowej w~organizmie przeprowadza się pomiar fałdów skórno-tłuszczowych lub stosuje się bioelektryczną metodę impedancji (ang. bioelectrical impedance analysis~- BIA).
Metoda BIA polega na przepuszczeniu przez organizm ładunku elektrycznego o~natężeniu poniżej 1mA.
Metoda ta jest nieinwazyjna i~bezpieczne jest jej stosowanie w~każdym stanie zdrowia pacjenta.
Pozwala nie tylko na wyznaczenie ilości tkanki tłuszczowej, ale także ilości wody w~organizmie, wapnia w~kościach i~masy mięśniowej\cite{book:dietetyka-zywienie-zdrowego-i-chorego-czlowieka}.
% typy dań, posiłków, wyposażenie kuchenne

\section{Architektura mikroserwisów}\label{sec:usefull-technologies}

Projektowanie złożonych systemów informatycznych umożliwiających bezproblemowe jednoczesne korzystanie przez miliony użytkowników jest zadaniem niebanalnym.
W klasycznym podejściu, implementowano systemy w~architekturze monolitycznej.
Aplikacja napisana w~takiej architekturze jest samowystarczalna w~kontekście jej zachowania.
Może komunikować się z~zewnętrznymi usługami lub źródłami danych w~celu wykonania operacji,
ale logika biznesowa potrzebna do wykonania każdej operacji jest w~całości zawarta w~obrębie aplikacji.
W przypadku wystąpienia potrzeby skalowania horyzontalnego takiej aplikacji,
konieczne jest powielanie całej aplikacji na każdym z~serwerów\cite{url:microsoft-web-architectures}.

\par
Architektura mikroserwisów, zgodnie z~tym co sugeruje nazwa, skupia się na budowaniu aplikacji będącej zbiorem niewielkich,
luźno powiązanych serwisów komunikujących się ze sobą na przykład za pomocą protokołu HTTP czy AMQP.
Serwisy implementowane i~wdrażane są niezależnie od siebie\cite{book:dot-net-microservices}.
Efektem projektowania niezależnych serwisów jest skalowanie tylko serwisów, które tego wymagają,
co pozwala na optymalne wykorzystanie zasobów\cite{book:mastering-microservices-with-java9}.

\par
Architektura monolityczna ma wiele zalet\cite{book:microservices-patterns}, spośród których do najważniejszych należą:
\begin{itemize}
    \item prostota implementacji,
    \item możliwość łatwego przeprowadzania radykalnych zmian w~programie,
    \item prostota testowania,
    \item prostota wdrażania aplikacji na środowisko produkcyjne,
    \item prostota skalowania aplikacji.
\end{itemize}

\par
Martin Fowler podkreśla, że w~przypadku wielu aplikacji architektura monolityczna jest jak najbardziej wystarczająca,
a w~przypadku gdy system jest wystarczająco złożony, żeby użycie mikroserwisów przyniosło realny zysk, zwykle lepiej jest zacząć od monolitu,
a następnie przeprowadzić migrację do architektury mikroserwisów poprzez wydzielenie modułów w~obrębie monolitu
i~późniejsze przekształcanie ich w~niezależne serwisy\cite{url:monolith-first}.

\par
W przypadku aplikacji monolitycznej łatwo jest doprowadzić do sytuacji w~której poszczególne moduły są ze sobą ściśle powiązane,
co zwykle ma bardzo negatywny wpływ na wydajność aplikacji, utrudnia wprowadzanie zmian w~kodzie i~prowadzi do występowania trudnych do wykrycia błędów w~implementacji.
Sytuację, w~której w~aplikacji powstaje dużo przypadkowych powiązań i~zależności Vaughn Vernon określił mianem "Wielkiej Kuli Błota"\cite{book:ddd-kompendium}.

\par
Do głównych zalet zastosowania mikroserwisów należy stosowanie luźnego powiązania serwisów,
co poniekąd wymusza, żeby zależności pomiędzy serwisami były bardziej przemyślane i~lepiej zaprojektowane.
Z pomocą we właściwym zaprojektowaniu serwisów i~zależności między nimi przychodzą strategiczne wzorce DDD.
Jednym z~takich wzorców jest dekompozycja w~oparciu o~poddziedziny\cite{book:ddd-evans}.
Dziedzina systemu jest dzielona na poddziedziny poprzez zdefiniowanie przestrzeni problemów biznesowych w~obrębie względnie niezależnych obszarów specjalizacji.
W przypadku architektury mikroserwisów można wyznaczyć poszczególne serwisy poprzez zdefiniowanie poddziedzin systemu
i~zaprojektowanie serwisu dla każdej z~nich\cite{book:microservices-patterns}.

\thispagestyle{normal}
