% !TeX spellcheck = pl_PL
\pagenumbering{gobble}
\addtocontents{toc}{\protect\setcounter{tocdepth}{-1}}
\chapter*{Changelog}\label{ch:changelog}

Jest to pomocniczy rozdział opisujący zmiany w~kolejnych wersjach pracy wysyłanej do promotora, żeby ułatwić współpracę z~promotorem.
Zostanie usunięty przed ostatecznym oddaniem pracy.

\subsubsection{v3}
\begin{itemize}
%\item Uwzględniono sugestie z~komentarzy:
%    \begin{itemize}
%	\end{itemize}
\item Zmiany:
    \begin{itemize}
    	\item Zrezygnowałem z~wykorzystania wzorca TemporalObject, bo nie wyrobię się z~zaimplementowaniem go. Encje ProductVersion i~RecipeVersion zostały zmergowane ze swoimi rodzicami.
		\item Poprawiłem interpunkcję w~wyliczeniach.
		\item Uzupełniłem opis wykorzystywanych technologii
		\item Na prośbę promotora dodałem z~powrotem screenshooty rozwiązań konkurencyjnych
		\item Zmiana sposobu podawania źródła w~podpisach obrazków
	\end{itemize}
\item Nowości:
    \begin{itemize}
		\item implementacja~- opis architektury
		\item implemetnacja~- opis dokumentacji
		\item opis zakresu implementacji
		\item zakończenie
		\item wprowadzenie do testów
		\item opis testów jednostkowych, integracyjnych, użyteczności
	\end{itemize}
%\item Podsumowanie:
%    \begin{itemize}
%	\end{itemize}
\item Do zrobienia:
    \begin{itemize}
        \item uzupełnić testy
        \item prezentacja aplikacji
        \item wykorzystanie bazy usda
	\end{itemize}
\end{itemize}

\subsubsection{v2}
\begin{itemize}
\item Uwzględniono sugestie z~komentarzy:
    \begin{itemize}
	\item poprawiłem błędy wyszczególnione błędy ortograficzne, interpunkcyjne, stylistyczne
	\item W~kwestii rozwiązań konkurencyjnych i~pracy offline;napisałem maile do supportu wszystkich porównywanych rozwiązań z~pytaniem czy jest możliwa praca offline. Wszyscy odpisali, że nie. (oczywiście poza aliantem, który działa tylko offline)
	\item dodałem zgodę na przetwarzanie danych osobowych w~wymaganiach funkcjonalnych i~use casach
	\item diagramy przypadków użycia powiększyłem o~ok 30\% a~klas o~ok 10\%
	\item dodałem reguły w~latex'ie które powinny wyeliminować większość wdów i~sierot
	\item dodałem wersje przeglądarek/systemów w~wymaganiach niefunkcjonalnych
	\end{itemize}
\item Zmiany:
    \begin{itemize}
	\item krótkie streszczenie na stronie tytułowej
	\item przegląd rozwiązań konkurencyjnych
	\item uzupełniłem opis problemu we wstępie
	\item założenia projektowe: uzupełniłem słownik
	\item diagramy przypadków użycia przerzuciłem do sekcji z~wymaganiami funkcjonalnymi w~założeniach projektowych
	\item poprawiłem scope wymagań funkcjonalnych
	\item uprościłem diagramy przypadków użycia
	\item uzupełniłem prototypy interfejsu
	\item w~dodatku (jdl) zamiast całego opisu domeny zostawiłem tylko definicję mikroserwisów. odniesienie do dodatku będzie w~rozdziale implementacja
	\item kategorie, ograniczenia, reguły i~diagramy klas zostały rozdzielone na sekcje dla każdej poddziedziny w~celu zwiększenia czytelności
	\item wyrzuciłem ze scope'u mało istotne kategorie; głównie z~poddziedzin administracyjnej i~wizyt
	\item zmiana oznaczeń kategorii/ograniczeń/reguł z~KAT/XXX na KAT/Y/XX, gdzie Y~to numer poddziedziny
	\item drobne poprawki stylistyczne w~całej pracy
	\end{itemize}
\item Nowości:
    \begin{itemize}
	\item stan wiedzy i~technik: przegląd literatury dietetycznej, opis architektury mikroserwisów
	\item projekt: opis podstawowej architektury systemu
	\item projekt: projekt bazy danych~- wprowadzenie teoretyczne
	\item implementacja: instalacja oprogramowania
	\item podrozdział zakres implementacji, w~którym wspomniałem o~ochronie danych osobowych
	\end{itemize}
\item Podsumowanie:
    \begin{itemize}
	\item Chciałbym zamknąć już wszystko od początku pracy do końca opisu projektu (tj. do końca rozdziału 3.)
	\end{itemize}
\item Do zrobienia:
    \begin{itemize}
	\item opisać: architektura systemu, prezentacja aplikacji, dokumentacja kodu, testy, podsumowanie
	\end{itemize}
\end{itemize}

\cleardoublepage
\pagenumbering{gobble}
%\thispagestyle{normal}
